%==============================================================================
% PAPER A - EMERGENZA QUANTISTICA DALLA POTENZIALITÀ PRIMORDIALE
% Traduzione italiana — per lettura personale
% Originale: paper_A.tex (Physical Review A / Foundations of Physics)
%==============================================================================

\documentclass[aps,pra,11pt,notitlepage,nofootinbib,longbibliography]{revtex4-2}

%==============================================================================
% PACKAGES
%==============================================================================

\usepackage[utf8]{inputenc}
\usepackage[T1]{fontenc}
\usepackage[italian]{babel}
\usepackage{amsmath}
\usepackage{amssymb}
\usepackage{mathrsfs}
\usepackage{braket}
\usepackage{amsthm}
\usepackage{hyperref}
\usepackage{cleveref}
\usepackage{geometry}
\usepackage{setspace}
\usepackage{graphicx}
\usepackage{float}
\usepackage{booktabs}
\usepackage{dnd_shared}

%==============================================================================
% HYPERREF CONFIGURATION
%==============================================================================

\hypersetup{
    colorlinks=true,
    linkcolor=blue,
    citecolor=blue,
    urlcolor=blue,
    bookmarksnumbered=true,
    pdftitle={Emergenza quantistica dalla potenzialità primordiale: il framework D-ND},
    pdfauthor={D-ND Research Collective},
    pdfsubject={Emergenza quantistica, stato primordiale, operatore di emergenza},
    pdfkeywords={emergenza quantistica, stato primordiale, non-dualità, misura di emergenza, freccia informazionale, decoerenza, transizione quantistico-classica, decomposizione hamiltoniana, dinamica di Lindblad}
}

%==============================================================================
% GEOMETRY
%==============================================================================

\geometry{
    letterpaper,
    top=1in,
    bottom=1in,
    left=1in,
    right=1in
}

%==============================================================================
% CUSTOM COMMANDS (Paper A specific)
%==============================================================================

\newcommand{\Veff}{V_{\text{eff}}}
\newcommand{\Hplus}{\hat{H}_+}
\newcommand{\Hminus}{\hat{H}_-}
\newcommand{\Hint}{\hat{H}_{\text{int}}}
\newcommand{\Vzero}{\hat{V}_0}
\newcommand{\HD}{\hat{H}_D}
\newcommand{\Kop}{\hat{K}}
\newcommand{\Mbar}{\overline{M}}
\newcommand{\sigV}{\sigma^2_V}
\newcommand{\sigE}{\sigma^2_{\mathcal{E}}}
\newcommand{\thetaNT}{\theta_{\text{NT}}}
\newcommand{\Rcoll}{R_{\text{Collective}}}

%==============================================================================
% ADDITIONAL THEOREM-LIKE ENVIRONMENTS
%==============================================================================

\newtheorem{openproblem}[theorem]{Problema aperto}

%==============================================================================
% BEGIN DOCUMENT
%==============================================================================

\begin{document}

\title{Emergenza quantistica dalla potenzialità primordiale:\\Il framework Duale-Non-Duale per la differenziazione degli stati}

\author{D-ND Research Collective (Track A)}
\affiliation{Ricerca indipendente}
\date{14 febbraio 2026}

\begin{abstract}
Presentiamo un framework a sistema chiuso per l'emergenza quantistica in cui uno stato primordiale di indifferenziazione---lo stato Null-All $\NT$---subisce una differenziazione costruttiva attraverso un operatore di emergenza $\emerge$, producendo la realtà osservabile come $\resultant = U(t)\emerge\NT$. A differenza della decoerenza ambientale, che descrive la perdita di coerenza attraverso l'interazione con gradi di libertà esterni, il nostro modello spiega la \emph{costruzione} della struttura classica all'interno di un sistema ontologico chiuso. Definiamo una misura di emergenza $\emeasure = 1 - |\langle\text{NT}|U(t)\emerge|\text{NT}\rangle|^2$ e stabiliamo la sua convergenza asintotica sotto condizioni specificate. Dimostriamo che per sistemi con spettro assolutamente continuo e densità spettrale integrabile, $M(t) \to 1$ (emergenza totale), e che per spettri discreti la media di Ces\`aro $\Mbar$ converge a un valore ben definito. Questi risultati definiscono una \emph{freccia di emergenza} informazionale---distinta dalle frecce termodinamica e gravitazionale del tempo---che sorge puramente dalla struttura differenziale del sistema quantistico. Deriviamo la decomposizione hamiltoniana esplicita in settori duale ($\Hplus$), anti-duale ($\Hminus$) e di interazione, e presentiamo un'equazione master di Lindblad per la decoerenza indotta dall'emergenza con tasso $\Gamma = \sigV/\hbar^2 \cdot \langle(\Delta\Vzero)^2\rangle$. Introduciamo sei assiomi (A$_1$--A$_5$ per la meccanica quantistica, A$_6$ per l'estensione cosmologica), fondando la dinamica dell'emergenza sia alla scala quantistica che a quella cosmologica. Deriviamo il limite classico che connette $M(t)$ al parametro d'ordine $Z(t)$ di una teoria lagrangiana efficace, stabiliamo la condizione di coerenza ciclica $\OmegaNT = 2\pi i$ che governa le orbite periodiche di emergenza, e proponiamo protocolli sperimentali concreti per sistemi di QED a circuito e ioni intrappolati con predizioni quantitative che distinguono l'emergenza D-ND dalla decoerenza standard.
\end{abstract}

\keywords{emergenza quantistica, stato primordiale, non-dualità, misura di emergenza, freccia informazionale, decoerenza, transizione quantistico-classica, decomposizione hamiltoniana, dinamica di Lindblad, validazione computazionale}

\maketitle


%==============================================================================
% SEZIONE 1: INTRODUZIONE
%==============================================================================

\section{Introduzione}
\label{sec:intro}

\subsection{Il problema: emergenza e differenziazione}
\label{sec:problem}

Un puzzle fondamentale ai fondamenti della fisica riguarda l'origine della differenziazione: come emerge la realtà classica osservabile, con i suoi stati e proprietà distinte, da un substrato quantistico indifferenziato? La narrativa standard si appella a tre meccanismi:

\begin{enumerate}
    \item \textbf{Freccia termodinamica}: il Secondo Principio stabilisce una direzione temporale tramite la meccanica statistica, ma presuppone una condizione iniziale asimmetrica (bassa entropia) la cui origine rimane inspiegata~\cite{Penrose2004}.
    \item \textbf{Freccia gravitazionale}: l'ipotesi dell'entropia gravitazionale di Penrose connette l'asimmetria temporale alla formazione dei buchi neri, ma è dipendente dalla scala e confinata ai regimi gravitazionali~\cite{Penrose2010}.
    \item \textbf{Decoerenza quantistica}: seguendo Zurek~\cite{Zurek2003,Zurek2009}, Joos e Zeh~\cite{JoosZeh1985} e Schlosshauer~\cite{Schlosshauer2004,Schlosshauer2019}, l'interazione con l'ambiente causa il collasso della sovrapposizione in stati puntatore. Tuttavia la decoerenza è intrinsecamente \emph{distruttiva}---descrive la perdita di informazione verso l'ambiente, non la creazione di informazione all'interno di un sistema chiuso.
\end{enumerate}

Tutti e tre i meccanismi affrontano l'\emph{apparenza} della classicità o la \emph{perdita} di coerenza. Nessuno affronta direttamente l'\emph{emergenza} di struttura e differenziazione da uno stato iniziale indifferente all'interno di un sistema chiuso.


\subsection{Lacuna nella letteratura}
\label{sec:gap}

La lacuna centrale è questa: \textbf{la decoerenza spiega il ``come'' della perdita di coerenza, ma non il ``perché'' della differenziazione emergente.} Più fondamentalmente, la decoerenza richiede un ambiente esterno---è un processo di \emph{sistema aperto}. Eppure l'universo nel suo complesso non ha ambiente esterno. Il programma ``it-from-bit'' di Wheeler~\cite{Wheeler1989} e la proposta senza confini di Hartle-Hawking~\cite{HartleHawking1983} suggeriscono entrambi che qualsiasi teoria fondamentale dell'emergenza debba applicarsi a sistemi chiusi.


\subsection{Proposta: emergenza costruttiva tramite $\emerge$}
\label{sec:proposal}

Proponiamo il \textbf{framework Duale-Non-Duale (D-ND)} come alternativa a sistema chiuso:

\begin{itemize}
    \item \textbf{Stato primordiale}: $\NT$ (stato Null-All) rappresenta la potenzialità pura, indifferenziata---una sovrapposizione uniforme di tutti gli autostati.
    \item \textbf{Operatore di emergenza}: $\emerge$ agisce su $\NT$ costruttivamente, selezionando e pesando direzioni specifiche nello spazio di Hilbert. A differenza dell'interazione ambientale, $\emerge$ è una caratteristica \emph{intrinseca} della struttura ontologica del sistema.
    \item \textbf{Misura di emergenza}: $\emeasure = 1 - |\langle\text{NT}|U(t)\emerge|\text{NT}\rangle|^2$ quantifica il grado di differenziazione dalla potenzialità iniziale.
    \item \textbf{Freccia di emergenza}: il comportamento asintotico di $M(t)$ stabilisce una terza freccia fondamentale---ortogonale alle frecce termodinamica e gravitazionale---che sorge dalla struttura differenziale del sistema quantistico.
\end{itemize}


\subsection{Contributi di questo lavoro}
\label{sec:contributions}

\begin{enumerate}
    \item Framework formale con sei assiomi (A$_1$--A$_5$ per la MQ, A$_6$ per l'estensione cosmologica).
    \item Teoremi asintotici rigorosi con condizioni di regolarità esplicite e controesempi.
    \item Decomposizione hamiltoniana esplicita in settori duale ($\Hplus$), anti-duale ($\Hminus$) e di interazione.
    \item Caratterizzazione teorico-informazionale di $\emerge$ tramite il principio di massima entropia~\cite{Jaynes1957}.
    \item Equazione master di Lindblad con tasso quantitativo di decoerenza $\Gamma = \sigV/\hbar^2 \cdot \langle(\Delta\Vzero)^2\rangle$.
    \item Ponte quantistico-classico che deriva il parametro d'ordine lagrangiano efficace $Z(t)$ da $M(t)$.
    \item Validazione computazionale tramite simulazione numerica per $N = 2, 4, 8, 16$.
    \item Protocolli sperimentali concreti per sistemi di QED a circuito e ioni intrappolati.
    \item Confronto completo con decoerenza, gravità quantistica e framework di geometria dell'informazione.
\end{enumerate}


%==============================================================================
% SEZIONE 2: IL FRAMEWORK D-ND
%==============================================================================

\section{Il framework Duale-Non-Duale}
\label{sec:framework}

\subsection{Assiomi A$_1$--A$_6$}
\label{sec:axioms}

\begin{axiom}[Dualità intrinseca]
\label{ax:A1}
Ogni fenomeno fisico ammette una decomposizione in componenti opposte complementari, $\Phi_+$ e $\Phi_-$, tali che l'unione $\Phi_+ \cup \Phi_-$ sia esaustiva e mutuamente esclusiva in qualsiasi misurazione.
\end{axiom}

\begin{axiom}[Non-dualità come sovrapposizione indeterminata]
\label{ax:A2}
Al di sotto di tutte le decomposizioni duali esiste uno stato primordiale indifferenziato, lo stato Null-All $\NT$, in cui nessuna dualità si è attualizzata:
\begin{equation}
\label{eq:NT}
\NT = \frac{1}{\sqrt{N}} \sum_{n=1}^{N} |n\rangle
\end{equation}
dove $\{|n\rangle\}$ copre la base completa di $\mathcal{H}$, con $N \to \infty$ per spazi a dimensione infinita.
\end{axiom}

\begin{axiom}[Struttura evolutiva input-output]
\label{ax:A3}
Ogni sistema evolve continuamente tramite cicli input-output accoppiati attraverso un operatore di evoluzione unitario $U(t) = e^{-iHt/\hbar}$:
\begin{equation}
\label{eq:Rt}
R(t) = U(t)\emerge\NT
\end{equation}
dove $R(t)$ è lo stato risultante e $\emerge$ è l'operatore di emergenza che agisce al confine tra non-dualità e manifestazione.
\end{axiom}

\begin{axiom}[Dinamica relazionale in substrato atemporale (rivisto)]
\label{ax:A4}
Il sistema totale soddisfa il vincolo di Wheeler-DeWitt~\cite{Wheeler1968}:
\begin{equation}
\hat{H}_{\text{tot}}|\Psi\rangle = 0
\end{equation}
sullo spazio di Hilbert esteso $\mathcal{H} = \mathcal{H}_{\text{clock}} \otimes \mathcal{H}_{\text{system}}$. La dinamica osservabile emerge relazionalmente tramite il meccanismo di Page-Wootters~\cite{PageWootters1983,GiovannettiLloydMaccone2015}:
\begin{equation}
|\psi(\tau)\rangle = {}_{\text{clock}}\langle\tau|\Psi\rangle
\end{equation}
Il parametro $t$ nell'Assioma~A$_3$ è identificato con $\tau$; non è tempo assoluto ma un'osservabile relazionale emergente.
\end{axiom}

\begin{axiom}[Consistenza autologica tramite struttura a punto fisso (rivisto)]
\label{ax:A5}
La struttura inferenziale del sistema ammette una mappa autoreferenziale $\Phi: \mathcal{S} \to \mathcal{S}$ sullo spazio degli stati delle descrizioni. Per il teorema del punto fisso di Lawvere~\cite{Lawvere1969}, $\Phi$ ammette almeno un punto fisso $s^* = \Phi(s^*)$, che rappresenta una descrizione autoconsistente in cui lo stato del sistema e la sua descrizione coincidono. Questo punto fisso è inerente alla struttura categoriale di $\mathcal{S}$ (non raggiunto per iterazione), dunque la chiusura autologica è matematicamente garantita.
\end{axiom}

\textbf{Forma operativa ($R+1=R$):} La condizione di punto fisso autologico ha un'espressione operativa: $R(t+1) = R(t)$ in $s^*$. Questa non è un'identità banale ma un \emph{criterio di convergenza}: il proto-assioma che genera ogni iterazione non cambia attraverso l'iterazione. Formalmente, ciò corrisponde alla condizione di contrazione di Banach: $\|R(t+1) - R(t)\| \leq \kappa \|R(t) - R(t-1)\|$ con $\kappa < 1$.

\begin{axiom}[Manifestazione olografica (estensione cosmologica)]
\label{ax:A6}
La geometria spaziotemporale $g_{\mu\nu}$ deve codificare la dinamica di collasso del campo di emergenza:
\begin{equation}
\label{eq:einstein-info}
R_{\mu\nu} - \frac{1}{2}Rg_{\mu\nu} + \Lambda g_{\mu\nu} = 8\pi G \cdot T_{\mu\nu}^{\text{info}}[\emerge, \Kgen]
\end{equation}
dove $T_{\mu\nu}^{\text{info}}$ è il tensore energia-impulso informazionale.
\end{axiom}

\textbf{Nota:} L'Assioma~A$_6$ non è necessario per i risultati sull'emergenza quantistica (\S\S2--5); estende il framework alle scale cosmologiche (Paper~E).


\subsection{Lo stato Null-All $\NT$}
\label{sec:NT}

Proprietà di $\NT$:
\begin{enumerate}
    \item \textbf{Completezza}: $\NT$ copre $\mathcal{H}$ uniformemente.
    \item \textbf{Normalizzazione}: $\langle\text{NT}|\text{NT}\rangle = 1$.
    \item \textbf{Valore atteso osservabile}: $\langle\text{NT}|\hat{O}|\text{NT}\rangle = \text{Tr}[\hat{O}]/N$.
    \item \textbf{Entropia massimale del sottosistema}: $\rho_{\text{NT}} = \NT\langle\text{NT}|$ è puro ($S_{\text{vN}} = 0$), ma la matrice densità ridotta di qualsiasi sottosistema è massimalmente mista.
    \item \textbf{Indipendenza dalla base}: il valore atteso $\text{Tr}[\hat{O}]/N$ è indipendente dalla scelta della base.
\end{enumerate}

\begin{remark}[Stato matematico]
$\NT$ è uno stato quantistico standard (sovrapposizione uniforme) senza privilegio ontologico intrinseco. La scelta è motivata da: (1)~simmetria massimale, (2)~analogia con lo stato senza confini di Hartle-Hawking, (3)~il principio informazionale che lo stato iniziale meno vincolato debba essere il punto di partenza per l'emergenza. La novità non risiede in $\NT$ ma nell'operatore di emergenza $\emerge$ e nella misura $M(t)$.
\end{remark}

\textbf{Struttura fisica: insiemi potenziale e potenziato.} Il continuum NT ammette una partizione in due insiemi complementari:
\begin{itemize}
    \item \textbf{Insieme $\mathcal{P}$ (Potenziale):} Regime sub-planckiano ($E < E_{\text{Planck}}$), corrispondente ai modi $\lambdak \approx 0$. $\mathcal{P}$ \emph{cresce} man mano che il sistema si differenzia, perché ogni attualizzazione restituisce le possibilità non selezionate al serbatoio potenziale.
    \item \textbf{Insieme $\mathcal{A}$ (Attualizzato/Potenziato):} Regime sopra-Planck, modi $\lambdak > 0$. $\mathcal{A}$ \emph{decresce} con l'aumento dell'entropia.
\end{itemize}

La relazione fondamentale è:
\begin{equation}
\label{eq:PA-conservation}
|\mathcal{P}| + |\mathcal{A}| = \text{const} = \dim(\mathcal{H}), \qquad \frac{d|\mathcal{P}|}{dt} = -\frac{d|\mathcal{A}|}{dt} > 0
\end{equation}

La partizione $\mathcal{P}/\mathcal{A}$ e $M(t)$ sono descrizioni complementari dell'emergenza operanti a livelli diversi. La partizione $\mathcal{P}/\mathcal{A}$ traccia la ridistribuzione dello spazio delle possibilità: ogni attualizzazione restituisce le possibilità non selezionate al serbatoio potenziale ($|\mathcal{P}|$ cresce). La misura di emergenza $M(t) = 1 - |\langle\text{NT}|U(t)\emerge|\text{NT}\rangle|^2$ traccia l'allontanamento dello stato risultante dalla sovrapposizione iniziale indifferenziata ($M(t)$ cresce verso~1). Le due misure si muovono in direzioni opposte perché catturano aspetti complementari: $M(t) \to 1$ significa che il sistema si è massimalmente differenziato da $\NT$, mentre $|\mathcal{P}| \to \dim(\mathcal{H})$ significa che le possibilità non realizzate sono tornate al serbatoio potenziale. Entrambe le affermazioni descrivono l'emergenza totale.


\subsection{L'operatore di emergenza $\emerge$}
\label{sec:E-operator}

$\emerge$ è un operatore autoaggiunto con decomposizione spettrale:
\begin{equation}
\label{eq:E-spectral}
\emerge = \sum_{k=1}^{M} \lambdak |e_k\rangle\langle e_k|
\end{equation}
dove $\lambdak \in [0,1]$ sono gli autovalori di emergenza e $\{|e_k\rangle\}$ è una base ortonormale.

\textbf{Caratterizzazione teorico-informazionale}: l'operatore di emergenza fisico massimizza l'entropia di von Neumann dello stato emergente:
\begin{equation}
\label{eq:E-maxent}
\emerge = \arg\max_{\emerge'} S_{\text{vN}}(\rho_{\emerge'}) \quad \text{soggetto a} \quad \text{Tr}[\emerge'^2] = \sigE
\end{equation}

\begin{remark}[Ostacoli a una derivazione dai principi primi]
Derivare $\emerge$ dai principi primi richiede di risolvere il problema spettrale inverso: dati lo spettro emergente $\{\lambdak\}$, ricostruire l'operatore. Ciò equivale in geometria non commutativa~\cite{ChamseddineConnes1997} a ricostruire l'operatore di Dirac dal suo spettro---un problema notoriamente posto da Kac~\cite{Kac1966} e noto per essere genericamente mal posto.
\end{remark}


\subsection{Equazione fondamentale: $R(t) = U(t)\emerge\NT$}
\label{sec:fundamental}

Lo stato risultante al tempo relazionale $t$ è:
\begin{equation}
\label{eq:Rt-expanded}
R(t) = \sum_{k,n} \lambdak \langle e_k|\text{NT}\rangle \langle n|e_k\rangle \, e^{-iE_n t/\hbar} |n\rangle
\end{equation}


\subsection{Struttura hamiltoniana del sistema D-ND}
\label{sec:hamiltonian}

L'hamiltoniana totale ammette una decomposizione naturale che riflette la struttura duale dell'Assioma~A$_1$:
\begin{equation}
\label{eq:HD}
\HD = \Hplus \oplus \Hminus + \Hint + \Vzero + \Kop
\end{equation}
dove $\Hplus$ governa l'evoluzione nel settore $\Phi_+$ (duale), $\Hminus$ governa il settore $\Phi_-$ (anti-duale), $\Hint = \sum_k g_k (\hat{a}_+^k \hat{a}_-^{k\dagger} + \text{h.c.})$ accoppia i settori, $\Vzero$ è il potenziale di fondo non relazionale e $\Kop$ è l'operatore di curvatura informazionale.

L'equazione di Schr\"odinger unificata:
\begin{equation}
i\hbar \frac{\partial}{\partial t}|\Psi\rangle = \left[\Hplus \oplus \Hminus + \Hint + \Vzero + \Kop\right]|\Psi\rangle
\end{equation}


%==============================================================================
% SEZIONE 3: MISURA DI EMERGENZA E TEOREMI ASINTOTICI
%==============================================================================

\section{La misura di emergenza e i teoremi asintotici}
\label{sec:emergence}

\subsection{Definizione: $M(t)$}
\label{sec:Mt-def}

\begin{equation}
\label{eq:Mt}
M(t) = 1 - |f(t)|^2, \qquad f(t) = \langle\text{NT}|U(t)\emerge|\text{NT}\rangle
\end{equation}

Espandendo nella base degli autostati dell'energia con $a_n \equiv \langle n|\emerge|\text{NT}\rangle \cdot \langle\text{NT}|n\rangle$:
\begin{equation}
\label{eq:ft-expansion}
f(t) = \sum_n a_n \, e^{-iE_n t/\hbar}
\end{equation}
\begin{equation}
\label{eq:Mt-expansion}
M(t) = 1 - \sum_n |a_n|^2 - \sum_{n \neq m} a_n a_m^* \, e^{-i\omega_{nm} t}
\end{equation}
dove $\omega_{nm} = (E_n - E_m)/\hbar$ sono le frequenze di Bohr.

\begin{remark}[Relazione con la purezza]
Per $\emerge = I$, $M(t)$ si riduce al complemento della probabilità di sopravvivenza. Per $\emerge$ generico, $M(t)$ è legato alla purezza dello stato ridotto dopo la proiezione della componente $\NT$. Il framework D-ND reinterpreta questa misura standard in un contesto ontologico a sistema chiuso.
\end{remark}


\subsection{Proposizione 1: quasi-periodicità e convergenza di Ces\`aro}
\label{sec:prop1}

\begin{proposition}[Convergenza asintotica dell'emergenza]
\label{prop:cesaro}
Sia $H$ con spettro discreto non degenere $\{E_n\}_{n=1}^{N}$, e sia $\emerge|\text{NT}\rangle \neq |\text{NT}\rangle$. Allora:

\emph{(i) Quasi-periodicità}: per $N$ finito, $M(t)$ è quasi-periodico con ampiezza di oscillazione limitata da $2\sum_{n \neq m}|a_n||a_m|$.

\emph{(ii) Media di Ces\`aro}:
\begin{equation}
\label{eq:Mbar}
\Mbar \equiv \lim_{T \to \infty} \frac{1}{T} \int_0^T M(t) \, dt = 1 - \sum_{n=1}^{N} |a_n|^2
\end{equation}

\emph{(iii) Positività}: $\Mbar > 0$ ogniqualvolta $\emerge|\text{NT}\rangle \neq |\text{NT}\rangle$.
\end{proposition}

\begin{proof}[Dimostrazione di (ii)]
Dall'espansione di $|f(t)|^2$, i termini diagonali contribuiscono $\sum_n |a_n|^2$. Per i termini fuori diagonale con $\omega_{nm} \neq 0$: $\lim_{T\to\infty} \frac{1}{T}\int_0^T e^{-i\omega_{nm}t}\,dt = 0$. Pertanto $\overline{|f|^2} = \sum_n |a_n|^2$ e $\Mbar = 1 - \sum_n |a_n|^2$.
\end{proof}

\textbf{Controesempio (non-monotonicità):} Per $N = 2$ con $\lambdak = \{1, 1/2\}$: $dM/dt = (\omega/4\hbar)\sin(\omega t/\hbar)$, dimostrando che la monotonicità puntuale \emph{non} vale per spettri discreti finiti.


\subsection{Teorema 1: emergenza totale per spettro continuo}
\label{sec:thm1}

\begin{theorem}[Emergenza totale tramite Riemann-Lebesgue]
\label{thm:continuous}
Sia $H$ con spettro assolutamente continuo con misura spettrale $\mu$. Se la funzione di densità spettrale $g(E) := \langle\text{NT}|\delta(H-E)\emerge|\text{NT}\rangle$ soddisfa $g \in L^1(\mathbb{R})$, allora:
\begin{equation}
\lim_{t \to \infty} M(t) = 1
\end{equation}
\end{theorem}

\begin{proof}
Per lo spettro continuo, $f(t) = \int g(E) e^{-iEt/\hbar}\,dE$. Per il lemma di Riemann-Lebesgue, $f(t) \to 0$ per $t \to \infty$, dunque $M(t) \to 1$.
\end{proof}

\begin{remark}[Stato di novità]
Il Teorema~\ref{thm:continuous} è un'applicazione diretta del lemma di Riemann-Lebesgue---il contenuto matematico è standard. Il contributo è l'\emph{interpretazione in un'ontologia a sistema chiuso}: lo spettro continuo sorge dalla struttura interna di $\emerge$ e $H$, non dal tracciare via gradi di libertà ambientali.
\end{remark}


\subsection{Teorema 2: limite asintotico per il caso commutativo}
\label{sec:thm2}

\begin{theorem}[Emergenza asintotica---regime commutativo]
\label{thm:commuting}
Se $[H, \emerge] = 0$, allora:
\begin{equation}
\Mbar_\infty = 1 - \sum_k |\lambdak|^2 |\langle e_k|\text{NT}\rangle|^4
\end{equation}
\end{theorem}

\begin{proof}
Quando $[H, \emerge] = 0$, la base congiunta di autostati $|k\rangle$ dà $a_k = \lambdak|\beta_k|^2$ dove $\beta_k = \langle k|\text{NT}\rangle$. Allora $|a_k|^2 = |\lambdak|^2|\beta_k|^4$, e la sostituzione nella Proposizione~\ref{prop:cesaro}(ii) dà il risultato.
\end{proof}


\subsection{Freccia di emergenza (non freccia del tempo)}
\label{sec:arrow}

Sottolineiamo: \textbf{$M(t)$ definisce una freccia di \emph{emergenza}, non una freccia del \emph{tempo}.} La freccia del tempo si riferisce all'asimmetria temporale (irreversibilità). La freccia di emergenza si riferisce all'asimmetria informazionale---gli stati differenziati si accumulano in media.

L'irreversibilità effettiva emerge attraverso tre meccanismi:
\begin{itemize}
    \item[(A)] \textbf{Spettro continuo} (Teorema~\ref{thm:continuous}): $M(t) \to 1$ rigorosamente.
    \item[(B)] \textbf{Dinamica di Lindblad}: i termini fuori diagonale decadono come $a_n a_m^* e^{-i\omega_{nm}t - \gamma_{nm}t}$, producendo convergenza esponenziale.
    \item[(C)] \textbf{Grande $N$}: lo spettro denso produce defasamento effettivo tramite interferenza distruttiva.
\end{itemize}


\subsection{Equazione master di Lindblad per la dinamica di emergenza}
\label{sec:lindblad}

Quando $\Vzero$ fluttua con varianza $\sigV$, la matrice densità ridotta soddisfa:
\begin{equation}
\label{eq:lindblad}
\frac{d\bar{\rho}}{dt} = -\frac{i}{\hbar}[\HD, \bar{\rho}] - \frac{\sigV}{2\hbar^2}[\Vzero, [\Vzero, \bar{\rho}]]
\end{equation}

Il tasso di decoerenza:
\begin{equation}
\label{eq:Gamma}
\Gamma = \frac{\sigV}{\hbar^2}\langle(\Delta\Vzero)^2\rangle
\end{equation}

\begin{remark}[Distinzione critica]
Nella decoerenza standard, il doppio commutatore sorge dal tracciare via i gradi di libertà ambientali~\cite{CaldeiraLeggett1983}. Nel D-ND, sorge dalla media sulle fluttuazioni \emph{intrinseche} di $\Vzero$---il landscape di pre-differenziazione. La decoerenza non è causata da un bagno esterno ma dal rumore inerente nel potenziale non relazionale.
\end{remark}

La misura di emergenza nel regime di Lindblad:
\begin{equation}
M(t) \to 1 - \sum_n |a_n|^2 e^{-\Gamma_n t}
\end{equation}
dove $\Gamma_n = (\sigV/\hbar^2)|\langle n|\Vzero|m\rangle - \langle m|\Vzero|m\rangle|^2$ sono i tassi di decoerenza dipendenti dallo stato, fornendo convergenza \emph{esponenziale} verso l'emergenza.


\subsection{Tasso di produzione di entropia}
\label{sec:entropy}

\begin{equation}
\frac{dS}{dt} = -k_B \text{Tr}\left[\frac{d\bar{\rho}}{dt} \cdot \ln\bar{\rho}\right]
\end{equation}

Il termine unitario si annulla identicamente (per la ciclicità della traccia), producendo:
\begin{equation}
\label{eq:second-law}
\frac{dS}{dt} = \frac{k_B \sigV}{2\hbar^2} \text{Tr}\left[[\Vzero, [\Vzero, \bar{\rho}]] \ln\bar{\rho}\right] \geq 0
\end{equation}

La disuguaglianza segue dalla struttura di Lindblad~\cite{Spohn1978}: qualsiasi generatore CPTP produce produzione di entropia non negativa. Ciò stabilisce una \textbf{seconda legge dell'emergenza}: l'entropia informazionale dello stato emergente è monotonamente non decrescente sotto dinamica D-ND con fluttuazioni del potenziale.


%==============================================================================
% SEZIONE 4: ENTROPIA, DECOERENZA E SPAZIOTEMPO EMERGENTE
%==============================================================================

\section{Connessione con entropia, decoerenza e spaziotempo emergente}
\label{sec:connections}

\subsection{Entropia di Von Neumann e $M(t)$}

$M(t)$ (differenziazione strutturale) e $S(t)$ (diversità informazionale) sono complementari: uno stato può essere altamente differenziato da $\NT$ pur rimanendo puro ($S = 0$), oppure vicino a $\NT$ ma con entropia massimale.

\subsection{Confronto con la letteratura sulla decoerenza}

\textbf{Darwinismo quantistico di Zurek}~\cite{Zurek2003,Zurek2009}: il D-ND diverge in quattro aspetti: (1)~gli stati puntatore sono intrinseci a $\emerge$, non selezionati esternamente; (2)~il D-ND si applica a sistemi chiusi; (3)~l'informazione si riconfigura anziché dissiparsi; (4)~la scala temporale dell'emergenza dipende dalla struttura dell'operatore.

\textbf{Decoerenza di Joos-Zeh}~\cite{JoosZeh1985}: il D-ND è fondazionale---deriva l'emergenza degli stati preferiti da $\NT$, mentre Joos-Zeh ne presuppone l'esistenza.

\textbf{Analisi di Schlosshauer}~\cite{Schlosshauer2004,Schlosshauer2019}: $\emerge$ è precisamente il meccanismo che Schlosshauer identifica come mancante: specifica come i risultati si attualizzano senza osservatori esterni.

\textbf{Limiti biologici di Tegmark}~\cite{Tegmark2000}: l'emergenza D-ND è indipendente dalla decoerenza ambientale. Effetti non-markoviani~\cite{BreuerPetruccione2002} possono ulteriormente indebolire tali limiti.


\subsection{Distinzione chiave: emergenza costruttiva vs.\ distruttiva}

\begin{table}[h]
\caption{Confronto tra decoerenza ed emergenza D-ND.}
\label{tab:comparison}
\begin{tabular}{lll}
\toprule
Aspetto & Decoerenza & Emergenza D-ND \\
\midrule
Flusso informativo & Verso l'ambiente (perdita) & All'interno del sistema chiuso \\
Apertura del sistema & Aperto (accoppiamento al bagno) & Chiuso (intrinseco) \\
Scala temporale & Parametri ambientali & Struttura spettrale dell'operatore \\
Meccanismo & Defasamento per interazione & Attualizzazione spettrale tramite $\emerge$ \\
Base puntatore & Simmetria ambientale & Autospazio ontologico di $\emerge$ \\
\bottomrule
\end{tabular}
\end{table}


\subsection{Spaziotempo emergente}

Il framework D-ND si interfaccia con i programmi di spaziotempo emergente: la gravità entropica di Verlinde~\cite{Verlinde2011}, AdS/CFT e emergenza olografica~\cite{Maldacena1998,RyuTakayanagi2006,VanRaamsdonk2010}, QBismo~\cite{Fuchs2014} e il principio dell'azione spettrale~\cite{ChamseddineConnes1997}.


%==============================================================================
% SEZIONE 5: PONTE QUANTISTICO-CLASSICO
%==============================================================================

\section{Ponte quantistico-classico: da $M(t)$ a $Z(t)$}
\label{sec:bridge}

\subsection{Parametro d'ordine classico}

Si definisca $Z(t) \equiv M(t) = 1 - |f(t)|^2$. Questa identificazione è naturale: $Z = 0$ corrisponde allo stato non-duale, $Z = 1$ all'emergenza totale.

\subsection{Equazione del moto efficace}

Nel limite a grana grossa (proiezione di Mori-Zwanzig per $N \gg 1$):
\begin{equation}
\label{eq:langevin}
\ddot{\bar{Z}} + c_{\text{eff}} \dot{\bar{Z}} + \frac{\partial \Veff}{\partial \bar{Z}} = \xi(t)
\end{equation}

\subsection{Derivazione del potenziale a doppio pozzo}

Il potenziale efficace che soddisfa le condizioni al contorno, l'instabilità al punto medio e la regolarità:
\begin{equation}
\label{eq:Veff}
\Veff(Z) = Z^2(1-Z)^2 + \lambdaDND \cdot \thetaNT \cdot Z(1-Z)
\end{equation}
dove $\lambdaDND = 1 - 2\bar{\lambda}$ parametrizza l'asimmetria e $\thetaNT = \text{Var}(\{\lambdak\})/\bar{\lambda}^2$. La forma quartica appartiene alla classe di universalità di Ginzburg-Landau~\cite{LandauLifshitz1980}.


\subsection{Condizione di coerenza ciclica: $\OmegaNT = 2\pi i$}
\label{sec:cyclic}

Per orbite chiuse nel piano complesso di $Z$, l'integrale d'azione attorno a un ciclo completo soddisfa:
\begin{equation}
\label{eq:OmegaNT}
\OmegaNT \equiv \oint_{C} \frac{dZ}{\sqrt{2(E - \Veff(Z))}} = 2\pi i
\end{equation}

\textbf{Derivazione:} Per $E = 0$ e $\Veff(Z) = Z^2(1-Z)^2$:
\begin{equation}
\oint_C \frac{dZ}{Z(1-Z)} = \oint_C \left(\frac{1}{Z} + \frac{1}{1-Z}\right) dZ = 2\pi i
\end{equation}

\begin{remark}[Struttura del contorno dipolare]
\label{rem:WKB}
L'integrando $1/\sqrt{2(E - \Veff)}$ ha \emph{punti di ramificazione} (non poli semplici) ai punti di inversione $Z = 0$ e $Z = 1$. Il contorno $C$ è un percorso di tipo WKB che passa tra i punti di inversione su \emph{fogli di Riemann diversi} della radice quadrata, in modo analogo al contorno di quantizzazione di Bohr-Sommerfeld. Su un singolo foglio, la decomposizione in frazioni parziali $1/Z + 1/(1-Z)$ darebbe residui che si cancellano: $\text{Res}_{Z=0} + \text{Res}_{Z=1} = 1 + (-1) = 0$. Tuttavia, il contorno WKB attraversa il taglio di ramo che connette i punti di inversione, arrivando a $Z = 1$ sul foglio opposto dove la radice quadrata cambia segno. Questa attraversamento di foglio inverte il segno dell'integrando vicino a $Z = 1$, producendo il risultato non nullo $\OmegaNT = 2\pi i$.

Questo è il meccanismo standard nella teoria WKB (Berry \& Mount 1972): gli integrali di tunneling attraverso regioni classicamente proibite acquisiscono contributi immaginari dalla struttura di ramificazione di $\sqrt{E - V}$. L'unità immaginaria riflette il carattere di tunneling dell'orbita che connette i due minimi del potenziale.

\textbf{Interpretazione strutturale D-ND}: l'attraversamento del foglio al taglio di ramo è l'espressione matematica del \emph{terzo incluso} (Paper~D, \S11; Assioma~A$_5$): il contorno non tratta i due poli simmetricamente (il che darebbe zero per cancellazione---il terzo escluso), ma passa attraverso il confine generativo tra di essi, dove avviene l'inversione di segno. $\OmegaNT = 2\pi i$ esiste precisamente perché il contorno accede alla struttura \emph{tra} i due poli.
\end{remark}


\subsection{Dominio di validità}

Il ponte è valido quando: (1)~$N \gg 1$; (2)~lo spettro è denso; (3)~$\tau_{\text{cg}} \gg \max\{1/\omega_{nm}\}$.


%==============================================================================
% SEZIONE 6: ESTENSIONE COSMOLOGICA
%==============================================================================

\section{Estensione cosmologica}
\label{sec:cosmo}

L'operatore di curvatura $C = \int d^4x\,\Kgen(x,t)|x\rangle\langle x|$ accoppia la curvatura spaziotemporale all'emergenza quantistica. L'equazione modificata $R(t) = U(t)\emerge C\NT$ produce una misura di emergenza dipendente dalla curvatura $M_C(t) = 1 - |\langle\text{NT}|U(t)\emerge C|\text{NT}\rangle|^2$.

\begin{remark}
L'estensione alla curvatura è schematica. La connessione ai programmi di gravità quantistica richiede una formalizzazione aggiuntiva sostanziale.
\end{remark}


%==============================================================================
% SEZIONE 7: PREDIZIONI SPERIMENTALI
%==============================================================================

\section{Predizioni sperimentali e falsificabilità}
\label{sec:experiments}

\subsection{Strategia sperimentale}

Le predizioni nuove sorgono in tre domini: (1)~dipendenza di $\Mbar$ dalla struttura dell'operatore; (2)~ponte quantistico-classico; (3)~emergenza a sistema chiuso senza accoppiamento ambientale.

\subsection{Protocollo 1: QED a circuito}
\label{sec:circuit-qed}

\textbf{Sistema}: $N = 4$ qubit transmon ($T_1 \sim 100\,\mu$s, $T_2 \sim 50\,\mu$s). Preparare $\NT$ tramite $H^{\otimes 4}|0000\rangle$. Implementare $\emerge$ tramite gate di fase controllata.

\textbf{Predizioni quantitative}: $\Mbar_{\text{lineare}} \approx 0.978$, $\Mbar_{\text{gradino}} \approx 0.969$ per $N = 16$. La differenza $\Delta\Mbar \approx 0.010$ è misurabile con l'attuale precisione tomografica ($\sigma_M \sim 0.01$).

\textbf{Predizione del tasso di decoerenza}: $\Gamma_{\text{D-ND}} \approx 0.22\,\omega_{\min}$, \emph{indipendente} dal fattore di qualità della cavità $Q$. La decoerenza standard predice $\Gamma \propto 1/Q$. Ciò fornisce un test discriminante diretto.


\subsection{Protocollo 2: ioni intrappolati}
\label{sec:trapped-ions}

\textbf{Sistema}: $N = 8$ ioni ${}^{171}\text{Yb}^+$ ($T_2 > 1$~s). Per $N = 256$ ($8$ qubit), $M(t)$ dovrebbe esibire crescita monòtona effettiva con $\Delta M \lesssim 1/N \approx 0.004$.


\subsection{Criteri di falsificabilità}

\begin{table}[h]
\caption{Test di falsificabilità per l'emergenza D-ND.}
\label{tab:falsifiability}
\begin{tabular}{lll}
\toprule
Test & Predizione D-ND & MQ standard \\
\midrule
$\Mbar$ dipende dallo spettro di $\emerge$ & $\Mbar = 1 - \sum\|a_n\|^2$ & Stessa formula \\
$\Mbar$ indip.\ dall'accopp.\ ambientale & $\partial\Mbar/\partial\gamma = 0$ & $\Mbar$ cresce con $\gamma$ \\
Scaling con $N$ & $\Delta M \sim 1/N$ & Dipendente dal modello \\
\bottomrule
\end{tabular}
\end{table}

\textbf{Valutazione onesta}: per $N \leq 16$, il D-ND e la MQ standard fanno predizioni dinamiche identiche. La discriminazione richiede sistemi a grande $N$ o il ponte quantistico-classico.


\subsection{Validazione computazionale}

La simulazione numerica per $N = 2, 4, 8, 16$ con spettro di emergenza lineare conferma: (i)~comportamento oscillatorio per $N$ piccolo; (ii)~$\Mbar$ converge alla predizione analitica entro $\pm 0.5\%$; (iii)~monotonicità effettiva per $N \geq 16$; (iv)~la dinamica di Lindblad (con $\sigma_V/\hbar = 0.1\omega_0$) mostra convergenza esponenziale compatibile con $\Gamma$ entro il $3\%$.


\subsection{Validità del ponte quantistico-classico}

\begin{table}[h]
\caption{Affidabilità del ponte vs.\ dimensione del sistema.}
\label{tab:bridge}
\begin{tabular}{lcll}
\toprule
$N$ & Errore del ponte & Oscillazione & Stato \\
\midrule
2 & $\gtrsim 100\%$ & $O(1)$ & Non valido---restare nel quantistico \\
4 & $15$--$25\%$ & $O(0.1)$ & Marginale \\
8 & $\sim 5\%$ & $O(0.01)$ & Valido \\
16 & $< 1\%$ & $< O(0.001)$ & Altamente valido \\
\bottomrule
\end{tabular}
\end{table}


%==============================================================================
% SEZIONE 8: DISCUSSIONE E CONCLUSIONI
%==============================================================================

\section{Discussione e conclusioni}
\label{sec:conclusions}

\subsection{Sintesi dei risultati}

\begin{enumerate}
    \item Fondazione assiomatica rivista: A$_4$ (Page-Wootters) e A$_5$ (punto fisso di Lawvere) fondati rigorosamente.
    \item Classificazione asintotica: quasi-periodicità (Proposizione~\ref{prop:cesaro}), emergenza totale per spettri continui (Teorema~\ref{thm:continuous}), limite commutativo (Teorema~\ref{thm:commuting}).
    \item Decomposizione hamiltoniana $\HD$ con accoppiamento di settore.
    \item Equazione master di Lindblad con $\Gamma$ quantitativo.
    \item Seconda legge dell'emergenza ($dS/dt \geq 0$).
    \item Caratterizzazione teorico-informazionale di $\emerge$.
    \item Ponte quantistico-classico con potenziale a doppio pozzo di Ginzburg-Landau.
    \item Validazione computazionale per $N = 2, 4, 8, 16$.
    \item Protocolli sperimentali con predizioni quantitative.
\end{enumerate}


\subsection{Limitazioni e questioni aperte}

\begin{enumerate}
    \item Derivazione dell'operatore: $\emerge$ rimane fenomenologico.
    \item Monotonicità a sistema finito: $M(t)$ oscilla per $N < \infty$.
    \item Discriminazione sperimentale: richiede grande $N$ o il ponte.
    \item Gravità quantistica: l'estensione alla curvatura è schematica.
    \item Rigore matematico: necessario un trattamento a dimensione infinita.
\end{enumerate}


\subsection{Osservazioni conclusive}

Il framework D-ND fornisce un'alternativa a sistema chiuso alla decoerenza ambientale. Postulando un operatore di emergenza intrinseco e uno stato primordiale indifferenziato, spieghiamo come la realtà classica sorga deterministicamente dalla potenzialità quantistica. La misura di emergenza $M(t)$ stabilisce una \emph{freccia di emergenza}---distinta dalle frecce termodinamica e gravitazionale---che definisce un'asimmetria informazionale universale, deterministica e intrinsecamente quantistica. Se il D-ND catturi l'effettivo meccanismo della transizione quantistico-classica può essere stabilito solo attraverso l'esperimento.


%==============================================================================
% RIFERIMENTI BIBLIOGRAFICI
%==============================================================================

\begin{thebibliography}{40}

\bibitem{BreuerPetruccione2002}
H.-P.~Breuer and F.~Petruccione,
\emph{The Theory of Open Quantum Systems}
(Oxford University Press, 2002).

\bibitem{CaldeiraLeggett1983}
A.~O.~Caldeira and A.~J.~Leggett,
``Path integral approach to quantum Brownian motion,''
\emph{Physica A}\ \textbf{121}, 587--616 (1983).

\bibitem{ChamseddineConnes1997}
A.~H.~Chamseddine and A.~Connes,
``The spectral action principle,''
\emph{Commun.\ Math.\ Phys.}\ \textbf{186}, 731--750 (1997).

\bibitem{Fuchs2014}
C.~A.~Fuchs, N.~D.~Mermin, and R.~Schack,
``An introduction to QBism,''
in \emph{Quantum Theory: Informational Foundations and Foils}, pp.~267--292 (Springer, 2014).

\bibitem{GiovannettiLloydMaccone2015}
V.~Giovannetti, S.~Lloyd, and L.~Maccone,
``Quantum time,''
\emph{Phys.\ Rev.\ D}\ \textbf{92}, 045033 (2015).

\bibitem{HartleHawking1983}
J.~B.~Hartle and S.~W.~Hawking,
``Wave function of the universe,''
\emph{Phys.\ Rev.\ D}\ \textbf{28}, 2960--2975 (1983).

\bibitem{Jaynes1957}
E.~T.~Jaynes,
``Information theory and statistical mechanics,''
\emph{Phys.\ Rev.}\ \textbf{106}, 620 (1957).

\bibitem{JoosZeh1985}
E.~Joos and H.~D.~Zeh,
``The emergence of classical properties through interaction with the environment,''
\emph{Z.\ Phys.\ B}\ \textbf{59}, 223--243 (1985).

\bibitem{Kac1966}
M.~Kac,
``Can one hear the shape of a drum?''
\emph{Amer.\ Math.\ Monthly}\ \textbf{73}, 1--23 (1966).

\bibitem{LandauLifshitz1980}
L.~D.~Landau and E.~M.~Lifshitz,
\emph{Statistical Physics, Part~1} (3rd ed.)
(Pergamon Press, 1980).

\bibitem{Lawvere1969}
F.~W.~Lawvere,
``Diagonal arguments and cartesian closed categories,''
in \emph{Category Theory, Homology Theory and their Applications II},
Lecture Notes in Mathematics, vol.~92, pp.~134--145 (Springer, 1969).

\bibitem{Lindblad1976}
G.~Lindblad,
``On the generators of quantum dynamical semigroups,''
\emph{Commun.\ Math.\ Phys.}\ \textbf{48}, 119--130 (1976).

\bibitem{Maldacena1998}
J.~M.~Maldacena,
``The large $N$ limit of superconformal field theories and supergravity,''
\emph{Adv.\ Theor.\ Math.\ Phys.}\ \textbf{2}, 231--252 (1998).

\bibitem{Moreva2014}
E.~Moreva \emph{et~al.},
``Time from quantum entanglement: An experimental illustration,''
\emph{Phys.\ Rev.\ A}\ \textbf{89}, 052122 (2014).

\bibitem{PageWootters1983}
D.~N.~Page and W.~K.~Wootters,
``Evolution without evolution: Dynamics described by stationary observables,''
\emph{Phys.\ Rev.\ D}\ \textbf{27}, 2885--2892 (1983).

\bibitem{Penrose2004}
R.~Penrose,
\emph{The Road to Reality}
(Jonathan Cape, London, 2004).

\bibitem{Penrose2010}
R.~Penrose,
\emph{Cycles of Time}
(Jonathan Cape, London, 2010).

\bibitem{RyuTakayanagi2006}
S.~Ryu and T.~Takayanagi,
``Holographic derivation of entanglement entropy from AdS/CFT,''
\emph{Phys.\ Rev.\ Lett.}\ \textbf{96}, 181602 (2006).

\bibitem{Schlosshauer2004}
M.~Schlosshauer,
``Decoherence, the measurement problem, and interpretations of quantum mechanics,''
\emph{Rev.\ Mod.\ Phys.}\ \textbf{76}, 1267--1305 (2004).

\bibitem{Schlosshauer2019}
M.~Schlosshauer,
``Quantum decoherence,''
\emph{Physics Reports}\ \textbf{831}, 1--57 (2019).

\bibitem{Spohn1978}
H.~Spohn,
``Entropy production for quantum dynamical semigroups,''
\emph{J.\ Math.\ Phys.}\ \textbf{19}, 1227--1230 (1978).

\bibitem{Tegmark2000}
M.~Tegmark,
``Importance of quantum decoherence in brain processes,''
\emph{Phys.\ Rev.\ E}\ \textbf{61}, 4194--4206 (2000).

\bibitem{VanRaamsdonk2010}
M.~Van~Raamsdonk,
``Building up spacetime with quantum entanglement,''
\emph{Gen.\ Rel.\ Grav.}\ \textbf{42}, 2323--2329 (2010).

\bibitem{Verlinde2011}
E.~Verlinde,
``On the origin of gravity and the laws of Newton,''
\emph{JHEP}\ \textbf{2011}(4), 29 (2011).

\bibitem{Wheeler1968}
J.~A.~Wheeler,
``Superspace and the nature of quantum geometrodynamics,''
in C.~DeWitt and J.~A.~Wheeler (Eds.), \emph{Battelle Rencontres}, pp.~242--307 (Benjamin, 1968).

\bibitem{Wheeler1989}
J.~A.~Wheeler,
``Information, physics, quantum: The search for links,''
in \emph{Proc.\ 3rd Int.\ Symp.\ Foundations of Quantum Mechanics} (1989).

\bibitem{Zurek2003}
W.~H.~Zurek,
``Decoherence and the transition from quantum to classical,''
\emph{Rev.\ Mod.\ Phys.}\ \textbf{75}, 715 (2003).

\bibitem{Zurek2009}
W.~H.~Zurek,
``Quantum Darwinism,''
\emph{Nature Phys.}\ \textbf{5}, 181--188 (2009).

\end{thebibliography}

\end{document}
