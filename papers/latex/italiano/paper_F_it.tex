%==============================================================================
% PAPER F - MOTORE DI INFORMAZIONE QUANTISTICA D-ND
% Traduzione italiana — per lettura personale
% Originale: paper_F.tex (Quantum / Physical Review A / Quantum Science and Technology)
%==============================================================================

\documentclass[aps,pra,11pt,notitlepage,nofootinbib,longbibliography]{revtex4-2}

%==============================================================================
% PACKAGES
%==============================================================================

\usepackage[utf8]{inputenc}
\usepackage[T1]{fontenc}
\usepackage[italian]{babel}
\usepackage{amsmath}
\usepackage{amssymb}
\usepackage{mathrsfs}
\usepackage{braket}
\usepackage{amsthm}
\usepackage{hyperref}
\usepackage{cleveref}
% natbib loaded by revtex4-2 automatically
\usepackage{geometry}
\usepackage{setspace}
\usepackage{graphicx}
\usepackage{float}
\usepackage{booktabs}
\usepackage{listings}
\usepackage{dnd_shared}

%==============================================================================
% HYPERREF CONFIGURATION
%==============================================================================

\hypersetup{
    colorlinks=true,
    linkcolor=blue,
    citecolor=blue,
    urlcolor=blue,
    bookmarksnumbered=true,
    pdftitle={Motore di informazione quantistica D-ND: gate quantistici modificati e framework computazionale},
    pdfauthor={D-ND Research Collective},
    pdfsubject={Calcolo quantistico, gate D-ND, densità possibilistica},
    pdfkeywords={informazione quantistica possibilistica, gate D-ND, insiemi universali di gate, IFS, correzione di errore quantistica, calcolo assistito dall'emergenza}
}

%==============================================================================
% GEOMETRY
%==============================================================================

\geometry{
    letterpaper,
    top=1in,
    bottom=1in,
    left=1in,
    right=1in
}

%==============================================================================
% LISTINGS STYLE (for pseudocode)
%==============================================================================

\lstset{
    basicstyle=\small\ttfamily,
    frame=single,
    breaklines=true,
    numbers=left,
    numberstyle=\tiny,
    xleftmargin=2em,
    framexleftmargin=1.5em
}

%==============================================================================
% CUSTOM COMMANDS (Paper F specific)
%==============================================================================

\newcommand{\Mdist}{M_{\text{dist}}}
\newcommand{\Ment}{M_{\text{ent}}}
\newcommand{\Mproto}{M_{\text{proto}}}
\newcommand{\HDND}{H_{\text{DND}}}
\newcommand{\CNOTDND}{\text{CNOT}_{\text{DND}}}
\newcommand{\PDND}{P_{\text{DND}}}
\newcommand{\Rlinear}{R_{\text{linear}}}
\newcommand{\Remit}{R_{\text{emit}}}
\newcommand{\CNOTstd}{\text{CNOT}_{\text{std}}}

% Conjecture environment
\theoremstyle{plain}
\newtheorem{conjecture}[theorem]{Congettura}
\newtheorem{openproblem}[theorem]{Problema aperto}

%==============================================================================
% DOCUMENT
%==============================================================================

\begin{document}

\title{Motore di informazione quantistica D-ND:\\Gate quantistici modificati e framework computazionale}

\author{D-ND Research Collective}
\affiliation{Ricerca indipendente}
\date{14 febbraio 2026}

\begin{abstract}
Formalizziamo gli aspetti computazionali quantistici del framework D-ND (Duale-Non-Duale)
introducendo un'architettura di informazione quantistica possibilistica che generalizza la
meccanica quantistica standard. Invece della pura sovrapposizione probabilistica, gli stati
quantistici D-ND sono caratterizzati da una misura di \emph{densit\`a possibilistica}
$\rhoDND$ che incorpora struttura di emergenza, accoppiamento non locale e invarianti
topologici. Definiamo quattro gate quantistici modificati---$\HDND$, $\CNOTDND$, $\PDND$
e Shortcut$_{\text{DND}}$---che preservano la struttura D-ND consentendo al contempo il
calcolo pratico. Dimostriamo che $\{\HDND, \CNOTDND, \PDND\}$ formano un insieme universale
di gate nel regime perturbativo, derivando unitari SU$(2^n)$ arbitrari dalle composizioni
dei gate. Viene presentato un modello di circuito completo con analisi degli errori e
garanzie di preservazione della coerenza. Sviluppiamo un framework di simulazione basato su
sistemi di funzioni iterate (IFS) con pseudocodice e analisi della complessit\`a polinomiale.
Posizioniamo il calcolo D-ND nel contesto dei risultati noti sul vantaggio quantistico
(BQP vs.\ BPP), mostrando come la soppressione degli errori assistita dall'emergenza fornisca
un percorso distinto verso l'accelerazione quantistica. Vengono discusse applicazioni agli
algoritmi di ricerca quantistica e al calcolo quantistico topologico. Questo lavoro collega
la teoria dell'informazione quantistica alla dinamica dell'emergenza, stabilendo il framework
D-ND come paradigma computazionale praticabile per algoritmi ibridi quantistico-classici a
breve termine.
\end{abstract}

\keywords{Informazione quantistica possibilistica, gate D-ND, insiemi universali di gate,
sistemi di funzioni iterate, correzione di errore quantistica, calcolo assistito dall'emergenza,
complessit\`a BQP, calcolo quantistico topologico}

\maketitle

%==============================================================================
\section{Introduzione}
\label{sec:intro}
%==============================================================================

Il calcolo quantistico ha raggiunto notevoli progressi teorici e sperimentali, eppure
persistono limitazioni fondamentali: la decoerenza, il collasso della misura e
l'interpretazione strettamente probabilistica della regola di Born vincolano lo spazio degli
algoritmi e delle applicazioni. Il framework D-ND (sviluppato nei Paper A--E) propone che i
sistemi quantistici non debbano essere puramente probabilistici; piuttosto, la
\emph{possibilit\`a} pu\`o coesistere con la probabilit\`a, mediata attraverso l'emergenza
e l'accoppiamento non locale.

\subsection{Chiarimento sulla notazione}
\label{sec:notation}

In tutto questo lavoro, il coefficiente di accoppiamento dell'emergenza $\lambda$ (senza
pedice) rappresenta il parametro di approssimazione lineare che quantifica l'intensit\`a
delle modifiche ai gate quantistici D-ND rispetto alle operazioni quantistiche standard.
Questo si distingue da:
$\lambdak$ del Paper~A: autovalori dell'operatore di emergenza nel substrato quantistico;
$\lambdaDND$ del Paper~B: costante di accoppiamento del potenziale nell'hamiltoniana
duale-non-duale;
$\lambdaauto$ del Paper~D: tasso di convergenza autologica nella dinamica dell'osservatore;
$\lambdacosmo$ del Paper~E: accoppiamento di emergenza cosmologica.
La notazione viene ulteriormente chiarita nel \S\ref{sec:paperA-connection}, dove
$\lambda = M(t)$ (la misura di emergenza) durante il regime di approssimazione lineare.

\subsection{Motivazioni}

\begin{enumerate}
\item \textbf{Oltre il probabilismo}: La meccanica quantistica standard tratta tutta
l'informazione come ampiezze probabilistiche. Il framework D-ND ammette stati
possibilistici---sovrapposizioni in cui alcuni rami possono essere
``proto-attuali'' (non ancora pienamente attualizzati) o ``soppressi'' dalla dinamica
dell'emergenza.

\item \textbf{Emergenza non locale}: Invece di considerare la non localit\`a come
un'azione spettrale a distanza, il framework D-ND la modella come struttura nel campo di
emergenza $\emerge$. I gate quantistici possono essere progettati per sfruttare questa
struttura.

\item \textbf{Robustezza topologica}: Il framework D-ND incorpora invarianti topologici
(cicli omologici, numeri di Betti) che forniscono correzione naturale degli errori e
miglioramento della fedelt\`a dei gate.

\item \textbf{Ibrido classico-quantistico}: Il framework di simulazione lineare consente
un'emulazione classica efficiente di certi circuiti D-ND, riducendo i requisiti hardware.

\item \textbf{Vantaggio quantistico attraverso l'emergenza}: A differenza degli approcci
standard che si affidano unicamente alla sovrapposizione quantistica, il framework D-ND
offre la soppressione degli errori assistita dall'emergenza, un nuovo percorso verso il
vantaggio quantistico.
\end{enumerate}

\subsection{Struttura del lavoro}

La Sezione~\ref{sec:framework} introduce la misura di densit\`a possibilistica e la sua
relazione con gli stati quantistici standard. La Sezione~\ref{sec:gates} definisce i
quattro gate modificati fondamentali con regole di composizione rigorose. La
Sezione~\ref{sec:circuit} sviluppa il modello di circuito e l'analisi degli errori. La
Sezione~\ref{sec:simulation} presenta il framework di simulazione basato su IFS con
pseudocodice. La Sezione~\ref{sec:applications} delinea le applicazioni, confronta con i
risultati noti sul vantaggio quantistico e stabilisce un ponte computazionale verso la
libreria THRML/Omega-Kernel di Extropic AI. La Sezione~\ref{sec:conclusions} conclude.
Le Appendici~\ref{app:prop22} e \ref{app:prop43} forniscono le dimostrazioni delle
proposizioni chiave.

%==============================================================================
\section{Framework di informazione quantistica D-ND}
\label{sec:framework}
%==============================================================================

\subsection{Densit\`a possibilistica $\rhoDND$}
\label{sec:rho-dnd}

Nella meccanica quantistica standard, lo stato di un sistema \`e dato da una matrice
densit\`a $\rho \in \mathcal{L}(\mathcal{H})$, dove $\mathcal{L}(\mathcal{H})$ \`e lo
spazio degli operatori lineari limitati sullo spazio di Hilbert $\mathcal{H}$. Il framework
D-ND generalizza questo a una \emph{densit\`a possibilistica} incorporando l'emergenza.

\begin{definition}[Densit\`a possibilistica --- Formula B10]
\label{def:rho-dnd}
Siano $\Mdist$, $\Ment$, $\Mproto$ tre misure reali non negative sugli stati base dello
spazio di Hilbert:
\begin{itemize}
\item $\Mdist$: \emph{capacit\`a distributiva} (quanto lo stato \`e ``distribuito'' tra gli
elementi della base)
\item $\Ment$: \emph{intensit\`a dell'entanglement} (grado della struttura di correlazione
non locale)
\item $\Mproto$: \emph{misura di proto-attualizzazione} (quanto un ramo \`e ``pronto'' a
diventare classico)
\end{itemize}
Allora la \textbf{densit\`a possibilistica} \`e:
\begin{equation}
\rhoDND = \frac{\Mdist + \Ment + \Mproto}{\sum_{\text{all states}} (\Mdist + \Ment + \Mproto)}
= \frac{M}{\Sigma M}
\label{eq:rho-dnd}
\end{equation}
dove $M = \Mdist + \Ment + \Mproto$ e $\Sigma M$ \`e la misura totale su tutto il sistema.
\end{definition}

\textbf{Interpretazione:}
Ogni componente di $M$ rappresenta un aspetto diverso dell'``essere disponibile al calcolo'':
$\Mdist$ tiene conto dell'ampiezza della sovrapposizione (analoga all'entropia di Shannon
nello spazio delle possibilit\`a); $\Ment$ cattura la struttura non locale (i rami che
partecipano a correlazioni a lungo raggio hanno $\Ment$ pi\`u elevata); e $\Mproto$ misura
quanto un ramo \`e vicino all'attualit\`a classica.

\textbf{Osservazione sull'indipendenza delle misure e il contenuto operativo.}
La Definizione~\ref{def:rho-dnd} richiede tre misure le cui definizioni devono essere fondate
operativamente:
\begin{enumerate}
\item \textbf{$\Mdist$ (Capacit\`a distributiva)}: L'entropia di Shannon della distribuzione
di probabilit\`a sugli stati base,
\begin{equation}
\Mdist = -\sum_i p_i \log p_i
\end{equation}
dove $p_i = |\langle i|\psi\rangle|^2$ sono le probabilit\`a degli stati base.

\item \textbf{$\Ment$ (Intensit\`a dell'entanglement)}: Per sistemi bipartiti, la negativit\`a
(Vidal \& Werner, 2002~\cite{Vidal2002}):
\begin{equation}
\Ment = \max(0, \text{Neg}(\rho_{AB})) = \max_k(0, -\lambda_k)
\end{equation}
dove $\lambda_k$ sono gli autovalori della trasposta parziale.

\item \textbf{$\Mproto$ (Misura di proto-attualizzazione)}: Definita a partire dalla misura
di emergenza del Paper~A:
\begin{equation}
\Mproto(t) = 1 - M(t) = |\langle\text{NT}|U(t)\emerge|\text{NT}\rangle|^2
\end{equation}
\end{enumerate}

Con queste identificazioni, $\rhoDND$ \`e un'autentica estensione delle matrici densit\`a
standard, che contiene informazione---la traiettoria di proto-attualizzazione
$\Mproto(t)$---che gli stati quantistici standard scartano.

\subsection{Connessione con gli stati quantistici standard}
\label{sec:standard-connection}

\begin{proposition}[Immersione nello spazio di Hilbert]
\label{prop:hilbert-embedding}
Se $\Mproto \equiv 0$ (nessuna proto-attualizzazione, regime quantistico puro) e
$\mathcal{H}$ \`e separabile, allora $\rhoDND$ definisce un operatore densit\`a valido
tramite:
\begin{equation}
\hat{\rho}_{\text{DND}} = \sum_i \frac{M(i)}{\Sigma M} |i\rangle\langle i|
\end{equation}
dove $M(i) = \Mdist(i) + \Ment(i)$ e $\Sigma M = \sum_i M(i)$. Questo soddisfa:
(i)~$\text{Tr}[\hat{\rho}_{\text{DND}}] = 1$, (ii)~$\hat{\rho}_{\text{DND}} \geq 0$,
(iii)~$\hat{\rho}_{\text{DND}} = \hat{\rho}_{\text{DND}}^\dagger$.
Il prodotto interno
$\langle\psi|\phi\rangle_{\text{DND}} = \text{Tr}[|\psi\rangle\langle\phi|\hat{\rho}_{\text{DND}}]
= \sum_i a_i^* b_i \rho_{\text{DND}}(i)$
(dove $|\psi\rangle = \sum_i a_i|i\rangle$, $|\phi\rangle = \sum_i b_i|i\rangle$) definisce
una struttura di spazio di Hilbert pesata che si riduce al prodotto interno standard quando
$M(i)$ \`e uniforme.
\end{proposition}

\emph{Dimostrazione}: Si veda l'Appendice~\ref{app:prop22}.

\subsection{Connessione con la misura di emergenza del Paper A}
\label{sec:paperA-connection}

Il Paper~A stabilisce la misura di emergenza fondamentale
$M(t) = 1 - |\langle\text{NT}|U(t)\emerge|\text{NT}\rangle|^2$,
che quantifica il grado di differenziazione dello stato rispetto allo stato non localizzato
$\NT$.

\begin{proposition}[$M(t)$ e proto-attualizzazione]
\label{prop:M-proto}
La misura di proto-attualizzazione $\Mproto$ pu\`o essere identificata con il complemento
della misura di emergenza del Paper~A:
\begin{equation}
\Mproto(t) = 1 - M(t) = |\langle\text{NT}|U(t)\emerge|\text{NT}\rangle|^2
\end{equation}
\end{proposition}

\textbf{Interpretazione:}
Quando $M(t) = 0$ (emergenza precoce): $\Mproto = 1$, il che significa che tutti i modi
rimangono proto-attuali. Quando $M(t) = 1$ (emergenza tardiva): $\Mproto = 0$, il che
significa che tutti i modi sono pienamente attualizzati. Il regime di transizione
($0 < M(t) < 1$) \`e la finestra D-ND in cui domina il comportamento ibrido
quantistico-classico.

\begin{proposition}[Misure distributiva e di entanglement]
\label{prop:M-decomp}
Le tre componenti soddisfano il vincolo:
\begin{equation}
\Mdist(t) + \Ment(t) = M(t), \qquad \Mproto(t) = 1 - M(t)
\end{equation}
cosicch\'e $\Mdist(t) + \Ment(t) + \Mproto(t) = 1$. La misura di emergenza $M(t)$ governa
la partizione: man mano che l'emergenza progredisce, il peso si trasferisce da $\Mproto$ a
$\Mdist + \Ment$.
\end{proposition}

\begin{proposition}[Riduzione agli stati quantistici standard]
\label{prop:reduction}
Quando $M(t) \to 1$ (equivalentemente $\Mproto \to 0$), la densit\`a possibilistica
$\rhoDND$ si riduce a uno stato quantistico standard:
\begin{equation}
\lim_{M(t) \to 1} \rhoDND = \rho_{\text{standard}} =
\frac{\Mdist + \Ment}{\sum_{\text{states}}(\Mdist + \Ment)}
\end{equation}
che soddisfa le probabilit\`a della regola di Born sotto misurazione.
\end{proposition}

\textbf{Osservazione sulle implicazioni per i circuiti:} Nei circuiti D-ND pratici,
$\lambda = M(t)$. Pertanto l'approssimazione lineare
$\Rlinear(t) = P(t) + \lambda \cdot \Remit(t)$ \`e valida durante l'emergenza precoce
($M(t) < 0{,}5$), dove la proto-attualizzazione \`e dominante e la componente classica
$P(t)$ \`e piccola.

%==============================================================================
\section{Gate quantistici modificati}
\label{sec:gates}
%==============================================================================

Definiamo quattro gate fondamentali adattati al framework D-ND. Ogni gate:
(1)~preserva la struttura di $\rhoDND$;
(2)~incorpora retroazione dal campo di emergenza $\emerge$;
(3)~si riduce ai gate standard quando $\Mproto \to 0$.

\subsection{Hadamard$_{\text{DND}}$ (Formula C1)}
\label{sec:hadamard}

L'Hadamard standard $H$ crea una sovrapposizione uniforme:
$H|0\rangle = (|0\rangle + |1\rangle)/\sqrt{2}$.

\begin{definition}
\label{def:hadamard-dnd}
Il gate \textbf{Hadamard$_{\text{DND}}$} modifica la ridistribuzione della densit\`a
accoppiandola alla struttura di emergenza su grafo:
\begin{equation}
\HDND |v\rangle = \frac{1}{\mathcal{N}_v}
\sum_{u \in \text{Nbr}(v)} w_u \cdot \delta V_u \, |u\rangle
\label{eq:hadamard-dnd}
\end{equation}
dove $v$ \`e un vertice nel grafo di emergenza (etichetta di stato), $\delta V_u$ \`e il
gradiente del potenziale del campo di emergenza al nodo vicino $u$, $w_u$ \`e il peso di
emergenza (autovalore di $\emerge$ in $u$), $\text{Nbr}(v)$ \`e l'intorno di $v$, e
$\mathcal{N}_v = \sqrt{\sum_{u \in \text{Nbr}(v)} |w_u \cdot \delta V_u|^2}$ \`e il
fattore di normalizzazione che assicura l'unitariet\`a.
\end{definition}

\textbf{Interpretazione fisica:}
Invece di creare una sovrapposizione uniforme, $\HDND$ pesa ogni vicino in base alla sua
``prontezza'' di emergenza ($w_u$) e al gradiente del potenziale locale. Un $\delta V$
elevato concentra la sovrapposizione; un $\delta V$ basso consente una distribuzione pi\`u
ampia. La normalizzazione $\mathcal{N}_v$ assicura $\|\HDND|v\rangle\| = 1$.

\textbf{Osservazione sull'unitariet\`a:} Quando il campo di emergenza \`e statico e il
grafo \`e regolare, $\HDND$ si riduce all'Hadamard standard. Per grafi di emergenza generali,
$\HDND$ \`e unitario per costruzione ma non \`e generalmente autoaggiunto.
$\HDND^2 = I$ vale solo nel caso simmetrico.

\subsection{CNOT$_{\text{DND}}$ con emergenza non locale (Formula C2)}
\label{sec:cnot}

\begin{definition}
\label{def:cnot-dnd}
Il gate \textbf{CNOT$_{\text{DND}}$} incorpora l'accoppiamento di emergenza non locale:
\begin{equation}
\CNOTDND = \CNOTstd \cdot e^{-i\,s\,\ell^*}
\label{eq:cnot-dnd}
\end{equation}
dove $\CNOTstd = \left(\begin{smallmatrix} I & 0 \\ 0 & X \end{smallmatrix}\right)$ \`e il
gate CNOT standard, $s = \frac{1}{n}\sum_{i \neq j} |\langle i|H|j\rangle|$ \`e il parametro
di distribuzione non locale, e $\ell^* = 1 - \delta V$ \`e il fattore di coerenza
dell'emergenza con $\delta V = \|\nabla\emerge\|/\|\emerge\| \in [0,1]$.
\end{definition}

\textbf{Effetto:}
Il fattore di fase $e^{-is\ell^*}$ applica una fase non locale globale che dipende sia dal
tasso di distribuzione $s$ sia dal fattore di coerenza $\ell^*$. Quando $\delta V$ \`e
elevato (emergenza forte), $\ell^*$ \`e piccolo e il gate si avvicina al CNOT standard.
Quando $\delta V$ \`e basso, viene applicata la piena fase non locale.

\textbf{Composizione:} $\CNOTDND^2 = e^{-2is\ell^*} \cdot I$ (involutorio a meno di una
fase globale).

\subsection{Phase$_{\text{DND}}$ con accoppiamento alla fluttuazione del potenziale (Formula C3)}
\label{sec:phase}

\begin{definition}
\label{def:phase-dnd}
Il gate \textbf{Phase$_{\text{DND}}$} accoppia la dinamica di fase al potenziale di
emergenza:
\begin{equation}
\PDND(\phi)|v\rangle = e^{-i(1 - \phi_{\text{phase}} \cdot \delta V)} |v\rangle
\label{eq:phase-dnd}
\end{equation}
dove $\phi_{\text{phase}}$ \`e il parametro di fase classico e $\delta V$ \`e il gradiente
del potenziale di emergenza in $v$.
\end{definition}

\textbf{Interpretazione:}
La fase effettiva dipende dal potenziale di emergenza. In regime di emergenza forte
($\delta V \to 1$), la fase \`e soppressa. In regime di emergenza debole, viene applicata la
fase piena. Questo crea un paesaggio di fase dipendente dal potenziale, sfruttabile per il
calcolo topologico.

\subsection{Shortcut$_{\text{DND}}$ per operazioni topologiche (Formula C4)}
\label{sec:shortcut}

\begin{definition}[Principio di riduzione della profondit\`a del circuito]
\label{def:shortcut-dnd}
Data una struttura di entanglement bersaglio su $m$ qubit (che normalmente richiede $|E|$
operazioni CNOT), il fattore di compressione topologica $\chi \in (0, 1]$ derivato dal primo
numero di Betti del grafo di emergenza determina il conteggio ridotto dei gate:
\begin{equation}
m_{\text{reduced}} = \lceil \chi \cdot |E| \rceil, \qquad
\chi = \frac{\beta_1(G_\emerge)}{\beta_1(G_\emerge) + |E|}
\label{eq:shortcut}
\end{equation}
dove $\beta_1(G_\emerge)$ \`e il primo numero di Betti del grafo di emergenza.
\end{definition}

\textbf{Osservazione:} Shortcut$_{\text{DND}}$ non \`e un singolo gate unitario ma una
strategia di compilazione dei circuiti: specifica come riordinare i gate $\CNOTDND$ usando
informazione topologica per ridurre la profondit\`a del circuito. Il circuito risultante
implementa la stessa struttura di entanglement con meno gate.

\subsection{Universalit\`a dei gate (regime perturbativo)}
\label{sec:universality}

\begin{proposition}[Universalit\`a dei gate]
\label{prop:universality}
Nel regime di emergenza debole ($\delta V \ll 1$), l'insieme $\{\HDND, \CNOTDND, \PDND\}$
forma un insieme universale di gate quantistici per i circuiti D-ND: per ogni unitario
$U \in \text{SU}(2^n)$, esiste una sequenza finita di gate da questo insieme che approssima
$U$ con precisione arbitraria.
\end{proposition}

\begin{proof}
\textbf{Universalit\`a standard:} $\{H, \text{CNOT}, P(\pi/4)\}$ forma un insieme
universale di gate (Nielsen \& Chuang~\cite{Nielsen2010}; teorema di Kitaev-Solovay).

\textbf{Riduzione al limite:} Quando $\delta V \to 0$, i gate D-ND si riducono ai gate
standard: $\HDND \to H$, $\CNOTDND \to \text{CNOT}$, $\PDND \to P(\phi)$ (dalle
Definizioni~\ref{def:hadamard-dnd}--\ref{def:phase-dnd}).

\textbf{Estensione perturbativa:} Per piccoli $\delta V > 0$, ogni gate D-ND differisce
dalla sua controparte standard di $O(\delta V)$:
$\|G_{\text{DND}} - G_{\text{standard}}\| = O(\delta V)$. La composizione di $N$ gate
accumula un errore al pi\`u $N \cdot O(\delta V)$. Poich\'e l'insieme di gate standard \`e
universale e le perturbazioni sono regolari, l'insieme di gate D-ND rimane denso in
SU$(2^n)$ per $\delta V$ sufficientemente piccoli.

\textbf{Stima dell'errore:} Per un circuito di $N$ gate con intensit\`a di emergenza
$\delta V$: $\varepsilon_{\text{approx}} \leq N \cdot C \cdot \delta V$, dove $C$ dipende
dalla geometria del gate. Scegliendo
$\delta V < \varepsilon_{\text{target}}/(N \cdot C)$ si raggiunge la precisione desiderata.
\end{proof}

\begin{openproblem}[Universalit\`a in regime di emergenza forte]
Se $\{\HDND, \CNOTDND, \PDND\}$ rimanga universale per $\delta V \in (0, 1]$ arbitrari \`e
una questione aperta. Una dimostrazione costruttiva richiederebbe famiglie parametriche
esplicite di decomposizioni universali di gate su $\delta V$, oppure un argomento topologico
che mostri che l'insieme di gate genera un sottogruppo denso di SU$(2^n)$ per tutti i valori
di $\delta V$.
\end{openproblem}

%==============================================================================
\section{Modello di circuito}
\label{sec:circuit}
%==============================================================================

\subsection{Regole di composizione dei circuiti D-ND}

Un \textbf{circuito D-ND} $C$ \`e una sequenza di gate $\{G_1, G_2, \ldots, G_k\}$ agenti
su $\rhoDND$, con composizione:
\begin{equation}
C(\rhoDND) = G_k \circ G_{k-1} \circ \cdots \circ G_1(\rhoDND)
\end{equation}

\textbf{Vincolo 4.1 (Consistenza dell'emergenza):} Tra gate consecutivi $G_i$ e $G_{i+1}$,
il campo di emergenza $\emerge$ deve soddisfare:
\begin{equation}
\text{spec}(\emerge_i) \cap \text{spec}(\emerge_{i+1}) \neq \emptyset
\end{equation}
il che assicura la continuit\`a del paesaggio di emergenza.

\textbf{Vincolo 4.2 (Preservazione della coerenza):} La perdita di coerenza totale \`e
limitata:
\begin{equation}
\sum_{i=1}^{k} (1 - \ell_i^*) \leq \Lambda_{\max}
\end{equation}
dove $\Lambda_{\max}$ \`e il budget massimo di coerenza consentito.

\subsection{Modello degli errori e preservazione della coerenza}
\label{sec:error-model}

\begin{proposition}[Soppressione degli errori assistita dall'emergenza]
\label{prop:error-suppression}
Sia $C$ un circuito D-ND di $k$ gate con operatori di Lindblad dipendenti dall'emergenza
$L_k^{\text{DND}}(t) = L_k \cdot (1 - M(t))$. Allora il tasso di errore per gate \`e
soppresso linearmente:
\begin{equation}
\varepsilon(t) = \varepsilon_0 \cdot (1 - M(t))
\label{eq:error-suppression}
\end{equation}
e la fedelt\`a totale del circuito soddisfa:
\begin{equation}
F_{\text{total}} = \prod_{i=1}^{k} [1 - \varepsilon_0(1 - M(t_i))]
\geq (1 - \varepsilon_0)^{k(1-\bar{M})}
\label{eq:fidelity}
\end{equation}
dove $\bar{M} = (1/k)\sum_i M(t_i)$ \`e il fattore medio di emergenza.
\end{proposition}

\emph{Dimostrazione}: Si veda l'Appendice~\ref{app:prop43}.

\textbf{Implicazione:} I circuiti D-ND con forte emergenza media ($\bar{M}$ vicino a 1)
raggiungono un significativo miglioramento della fedelt\`a rispetto ai circuiti standard. La
soppressione \`e lineare per gate ma si compone favorevolmente lungo circuiti profondi.
Questo complementa la correzione di errore quantistica standard.

%==============================================================================
\section{Framework di simulazione}
\label{sec:simulation}
%==============================================================================

\subsection{Approccio IFS (Sistema di funzioni iterate)}
\label{sec:ifs}

Quando l'emergenza \`e forte, un'approssimazione tramite sistema di funzioni iterate diventa
praticabile.

\begin{definition}
\label{def:ifs}
Siano $\{f_1, f_2, \ldots, f_n\}$ mappe di contrazione sullo spazio delle densit\`a
(Definizione~\ref{def:rho-dnd}), con fattori di contrazione
$\{\lambda_1, \ldots, \lambda_n\}$ (ogni $\lambda_i < 1$). Un IFS \`e:
\begin{equation}
\rho_{\text{DND}}^{(n+1)} = \sum_{i=1}^{n} p_i \, f_i(\rho_{\text{DND}}^{(n)})
\end{equation}
dove $p_i$ sono pesi determinati dalla struttura del grafo di emergenza.
\end{definition}

\textbf{Limiti di applicabilit\`a:} Il framework IFS si applica specificamente ai circuiti
D-ND nel regime di emergenza lineare ($M(t) < 0{,}5$, $\lambda < 0{,}5$). \emph{Non}
affermiamo che circuiti quantistici arbitrari possano essere simulati classicamente in tempo
polinomiale. Per i circuiti quantistici completi ($M(t) \to 1$), si applica la simulazione
standard BQP-hard. La struttura IFS emerge naturalmente dalla dinamica D-ND perch\'e
l'operatore di emergenza crea strutture ramificate autosimilari (Paper~C \S3.1). Si veda
Barnsley~\cite{Barnsley1988} per i fondamenti matematici degli IFS.

\subsection{Approssimazione lineare $\Rlinear = P + \lambda \cdot R(t)$ (Formula C7)}
\label{sec:linear-approx}

Per l'implementazione pratica:
\begin{equation}
\Rlinear(t) = P(t) + \lambda \cdot \Remit(t)
\label{eq:linear-approx}
\end{equation}
dove $P(t)$ \`e la componente probabilistica (simulazione quantistica standard con
$\Mproto = 0$), $\lambda$ \`e il coefficiente di accoppiamento dell'emergenza, e:
\begin{equation}
\Remit(t) = \int_0^t M(s) \, e^{-\gamma(t-s)} \, ds
\end{equation}
dove $\gamma$ \`e il tasso di decadimento della memoria di emergenza.

\subsection{Pseudocodice per la simulazione IFS D-ND}
\label{sec:pseudocode}

\begin{lstlisting}[caption={Simulazione di circuito quantistico D-ND tramite IFS},label={lst:ifs}]
Input: rho_0, circuito C, tempo T, lambda, gamma, epsilon
Output: rho_finale, statistiche_misura

1. INIZIALIZZAZIONE
   P(0) <- rho_0
   M(0) <- CalcoloMisuraEmergenza(rho_0)
   t <- 0, dt <- T / NumPassi

2. PER OGNI gate G_i in C:
   3. P(t+dt) <- SimQuantisticaStandard(P(t), G_i, dt)
   4. M(t+dt) <- M(t) + dt * dM/dt(t)
   5. R_emit(t+dt) <- exp(-gamma*dt)*R_emit(t)
                     + dt*M(t)
   6. dU_corr <- MappaEsponenziale(dV, lambda, ell*)
      P(t+dt) <- dU_corr * P(t+dt) * dU_corr_dag
   7. epsilon_eff <- eps_0 * (1 - M(t+dt))
   8. rho_DND(t+dt) <- P(t+dt) + lambda*R_emit(t+dt)
      Rinormalizza
   9. t <- t + dt

10. RESTITUISCI rho_DND(T), misure
\end{lstlisting}

\textbf{Complessit\`a:} $O(n^3 \cdot T)$ quando $\lambda < 0{,}3$ (emergenza debole),
$O(n^4 \cdot T)$ per emergenza moderata, e $O(2^n \cdot T)$ per emergenza forte (regime di
simulazione standard).

\subsection{Analisi degli errori dell'approssimazione lineare}
\label{sec:error-analysis}

\begin{proposition}[Stima dell'errore per l'approssimazione lineare]
\label{prop:error-bound}
Siano $R_{\text{exact}}(t)$ l'evoluzione esatta dello stato D-ND e
$\Rlinear(t) = P(t) + \lambda \cdot \Remit(t)$. Allora:
\begin{equation}
\|R_{\text{exact}}(t) - \Rlinear(t)\| \leq C \cdot \lambda^2 \cdot \|\Remit(t)\|^2
\label{eq:error-bound}
\end{equation}
dove $C \approx T \cdot \log(n) \cdot \rho_{\max}$ per un circuito di profondit\`a $T$ su
$n$ qubit con spettro di emergenza limitato da $\rho_{\max}$.
\end{proposition}

\emph{Traccia della dimostrazione:} L'evoluzione esatta soddisfa
$R_{\text{exact}} = \mathcal{U}_{\text{full}} R(0)$ mentre l'approssimazione lineare usa
$\mathcal{U}_{\text{linear}} = \mathcal{U}_{\text{standard}} + \lambda\mathcal{U}_{\text{correction}}$.
L'errore $\Delta = \mathcal{U}_{\text{full}} - \mathcal{U}_{\text{linear}} = O(\lambda^2)$
per la teoria delle perturbazioni.

\textbf{Regime di validit\`a:} $M(t) < 0{,}5$ (emergenza da precoce a intermedia). Per
$\lambda < 0{,}3$, l'errore relativo rimane inferiore all'1,2\%, adeguato per applicazioni
NISQ. Per $\lambda \geq 0{,}5$, l'approssimazione lineare decade e si rende necessaria la
simulazione quantistica completa.

\subsection{Confronto con la simulazione quantistica standard}
\label{sec:comparison}

\begin{table}[H]
\centering
\begin{tabular}{lll}
\toprule
\textbf{Aspetto} & \textbf{Standard} & \textbf{D-ND lineare} \\
\midrule
Complessit\`a temporale & $O(2^n \cdot T)$ & $O(n^3 \cdot T)$ quando $\lambda < 0{,}3$ \\
Memoria & $O(2^n)$ & $O(n^2)$ \\
Accuratezza (em.\ bassa) & Perfetta & $\sim$99\% \\
Hardware & Processore quantistico & Classico + oracolo di emergenza \\
Gestione errori & QEC a livello di circuito & Soppressione assistita dall'emergenza \\
\bottomrule
\end{tabular}
\caption{Confronto tra gli approcci di simulazione.}
\label{tab:comparison}
\end{table}

%==============================================================================
\section{Applicazioni e vantaggio quantistico}
\label{sec:applications}
%==============================================================================

\subsection{Ricerca quantistica con accelerazione emergente}
\label{sec:quantum-search}

\textbf{Problema:} Ricercare un elemento marcato in un database non ordinato di dimensione
$N$. L'algoritmo di Grover standard raggiunge un'accelerazione $O(\sqrt{N})$.

\textbf{Miglioramento D-ND:} Utilizzando gate $\HDND$ che pesano preferenzialmente i rami
ad alta emergenza, possiamo concentrare la densit\`a possibilistica sull'elemento marcato.

\begin{conjecture}
\label{conj:search}
Per circuiti in cui l'emergenza \`e controllata ($\Mproto \propto t$), la ricerca quantistica
D-ND pu\`o ottenere un miglioramento per fattore costante rispetto al Grover standard, con
complessit\`a di interrogazione $O(\sqrt{N}/\alpha)$ dove $\alpha \geq 1$ \`e un fattore di
amplificazione dell'emergenza.
\end{conjecture}

\textbf{Osservazione sui limiti inferiori:} Il teorema BBBV (Bennett et
al.~\cite{Bennett1997}) stabilisce che qualsiasi ricerca quantistica richiede
$\Omega(\sqrt{N})$ interrogazioni all'oracolo. Qualsiasi accelerazione D-ND oltre questo
limite richiederebbe un modello di oracolo fondamentalmente diverso. Il miglioramento
affermato qui \`e un fattore costante $\alpha$ all'interno del modello di oracolo standard.

\subsection{Calcolo quantistico topologico}
\label{sec:topological}

Il framework D-ND \`e naturalmente adatto al calcolo quantistico topologico:
(1)~gli stati sono protetti da invarianti topologici (cicli omologici nel grafo di
emergenza); (2)~l'intreccio (braiding) tramite Shortcut$_{\text{DND}}$ implementa
efficientemente lo scambio di anioni non abeliani; (3)~il campo di emergenza fornisce una
protezione topologica aggiuntiva oltre alla soppressione intrinseca degli errori.

Per emergenza moderata, la riduzione del sovraccarico \`e:
\begin{equation}
\text{Riduzione del sovraccarico} = 1 - \frac{\Mproto}{\Mdist + \Ment}
\end{equation}

\subsection{Posizionamento nell'ambito di BQP vs.\ BPP}
\label{sec:bqp}

Il framework D-ND fornisce un meccanismo distinto per l'accelerazione quantistica:
\begin{enumerate}
\item \textbf{Complessit\`a assistita dall'emergenza:} La misura di emergenza $M(t)$
fornisce una risorsa controllabile in modo continuo.
\item \textbf{Classe di complessit\`a ibrida:} Definiamo BQP$_{\text{DND}}$ come la classe
dei problemi risolvibili da circuiti D-ND con sovraccarico di emergenza polinomiale.
\item \textbf{Vantaggio dalla soppressione degli errori:} La
Proposizione~\ref{prop:error-suppression} mostra
$\varepsilon(t) = \varepsilon_0(1 - M(t))$, consentendo circuiti pi\`u profondi con
emergenza forte.
\end{enumerate}

\subsection{Problema aperto: vantaggio quantistico tramite amplificazione di ampiezza D-ND}
\label{sec:open-problem}

\begin{openproblem}
Dimostrare o confutare che i circuiti quantistici D-ND possano raggiungere un'accelerazione
superpolinomiale per una classe naturale di problemi, utilizzando un'amplificazione di
ampiezza modulata dall'emergenza distinta dal Grover standard.
\end{openproblem}

\textbf{Approccio candidato:} Inizializzare a $\NT$. Applicare l'oracolo modulato
dall'emergenza
$O_{\text{DND}}(t) = I - (1 + M(t))|x^*\rangle\langle x^*|$ e l'operatore di diffusione
$D_{\text{DND}}(t) = (1-M(t)) D_{\text{Grover}} + M(t) D_{\text{random}}$. Il conteggio
delle iterazioni diventa:
\begin{equation}
T_{\text{DND}} \sim \frac{\sqrt{N/k}}{\sqrt{1 + \lambda\Psi_C}}
\end{equation}
dove $\Psi_C$ \`e un fattore di miglioramento della coerenza derivato dalla struttura del
circuito. La complessit\`a totale delle interrogazioni rimane $\Omega(\sqrt{N/k})$ per il
teorema BBBV, quindi ci\`o va inteso come un miglioramento per fattore costante per $n$
fissato.

\subsection{Connessione con il campionamento termodinamico: il ponte THRML/Omega-Kernel}
\label{sec:thrml}

Recenti sviluppi nel calcolo termodinamico da parte di Extropic AI forniscono un percorso
diretto di validazione sperimentale per la teoria dell'informazione quantistica D-ND. La
libreria THRML/Omega-Kernel implementa il campionamento di modelli grafici probabilistici
attraverso principi termodinamici, con un'architettura isomorfa al framework D-ND.

\subsubsection{SpinNode come dipolo D-ND}

Lo SpinNode di THRML con stati $\{-1, +1\}$ corrisponde al dipolo singolare-duale D-ND:
\begin{equation}
\text{SpinNode} \in \{-1, +1\} \leftrightarrow
\text{dipolo D-ND} \in \{|\varphi_+\rangle, |\varphi_-\rangle\}
\end{equation}
Il modello basato sull'energia di Ising $E = -\sum_{i,j} J_{ij}s_is_j - \sum_i h_is_i$ si
mappa sul potenziale efficace D-ND $V_{\text{eff}}$ con $J_{ij}$ come hamiltoniana di
interazione e $h_i$ come potenziale a singola particella.

\subsubsection{Campionamento di Gibbs a blocchi come emergenza iterativa}

Il campionamento di Gibbs a blocchi di THRML---che suddivide il grafo in blocchi alternati con
aggiornamenti condizionali---\`e isomorfo al processo di emergenza D-ND: lo stato iniziale
casuale $\leftrightarrow$ campionamento da $\NT$; ogni passata di Gibbs $\leftrightarrow$
un'applicazione di $\emerge$; convergenza all'equilibrio $\leftrightarrow$ emergenza completa
con $M \approx 1$.

Ogni passata di Gibbs campiona
$p(s_B|s_{B^c}) \propto \exp(-\beta E(s_B, s_{B^c}))$, dove il fattore di Boltzmann
$\exp(-\beta E)$ corrisponde all'amplificazione selettiva dei rami ad alta coerenza da parte
dell'operatore di emergenza.

\subsubsection{Corrispondenza dei gate}

I quattro gate D-ND si mappano sulle operazioni THRML:
$\HDND \leftrightarrow$ ridistribuzione a blocchi;
$\CNOTDND \leftrightarrow$ aggiornamento condizionale inter-blocco;
$\PDND \leftrightarrow$ modulazione di temperatura/bias;
Shortcut$_{\text{DND}} \leftrightarrow$ aggiornamento simultaneo multi-blocco.

\subsubsection{Importanza per la validazione sperimentale}

Il framework THRML fornisce un percorso diretto di validazione sperimentale:
(1)~codebase esistente e funzionante (JAX, accelerata su GPU);
(2)~roadmap hardware termodinamico (processori Extropic AI);
(3)~ponte ibrido classico-quantistico;
(4)~verifica dell'emergenza tramite distribuzioni di probabilit\`a condizionali;
(5)~compatibilit\`a algoritmica (varianti di Grover, VQE, QAOA).

\subsection{Metriche di simulazione dal framework ibrido D-ND}
\label{sec:metrics}

Quattro metriche chiave quantificano la transizione ibrida quantistico-classica:

\textbf{Misura di coerenza:}
$C(t) = |\langle\Psi(t)|\Psi(0)\rangle|^2 = \text{Tr}[\rho(t)\rho(0)]$.
Quando $C(t) = 1$: coerenza perfetta. Quando $C(t) \to 0$: decoerenza completa.

\textbf{Misura di tensione:}
$T(t) = \|\partial\rho/\partial t\|^2 = \text{Tr}[(\dot{\rho})^\dagger\dot{\rho}]$.
$T(t)$ elevata: emergenza attiva. $T(t)$ bassa: equilibrio.

\textbf{Tasso di emergenza:}
$dM/dt = (d/dt)[1 - |\langle\text{NT}|U(t)\emerge|\text{NT}\rangle|^2]$.
$dM/dt$ rapido: forte accoppiamento di emergenza.

\textbf{Criterio di convergenza:}
$|C(t) - C(t-1)| < \varepsilon$ per una tolleranza $\varepsilon$ specificata dall'utente.

%==============================================================================
\section{Conclusioni}
\label{sec:conclusions}
%==============================================================================

Abbiamo formalizzato gli aspetti computazionali quantistici del framework D-ND:

\begin{enumerate}
\item \textbf{Densit\`a possibilistica $\rhoDND$} unifica la sovrapposizione quantistica con
la struttura di emergenza, abilitando uno spazio informativo pi\`u ricco.

\item \textbf{Quattro gate modificati} ($\HDND$, $\CNOTDND$, $\PDND$,
Shortcut$_{\text{DND}}$) forniscono un insieme completo di gate adattato alla dinamica D-ND.

\item \textbf{Universalit\`a dei gate} (Proposizione~\ref{prop:universality}) dimostra che
l'insieme di gate pu\`o approssimare unitari SU$(2^n)$ arbitrari nel regime perturbativo.
L'universalit\`a in regime di emergenza forte rimane un problema aperto.

\item \textbf{Soppressione degli errori assistita dall'emergenza}
(Proposizione~\ref{prop:error-suppression}) mostra il miglioramento della fedelt\`a con
$\varepsilon(t) = \varepsilon_0(1-M(t))$, complementare alla QEC standard.

\item \textbf{Framework di simulazione lineare} consente l'approssimazione classica in tempo
polinomiale quando $\lambda < 0{,}3$, riducendo i requisiti hardware.

\item \textbf{Applicazioni} alla ricerca quantistica (miglioramento per fattore costante),
al calcolo quantistico topologico (sovraccarico ridotto) e al ponte con il calcolo
termodinamico THRML sono dimostrate.
\end{enumerate}

\textbf{Direzioni future:}
Implementazione hardware su qubit superconduttori;
libreria di algoritmi D-ND per ottimizzazione e apprendimento automatico;
realizzazione efficiente dell'oracolo di emergenza;
integrazione con algoritmi quantistici variazionali;
validazione sperimentale su dispositivi NISQ.

%==============================================================================
\section*{Ringraziamenti}
%==============================================================================

Questo lavoro si basa sul framework teorico D-ND sviluppato nei Paper A--E. Gli autori
ringraziano le comunit\`a di ricerca in informazione quantistica e dinamica dell'emergenza
per le intuizioni fondamentali.

%==============================================================================
% APPENDICES
%==============================================================================

\appendix

\section{Dimostrazione della Proposizione~\ref{prop:hilbert-embedding}}
\label{app:prop22}

\begin{proof}
\textbf{Costruzione dell'operatore densit\`a:} Quando $\Mproto = 0$, si ha
$M(i) = \Mdist(i) + \Ment(i) \geq 0$ per ogni stato base $|i\rangle$. Si definisca
$\Sigma M = \sum_i M(i) > 0$. Allora:
\begin{equation}
\hat{\rho}_{\text{DND}} = \sum_i \frac{M(i)}{\Sigma M} |i\rangle\langle i|
\end{equation}

\textbf{Propriet\`a della matrice densit\`a:}
(i)~$\text{Tr}[\hat{\rho}_{\text{DND}}] = \sum_i M(i)/\Sigma M = 1$;
(ii)~tutti gli autovalori $M(i)/\Sigma M \geq 0$;
(iii)~$\hat{\rho}_{\text{DND}}$ \`e diagonale in una base reale, quindi autoaggiunta.

\textbf{Prodotto interno pesato:} Per $|\psi\rangle = \sum_i a_i|i\rangle$ e
$|\phi\rangle = \sum_j b_j|j\rangle$:
\begin{equation}
\langle\psi|\phi\rangle_{\text{DND}} = \text{Tr}[|\psi\rangle\langle\phi|\hat{\rho}_{\text{DND}}]
= \sum_i a_i^* b_i \frac{M(i)}{\Sigma M}
\end{equation}

\textbf{Verifica dello spazio di Hilbert:}
La sesquilinearit\`a segue dalla linearit\`a della traccia e della somma.
Simmetria coniugata: $\langle\psi|\phi\rangle^*_{\text{DND}} = \sum_i a_i b_i^* M(i)/\Sigma M
= \langle\phi|\psi\rangle_{\text{DND}}$.
Definita positivit\`a: $\langle\psi|\psi\rangle_{\text{DND}} = \sum_i |a_i|^2 M(i)/\Sigma M \geq 0$,
con uguaglianza se e solo se $a_i = 0$ per tutti gli $i$ nel supporto di $M$.

\textbf{Recupero della regola di Born:}
$P(i) = \langle i|\hat{\rho}_{\text{DND}}|i\rangle = M(i)/\Sigma M$.
Quando $M(i)$ \`e uniforme, il prodotto interno pesato si riduce al prodotto interno
standard.
\end{proof}

\section{Dimostrazione della Proposizione~\ref{prop:error-suppression}}
\label{app:prop43}

\begin{proof}
\textbf{Equazione di Lindblad dipendente dall'emergenza:}
L'evoluzione con decoerenza e accoppiamento dell'emergenza segue:
\begin{equation}
\frac{d\rho}{dt} = -\frac{i}{\hbar}[H, \rho] + \mathcal{D}_{\text{DND}}[\rho]
\end{equation}
dove $\mathcal{D}_{\text{DND}}[\rho] = \sum_k (L_k^{\text{DND}} \rho (L_k^{\text{DND}})^\dagger
- \tfrac{1}{2}\{(L_k^{\text{DND}})^\dagger L_k^{\text{DND}}, \rho\})$ con
$L_k^{\text{DND}}(t) = L_k \cdot (1 - M(t))$.

\textbf{Errore per gate:} Il tasso effettivo di Lindblad scala come $(1-M(t))^2$ all'ordine
dominante. Per $\varepsilon_0 \ll 1$, l'errore per gate \`e
$\varepsilon(t) = \varepsilon_0(1-M(t))$ (da
$\|L_k^{\text{DND}}\| = (1-M(t))\|L_k\|$).

\textbf{Fedelt\`a del circuito:} La fedelt\`a per gate \`e
$F_i = 1 - \varepsilon_0(1-M(t_i))$. Per $k$ gate:
\begin{equation}
\ln F_{\text{total}} = \sum_{i=1}^{k} \ln[1 - \varepsilon_0(1-M(t_i))]
\approx -\varepsilon_0 \sum_{i=1}^{k} (1-M(t_i)) = -\varepsilon_0 k(1-\bar{M})
\end{equation}
Dunque $F_{\text{total}} \approx e^{-\varepsilon_0 k(1-\bar{M})}$.

\textbf{Confronto:} Circuito standard ($M=0$):
$F_{\text{std}} \approx e^{-\varepsilon_0 k}$.
Miglioramento della fedelt\`a D-ND: $F_{\text{DND}}/F_{\text{std}} = e^{\varepsilon_0 k\bar{M}}$.

\textbf{Rappresentazione di Kraus:} Gli operatori di Kraus
$K_0 = \sqrt{1-\varepsilon_0(1-M(t))}I$ e
$K_j = \sqrt{\varepsilon_0(1-M(t))/3}\,\sigma_j$ ($j=1,2,3$) soddisfano la completezza
$\sum_j K_j^\dagger K_j = I$ e confermano la probabilit\`a di errore
$\varepsilon_0(1-M(t))$ per gate.
\end{proof}

%==============================================================================
% REFERENCES
%==============================================================================

\begin{thebibliography}{20}

\bibitem{Dirac1930}
P.~A.~M. Dirac,
\emph{The Principles of Quantum Mechanics} (Oxford University Press, 1930).

\bibitem{Nielsen2010}
M.~A. Nielsen and I.~L. Chuang,
\emph{Quantum Computation and Quantum Information} (Cambridge University Press, 2010).

\bibitem{Aharonov1997}
D.~Aharonov and M.~Ben-Or,
``Fault-tolerant quantum computation with constant error,''
\emph{SIAM J. Comput.} \textbf{38}(4), 1207--1282 (1997).

\bibitem{Nayak2008}
C.~Nayak, S.~H. Simon, A.~Stern, M.~Freedman, and S.~Das Sarma,
``Non-Abelian anyons and topological quantum computation,''
\emph{Rev. Mod. Phys.} \textbf{80}(3), 1083 (2008).

\bibitem{AspuruGuzik2005}
A.~Aspuru-Guzik, A.~D. Dutoi, P.~J. Love, and M.~Head-Gordon,
``Simulated quantum computation of molecular energies,''
\emph{Science} \textbf{309}(5741), 1704--1707 (2005).

\bibitem{Harrow2009}
A.~W. Harrow, A.~Hassidim, and S.~Lloyd,
``Quantum algorithm for linear systems of equations,''
\emph{Phys. Rev. Lett.} \textbf{103}(15), 150502 (2009).

\bibitem{Grover1997}
L.~K. Grover,
``Quantum mechanics helps in searching for a needle in a haystack,''
\emph{Phys. Rev. Lett.} \textbf{79}(2), 325 (1997).

\bibitem{Kitaev2003}
A.~Y. Kitaev,
``Fault-tolerant quantum computation by anyons,''
\emph{Ann. Phys.} \textbf{303}(1), 2--30 (2003).

\bibitem{PaperA}
D-ND Research Collective,
``Quantum Emergence from Primordial Potentiality: The Dual-Non-Dual Framework for State Differentiation''
(this volume).

\bibitem{PaperE}
D-ND Research Collective,
``Cosmological Extension of the Dual-Non-Dual Framework''
(this volume).

\bibitem{Hutchinson1981}
J.~E. Hutchinson,
``Fractals and self-similarity,''
\emph{Indiana Univ. Math. J.} \textbf{30}(5), 713--747 (1981).

\bibitem{Falconer1990}
K.~J. Falconer,
\emph{Fractal Geometry: Mathematical Foundations and Applications} (John Wiley \& Sons, 1990).

\bibitem{Shor1997}
P.~W. Shor,
``Polynomial-time algorithms for prime factorization and discrete logarithms on a quantum computer,''
\emph{SIAM J. Comput.} \textbf{26}(5), 1484--1509 (1997).

\bibitem{Preskill2018}
J.~Preskill,
``Quantum computing in the NISQ era and beyond,''
\emph{Quantum} \textbf{2}, 79 (2018).

\bibitem{Wootters1998}
W.~K. Wootters,
``Entanglement of formation of an arbitrary state of two qubits,''
\emph{Phys. Rev. Lett.} \textbf{80}(10), 2245 (1998).

\bibitem{Vidal2002}
G.~Vidal and R.~F. Werner,
``Computable measure of entanglement,''
\emph{Phys. Rev. A} \textbf{65}(3), 032314 (2002).

\bibitem{Barnsley1988}
M.~F. Barnsley,
\emph{Fractals Everywhere} (Academic Press, 1988).

\bibitem{Bennett1997}
C.~H. Bennett, E.~Bernstein, G.~Brassard, and U.~Vazirani,
``Strengths and weaknesses of quantum computing,''
\emph{SIAM J. Comput.} \textbf{26}(5), 1510--1523 (1997).

\bibitem{Lupasco1951}
S.~Lupasco,
\emph{Le principe d'antagonisme et la logique de l'\'energie} (Hermann, 1951).

\bibitem{Nicolescu2002}
B.~Nicolescu,
\emph{Manifesto of Transdisciplinarity} (SUNY Press, 2002).

\end{thebibliography}

\end{document}
