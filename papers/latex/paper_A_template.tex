%==============================================================================
% PAPER A - FOUNDATION PAPER
% Target Journal: Physical Review A
% Document Class: revtex4-2
%==============================================================================
%
% D-ND COSMOLOGY FRAMEWORK - LATEX CONVERSION TEMPLATE
%
% This template demonstrates the complete LaTeX structure for Paper A, the
% foundational paper of the D-ND academic suite. It serves as a reference
% for adapting the other six papers to their respective target journals.
%
% JOURNAL ADAPTATION GUIDE:
% - Paper A → Physical Review A (this file)
% - Paper B → Physical Review E (change: \documentclass{revtex4-2})
% - Paper C → Journal of Mathematical Physics (change: \documentclass[11pt]{aip})
% - Paper D → Foundations of Physics (change: \documentclass[twocolumn]{springer})
% - Paper E → Classical and Quantum Gravity (change: \documentclass{iopart})
% - Paper F → Quantum (change: \documentclass{quantumarticle})
% - Paper G → Cognitive Science/Minds and Machines (change: \documentclass{springer})
%
%==============================================================================

\documentclass[aps,pra,11pt,notitlepage,nofootinbib]{revtex4-2}

%==============================================================================
% PACKAGES - CORE MATHEMATICS AND PHYSICS
%==============================================================================

\usepackage[utf-8]{inputenc}          % UTF-8 text encoding
\usepackage[T1]{fontenc}              % Extended font encoding
\usepackage{amsmath}                  % AMS mathematical environments
\usepackage{amssymb}                  % AMS symbols (extended)
\usepackage{mathrsfs}                 % Ralph Smith's formal script font (for \mathcal)
\usepackage{braket}                   % Braket notation: \ket, \bra, \braket
\usepackage{amsthm}                   % Theorem environment definitions

%==============================================================================
% PACKAGES - CROSS-REFERENCES AND CITATIONS
%==============================================================================

\usepackage{hyperref}                 % Hyperlinks in PDF
\usepackage{cleveref}                 % Smart cross-references: \cref, \Cref
\usepackage[square,numbers,sort&compress]{natbib}  % Bibliography with citations

%==============================================================================
% PACKAGES - DOCUMENT FORMATTING
%==============================================================================

\usepackage{geometry}                 % Page layout customization
\usepackage{setspace}                 % Line spacing control
\usepackage{graphicx}                 % Graphics/figures inclusion
\usepackage{float}                    % Figure placement options [H], [t], etc.

%==============================================================================
% D-ND SHARED NOTATION PACKAGE
%==============================================================================
% This package (dnd_shared.sty) contains all D-ND specific notation commands
% and theorem environments, ensuring consistency across all papers.
%
% Located at: domain/AWARENESS/3_π_PRAGMATIC/latex/dnd_shared.sty
% The .sty file is included in this directory for compilation.

\usepackage{dnd_shared}

%==============================================================================
% HYPERREF CONFIGURATION (PDF links and bookmarks)
%==============================================================================

\hypersetup{
    colorlinks=true,
    linkcolor=blue,
    citecolor=blue,
    urlcolor=blue,
    bookmarksnumbered=true,
    pdftitle={D-ND Cosmology Framework: Foundation Theory},
    pdfauthor={D-ND Research Collective},
    pdfsubject={Quantum Cosmology, Emergence, Classical Limit},
    pdfkeywords={D-ND, emergence, quantum, cosmology, classical limit}
}

%==============================================================================
% GEOMETRY CONFIGURATION
%==============================================================================
% Note: revtex4-2 manages margins automatically; customize if needed

\geometry{
    margin=1in,
    includehead=true,
    includefoot=true
}

%==============================================================================
% CUSTOM MACROS (SUPPLEMENTARY)
%==============================================================================
% These supplement the shared commands in dnd_shared.sty

% Quantum mechanics notation
\newcommand{\Hilbert}{\mathcal{H}}    % Hilbert space
\newcommand{\HilbertNT}{\Hilbert_{\text{NT}}}  % NT Hilbert space
\newcommand{\Obs}{\mathcal{O}}        % Observable operator

% D-ND specific derived quantities
\newcommand{\emergence@threshold}{\emerge_{\text{th}}}  % Emergence threshold
\newcommand{\emergence@critical}{\emerge^*}  % Critical emergence value
\newcommand{\coupling@strength}{\alpha_{\text{couple}}}  % Coupling strength

% Partial derivatives shorthand
\newcommand{\pd}[2]{\frac{\partial #1}{\partial #2}}  % ∂/∂
\newcommand{\pdd}[3]{\frac{\partial^2 #1}{\partial #2 \partial #3}}  % ∂²/∂∂

% Time evolution
\newcommand{\evolve}[1]{#1(t)}        % Time-evolved quantity

%==============================================================================
% BEGIN DOCUMENT
%==============================================================================

\begin{document}

%==============================================================================
% TITLE PAGE / ARTICLE METADATA
%==============================================================================

\title{The D-ND Framework: Quantum Emergence and Classical Reality}

\author{D-ND Research Collective}
\affiliation{Department of Theoretical Physics, Foundation Research Institute}

\date{\today}

%==============================================================================
% ABSTRACT
%==============================================================================

\begin{abstract}

The Domain-Nested-Dynamics (D-ND) framework presents a unified theory of quantum
emergence and the emergence of classical reality from quantum mechanics. This
foundation paper establishes the core axioms, mathematical structure, and physical
implications of the framework. We introduce the emergence operator $\emerge$, the
order parameter $\orderparam$, and the coupling constants $\chiDND$ and $\lambdaDND$.
The framework demonstrates how classical dynamics emerge from quantum systems through
a process of nested domain interactions, with implications for cosmology, quantum
measurement, and the classical limit problem. We establish the theoretical foundation
upon which the subsequent six papers build, extending to experimental signatures,
mathematical generalizations, and cognitive physics applications.

\end{abstract}

%==============================================================================
% SECTION 1: INTRODUCTION
%==============================================================================

\section{Introduction}
\label{sec:introduction}

The relationship between quantum mechanics and classical physics remains one of the
most fundamental unresolved questions in physics. The emergence of classical behavior
from quantum systems---the classical limit problem---continues to challenge our
understanding of the foundations of physics. The D-ND framework addresses this through
a novel mechanism of nested domain dynamics.

\subsection{Motivation and Context}
\label{subsec:motivation}

Classical mechanics is an extraordinarily successful theory for describing macroscopic
phenomena. Yet quantum mechanics governs the microscopic realm. How classical physics
emerges from quantum mechanics is not obvious, and many approaches have been proposed:
decoherence theory, Bohmian mechanics, many-worlds interpretations, and others.

The D-ND framework takes a different approach, proposing that emergence occurs through
a process of nested domain interactions, where each domain operates according to specific
dynamical laws that couple through emergence operators.

\subsection{Paper Roadmap}
\label{subsec:roadmap}

This paper (Paper A) establishes the theoretical foundation. The subsequent papers extend
this framework:
\begin{itemize}
    \item \textbf{Paper B}: Physical processes and experimental signatures in quantum systems
    \item \textbf{Paper C}: Mathematical generalizations using differential geometry
    \item \textbf{Paper D}: Philosophical foundations and physics principles
    \item \textbf{Paper E}: Cosmological implications and gravitational aspects
    \item \textbf{Paper F}: Quantum information and entanglement structures
    \item \textbf{Paper G}: Cognitive applications and consciousness physics
\end{itemize}

%==============================================================================
% SECTION 2: AXIOMS OF THE D-ND FRAMEWORK
%==============================================================================

\section{Axioms of the D-ND Framework}
\label{sec:axioms}

We establish the D-ND framework through a set of core axioms. These axioms are
fundamental assumptions that define the structure of the theory.

\begin{axiom}[Nested Domain Structure]
\label{ax:nested}
Physical reality consists of nested domains $D_i$, where each domain $i$ operates
according to its own dynamical laws $\mathcal{L}_i$, with interactions governed by
emergence operators $\emerge_i$.
\end{axiom}

\begin{axiom}[Emergence Operator Principle]
\label{ax:emergence}
Between any two nested domains $D_i$ and $D_j$, there exists an emergence operator
$\emerge_{ij}$ that characterizes the flow of information and correlations from
quantum $(D_i)$ to classical $(D_j)$ domains.
\end{axiom}

\begin{axiom}[Order Parameter Dynamics]
\label{ax:order}
The emergence of classical behavior is characterized by an order parameter
$\orderparam$ that measures the degree of classical behavior within the nested
domain structure, with $0 \le \orderparam \le 1$.
\end{axiom}

\begin{axiom}[Conservation of Information]
\label{ax:conservation}
Total information content, including quantum correlations and classical observables,
is conserved across domain transitions, though information distribution changes
through emergence operators.
\end{axiom}

\begin{remark}
These four axioms differ from conventional axioms of quantum mechanics. Rather than
assuming the wavefunction and Born rule as fundamental, D-ND derives these from nested
domain interactions. This provides an alternative foundational perspective.
\end{remark}

%==============================================================================
% SECTION 3: EMERGENCE DYNAMICS
%==============================================================================

\section{Emergence Dynamics}
\label{sec:emergence}

The core mechanism of the D-ND framework is the emergence operator and its dynamics.

\subsection{The Emergence Operator}
\label{subsec:emergence_operator}

The emergence operator $\emerge$ acts on quantum states in the form:

\begin{equation}
\label{eq:emergence_operator}
\emerge \ket{\psi} = \lambda_{\emerge} \ket{\psi_{\text{classical}}}
\end{equation}

where $\lambda_{\emerge}$ is the emergence coupling strength and $\ket{\psi_{\text{classical}}}$
is the classical approximation of the quantum state.

\subsection{Order Parameter Evolution}
\label{subsec:order_evolution}

The order parameter $\orderparam$ governs the transition from quantum to classical:

\begin{equation}
\label{eq:order_parameter}
\frac{d}{dt}\orderparam = -\Gamma \left(\orderparam - \orderparam_{\text{eq}}\right)
\end{equation}

where $\Gamma$ is a relaxation rate and $\orderparam_{\text{eq}}$ is the equilibrium
value determined by the system parameters.

\begin{proposition}
\label{prop:order_convergence}
For any system satisfying the D-ND axioms, the order parameter $\orderparam$ converges
to an equilibrium value $\orderparam_{\text{eq}}$ exponentially in time.
\end{proposition}

\begin{proof}
The differential equation \eqref{eq:order_parameter} is a first-order linear ODE with
exponential solution:
\begin{equation}
\orderparam(t) = \orderparam_{\text{eq}} + \left(\orderparam(0) - \orderparam_{\text{eq}}\right)
e^{-\Gamma t}
\end{equation}
This shows convergence to $\orderparam_{\text{eq}}$ as $t \to \infty$.
\end{proof}

\subsection{Measurement and Emergence}
\label{subsec:measurement}

The emergence measurement function $\emeasure$ quantifies the classical behavior:

\begin{equation}
\label{eq:emergence_measure}
\emeasure = \frac{1}{2}\left(1 + \tanh\left(\beta(\emerge - \emerge_{\text{th}})\right)\right)
\end{equation}

where $\emerge_{\text{th}}$ is a threshold emergence value and $\beta$ controls the
steepness of the transition.

%==============================================================================
% SECTIONS 4-5: CLASSICAL LIMIT (COMBINED)
%==============================================================================

\section{Classical Limit and Effective Dynamics}
\label{sec:classical_limit}

\subsection{Transition to Classical Mechanics}
\label{subsec:classical_transition}

As the order parameter $\orderparam$ increases toward unity, the system exhibits
increasingly classical behavior. The Hamiltonian in the classical limit becomes:

\begin{equation}
\label{eq:classical_hamiltonian}
H_{\text{classical}} = \lim_{\orderparam \to 1} \emerge^{\dagger} H_{\text{quantum}} \emerge
\end{equation}

\subsection{Correspondence with Classical Physics}
\label{subsec:correspondence}

The D-ND framework reproduces classical mechanics in the appropriate limit:

\begin{theorem}[Classical Correspondence Limit]
\label{thm:classical_correspondence}
In the limit where $\orderparam \to 1$ and $\hbar_{\text{eff}} \to 0$, the equations
of motion derived from D-ND reduce to classical Hamiltonian equations with:
\begin{equation}
\label{eq:classical_eom}
\dot{q}_i = \frac{\partial H}{\partial p_i}, \quad \dot{p}_i = -\frac{\partial H}{\partial q_i}
\end{equation}
\end{theorem}

%==============================================================================
% SECTION 6: CURVATURE AND DIFFERENTIAL GEOMETRY
%==============================================================================

\section{Curvature and Geometric Structure}
\label{sec:curvature}

The space of emergence states admits a natural geometric structure characterized by
curvature.

\subsection{Emergence Manifold}
\label{subsec:emergence_manifold}

The space of possible emergence values forms a manifold $\mathcal{M}_{\emerge}$ with
metric tensor:

\begin{equation}
\label{eq:emergence_metric}
g_{ij} = \expect{\frac{\partial \emerge}{\partial \xi^i}\frac{\partial \emerge}{\partial \xi^j}}
\end{equation}

where $\xi^i$ are coordinates on the manifold.

\subsection{Ricci Curvature}
\label{subsec:ricci}

The Ricci curvature tensor encodes information about the local structure of emergence:

\begin{equation}
\label{eq:ricci_curvature}
R_{ij} = \frac{\partial^2 \ln g}{\partial \xi^i \partial \xi^j}
\end{equation}

Higher curvature indicates steeper emergence transitions.

%==============================================================================
% SECTION 7: EXPERIMENTAL SIGNATURES AND OBSERVATIONS
%==============================================================================

\section{Experimental Signatures}
\label{sec:experimental}

The D-ND framework makes predictions testable through various experimental approaches.

\subsection{Signatures in Quantum Systems}
\label{subsec:quantum_signatures}

Key experimental signatures of D-ND emergence include:

\begin{enumerate}
    \item \textbf{Emergence threshold detection}: The emergence function $\emeasure$
    should produce measurable signatures at the threshold $\emerge_{\text{th}}$.

    \item \textbf{Order parameter dynamics}: The temporal evolution of $\orderparam$
    should follow predictions of \eqref{eq:order_parameter}.

    \item \textbf{Coupling constant measurements}: The coupling $\chiDND$ should be
    determinable from spectral measurements.
\end{enumerate}

\subsection{Cosmological Tests}
\label{subsec:cosmo_tests}

The cosmological implications of D-ND lead to testable predictions for:
\begin{itemize}
    \item Structure formation timescales
    \item Gravitational wave signatures
    \item Cosmic microwave background properties
\end{itemize}

These are detailed in Paper E (Classical and Quantum Gravity).

%==============================================================================
% SECTION 8: CONCLUSION
%==============================================================================

\section{Conclusion}
\label{sec:conclusion}

The D-ND framework provides a novel approach to the classical limit problem, grounded
in a set of fundamental axioms about nested domain structure and emergence operators.
We have established:

\begin{enumerate}
    \item The core axioms defining the D-ND framework (\secref{axioms})
    \item The emergence operator and order parameter dynamics (\secref{emergence})
    \item Correspondence with classical mechanics in the appropriate limit (\secref{classical_limit})
    \item The geometric structure underlying emergence (\secref{curvature})
    \item Experimental signatures for validation (\secref{experimental})
\end{enumerate}

This foundation supports the extended framework developed in the companion papers,
which address:
\begin{itemize}
    \item Quantum measurement and decoherence (Paper B)
    \item Mathematical generalizations (Paper C)
    \item Philosophical foundations (Paper D)
    \item Cosmology and gravity (Paper E)
    \item Quantum information (Paper F)
    \item Cognitive physics (Paper G)
\end{itemize}

Future work will refine the experimental signatures, extend to quantum field theory,
and develop practical applications in quantum technology.

%==============================================================================
% REFERENCES
%==============================================================================

\begin{thebibliography}{99}

% Format: \bibitem{key} Author(s), Title, Journal Volume (Year) Pages

\bibitem{zurek2003} Zurek, W.~H., Rev.\ Mod.\ Phys.\ \textbf{75}, 715 (2003).

\bibitem{wallace2012} Wallace, D., The Emergent Multiverse (Oxford University Press, 2012).

\bibitem{griffiths1984} Griffiths, R.~B., J.\ Stat.\ Phys.\ \textbf{36}, 219 (1984).

\bibitem{omnes1994} Omnès, R., The Interpretation of Quantum Mechanics (Princeton
University Press, 1994).

\bibitem{caldirola1976} Caldirola, P., Physica \textbf{A86}, 305 (1976).

\bibitem{lamme2018} Lamme, V.~A.~F., Trends Cogn.\ Sci.\ \textbf{22}, 804 (2018).

% Add additional references as needed
% Using natbib format: \cite{key}, \citep{key}, \citet{key}

\end{thebibliography}

%==============================================================================
% APPENDICES (OPTIONAL)
%==============================================================================

\appendix

\section{Mathematical Notation Reference}
\label{app:notation}

This appendix provides a quick reference for the notation used throughout the paper.

\subsection{Quantum Mechanical Notation}

\begin{center}
\begin{tabular}{ll}
\hline
Symbol & Definition \\
\hline
$\ket{\psi}$ & Quantum state (ket vector) \\
$\bra{\psi}$ & Dual state (bra vector) \\
$\braket{\phi}{\psi}$ & Inner product \\
$\Hilbert$ & Hilbert space \\
$\Obs$ & Observable operator \\
\hline
\end{tabular}
\end{center}

\subsection{D-ND Framework Notation}

\begin{center}
\begin{tabular}{ll}
\hline
Symbol & Definition \\
\hline
$\NT$ & Nested Topology state $\ket{\text{NT}}$ \\
$\emerge$ & Emergence operator \\
$\emeasure$ & Emergence measurement function $M(t)$ \\
$\orderparam$ & Order parameter for classical emergence \\
$\chiDND$ & D-ND coupling constant \\
$\lambdaDND$ & D-ND coupling strength \\
$\Tcog$ & Cognitive timescale \\
$\GS$ & Singularity Constant \\
\hline
\end{tabular}
\end{center}

%==============================================================================
% JOURNAL-SPECIFIC ADAPTATION NOTES
%==============================================================================

% IMPORTANT: The following comments provide guidance for adapting this template
% to the other six target journals. These are NOT part of the compiled PDF.

% ============================================================================
% PAPER A → PHYSICAL REVIEW A (CURRENT)
% ============================================================================
% Document class: revtex4-2
% \documentclass[aps,pra,11pt,notitlepage,nofootinbib]{revtex4-2}
%
% Characteristics:
% - Automatically handles two-column format for physical review journals
% - \author and \affiliation commands are specific to revtex4-2
% - Uses \cite{} and \ref{} natively
% - Abstract must be in \begin{abstract}...\end{abstract} block
% - References via \begin{thebibliography} or \bibliography{file}
%
% Adaptation: NO CHANGES NEEDED - this is the reference implementation

% ============================================================================
% PAPER B → PHYSICAL REVIEW E (SISTER JOURNAL TO A)
% ============================================================================
% Document class: revtex4-2 (same as A)
% \documentclass[aps,pre,11pt,notitlepage,nofootinbib]{revtex4-2}
%
% Changes from Paper A:
% 1. Change option from [aps,pra,...] to [aps,pre,...]
% 2. Rest of document structure remains identical
% 3. Physical Review E focuses on statistical and soft matter physics
% 4. Citation and reference formats are identical to Physical Review A
%
% To adapt: Simply change the documentclass option from 'pra' to 'pre'
% All sections, theorems, and notation remain valid.

% ============================================================================
% PAPER C → JOURNAL OF MATHEMATICAL PHYSICS (AIP)
% ============================================================================
% Document class: aip
% \documentclass[11pt]{aip}
%
% Changes from Paper A:
% 1. \documentclass[11pt]{aip} replaces revtex4-2
% 2. \author{} and \affiliation{} may need adjustment:
%    Replace with:
%    \author{D-ND Research Collective}
%    \address{Department of Theoretical Physics, Foundation Research Institute}
% 3. Use \begin{abstract}...\end{abstract} (same as revtex4-2)
% 4. Bibliography: may need to adjust to aip style (\bibliography{} recommended)
% 5. Section headings and theorem formatting compatible with revtex4-2
%
% Key differences:
% - AIP is single-column format typically
% - Mathematical emphasis is higher
% - May require more explicit theorem numbering
%
% To adapt: Change documentclass, author/affiliation fields, may need to adjust
% \usepackage commands for aip compatibility (some packages may not be needed).

% ============================================================================
% PAPER D → FOUNDATIONS OF PHYSICS (SPRINGER)
% ============================================================================
% Document class: springer (or svjour3)
% \documentclass[twocolumn]{springer}
%   OR
% \documentclass{svjour3}
%
% Changes from Paper A:
% 1. \documentclass[twocolumn]{springer} or \documentclass{svjour3}
% 2. \author{} and \title{} work similarly
%    But use: \author{D-ND Research Collective \and Other Authors}
% 3. \affiliation{} may change format
% 4. Abstract format: may need \begin{abstract}...\end{abstract}
% 5. Section headings compatible with revtex4-2
% 6. Theorem environments may need adjustment:
%    Springer uses slightly different theorem styling
%
% Key differences:
% - Springer uses different citation style (numeric or author-year)
% - May require \usepackage{spbasic} for some formatting
% - Reference format may differ
%
% To adapt: Change documentclass, adjust author/title formatting, verify theorem
% environments display correctly. Citation format may need adjustment.

% ============================================================================
% PAPER E → CLASSICAL AND QUANTUM GRAVITY (IOPART)
% ============================================================================
% Document class: iopart
% \documentclass{iopart}
%
% Changes from Paper A:
% 1. \documentclass{iopart}
% 2. \author{}, \address{}, \ead{} (for email) specific to iopart:
%    \author{D-ND Research Collective}
%    \address{Department of Theoretical Physics, Foundation Research Institute}
%    \ead{contact@institute.org}
% 3. Abstract: \begin{abstract}...\end{abstract} (compatible)
% 4. Keywords: \begin{keywords}...quantum...gravity...\end{keywords}
% 5. Citation format using natbib compatible
% 6. Theorem environments compatible with amsthm
%
% Key differences:
% - IOP is popular for relativity and quantum gravity papers
% - Automatically formats two-column layout
% - Provides specific formatting for references to equations, figures
%
% To adapt: Change documentclass to iopart, adjust metadata fields (author,
% address, email). Sections and theorem formatting remain compatible.

% ============================================================================
% PAPER F → QUANTUM (QUANTUMARTICLE)
% ============================================================================
% Document class: quantumarticle
% \documentclass{quantumarticle}
%
% Changes from Paper A:
% 1. \documentclass{quantumarticle}
% 2. Title and author format:
%    \author{D-ND Research Collective}
%    \affiliation{Department of Theoretical Physics, Foundation Research Institute}
% 3. Abstract: \begin{abstract}...\end{abstract} (compatible)
% 4. Abstract in both English and other languages supported
% 5. Keywords: \begin{keywords}...quantum information...\end{keywords}
% 6. Citation format using hyperref (highly compatible)
%
% Key differences:
% - Quantum journal emphasizes quantum information and computation
% - Modern, open-access journal from Quantum organization
% - Excellent PDF rendering and indexing
% - Uses hyperref extensively
%
% To adapt: Change documentclass to quantumarticle, keep all other formatting.
% This is one of the most compatible transformations.

% ============================================================================
% PAPER G → COGNITIVE SCIENCE / MINDS AND MACHINES (SPRINGER)
% ============================================================================
% Document class: springer (svjour3)
% \documentclass{svjour3}
%
% Changes from Paper A:
% 1. Same as Paper D (Springer document class)
% 2. \documentclass{svjour3}
% 3. Author and affiliation formatting per Springer standards
% 4. Abstract and section formatting compatible
% 5. Citation style per journal preference
%
% Key differences:
% - Cognitive Science and Minds and Machines emphasizes interdisciplinary work
% - May require different section organization
% - Might benefit from separate sections for cognitive implications
% - Philosophy of mind sections may be reordered
%
% To adapt: Use Springer document class (svjour3), adjust author/affiliation
% formatting. All mathematical and theorem sections remain valid.

%==============================================================================
% END DOCUMENT
%==============================================================================

\end{document}
