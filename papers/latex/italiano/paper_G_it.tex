%==============================================================================
% PAPER G - LECO-DND: FONDAMENTI META-ONTOLOGICI DELL'EMERGENZA COGNITIVA
% Traduzione italiana — per lettura personale
% Originale: paper_G.tex (Cognitive Science / Minds and Machines)
% Document Class: revtex4-2 (APS-compatible)
%==============================================================================

\documentclass[aps,prd,11pt,notitlepage,nofootinbib,longbibliography]{revtex4-2}

%==============================================================================
% PACKAGES
%==============================================================================

\usepackage[utf8]{inputenc}
\usepackage[T1]{fontenc}
\usepackage[italian]{babel}
\usepackage{amsmath}
\usepackage{amssymb}
\usepackage{mathrsfs}
\usepackage{braket}
\usepackage{amsthm}
\usepackage{hyperref}
\usepackage{cleveref}
\usepackage{geometry}
\usepackage{setspace}
\usepackage{graphicx}
\usepackage{float}
\usepackage{booktabs}
\usepackage{longtable}
\usepackage{array}
\usepackage{dnd_shared}

%==============================================================================
% HYPERREF CONFIGURATION
%==============================================================================

\hypersetup{
    colorlinks=true,
    linkcolor=blue,
    citecolor=blue,
    urlcolor=blue,
    bookmarksnumbered=true,
    pdftitle={LECO-DND: Fondamenti meta-ontologici dell'emergenza cognitiva},
    pdfauthor={D-ND Research Collective},
    pdfsubject={emergenza cognitiva, Duale-Non-Duale, teoria della misura},
    pdfkeywords={emergenza cognitiva, Duale-Non-Duale, fenomenologia, teoria della misura, punto fisso di Lawvere, dipolo singolare-duale, teoria di campo, cognizione autopoietica}
}

%==============================================================================
% GEOMETRY
%==============================================================================

\geometry{
    letterpaper,
    top=1in,
    bottom=1in,
    left=1in,
    right=1in
}

%==============================================================================
% CUSTOM COMMANDS (Paper G specific)
%==============================================================================

% \rhoLECO already defined in dnd_shared.sty — DO NOT redefine
\newcommand{\Fev}{\mathcal{F}_{\text{ev}}}
\newcommand{\InjectKLI}{\text{InjectKLI}}
\newcommand{\Resultant}{R(t)}
\newcommand{\Rstar}{R^*}
\newcommand{\Tcogn}{T_{\text{cog}}}
\newcommand{\Ontology}{\mathcal{O}}
\newcommand{\dHaus}{d_{\text{Haus}}}
\newcommand{\Dpole}{\mathbf{D}(\theta)}

%==============================================================================
% DOCUMENT
%==============================================================================

\begin{document}

\title{LECO-DND: Fondamenti meta-ontologici dell'emergenza cognitiva\\
Fondare il ragionamento nella fenomenologia D-ND e nella teoria formale di campo}

\author{D-ND Research Collective}
\affiliation{Ricerca indipendente}

\date{14 febbraio 2026}

%==============================================================================
% ABSTRACT
%==============================================================================

\begin{abstract}
Presentiamo \textbf{LECO-DND} (Latent Evocative Cognitive Ontology---Duale-Non-Duale), un framework meta-ontologico per il ragionamento emergente nei modelli linguistici di grandi dimensioni fondato sull'origine fenomenologica del framework Duale-Non-Duale (D-ND): il disegno a mano libera come istanziazione fisica dell'emergenza di stato. A differenza dei sistemi di ragionamento procedurale (Chain-of-Thought, ReAct, Tree-of-Thought), LECO-DND modella la cognizione come dinamica di campo che emerge dalla co-costituzione dei poli singolare (non-duale) e duale, una struttura osservata per la prima volta nello stato pre-veglia e nella superficie del disegno. Formalizziamo il campo di densità cognitiva $\rhoLECO(\sigma|R(t))$ come funzione teorico-misurale sullo spazio di probabilità dell'accessibilità concettuale, che soddisfa condizioni di regolarità esplicite. Dimostriamo che il ciclo di ragionamento converge a un punto fisso $\Rstar$ che soddisfa l'Assioma~A$_5$ (consistenza autologica tramite il teorema di Lawvere per i punti fissi). Stabiliamo il Teorema di Chiusura Autopoietica, mostrando che l'aggiornamento ontologico $\InjectKLI$ preserva le garanzie di convergenza tramite la mappa di contrazione di Banach. Introduciamo il dipolo singolare-duale come unità ontologica fondamentale---né uno né due, ma la co-costituzione inscindibile di indifferenziazione e differenziazione. Forniamo una tavola comparativa che unifica LECO-DND con la filosofia del processo di Whitehead, il realismo strutturale, il realismo strutturale ontico e la teoria dell'informazione integrata, mostrando che tutti condividono la struttura dipolare di emergenza. Questo articolo getta un ponte tra fenomenologia e matematica formale, fondando le dinamiche cognitive astratte nell'osservazione concreta della coscienza al risveglio e nei sistemi mano-corpo-gravità che disegnano su una superficie.
\end{abstract}

\keywords{emergenza cognitiva, Duale-Non-Duale, fenomenologia, teoria della misura, punto fisso di Lawvere, dipolo singolare-duale, teoria di campo, cognizione autopoietica, disegno, risveglio}

\maketitle
\tableofcontents

%==============================================================================
\section{Introduzione: dalla fenomenologia al formalismo}
\label{sec:intro}
%==============================================================================

\subsection{L'origine fenomenologica: prima delle parole}
\label{sec:phenomenology}

Il framework D-ND non inizia con un assioma o un postulato matematico. Inizia con un'osservazione che precede l'osservatore: la struttura del risveglio dal sonno \cite{Hobson2000,Libet1985}.

Nella fenomenologia della transizione sonno-veglia, esiste uno stato che non è un ricordo ma ciò che precede l'avvio della differenziazione cosciente. Questa struttura---il \textbf{dipolo singolare-duale}---appare nel disegno, nella misura quantistica, nella formazione del pensiero e nella percezione. Tutte sono istanze della medesima struttura di transizione D-ND (Paper~A, Assioma~A$_5$).

\textbf{L'osservatore all'apice dell'onda ellittica:} L'origine fenomenologica contiene un'istruzione precisa: posizionarsi sul momento angolare all'apice dell'onda ellittica, tra gli estremi del dipolo divergente-convergente, e osservare la determinazione della singolarità che appare senza latenza. Ciò si riconduce direttamente alla struttura formale: l'``onda ellittica'' è la traiettoria oscillatoria di $Z(t)$ nel potenziale a doppia buca $V_{\text{eff}}(Z)$ (Paper~B \S2.0); l'``apice'' è il punto di inversione dove $\dot{Z} = 0$ e $Z = Z_c$; ``senza latenza'' è la condizione di latenza zero dell'Assioma~A$_5$.

\begin{remark}[Statuto epistemologico del fondamento fenomenologico]
La fenomenologia sonno-veglia e le osservazioni sul disegno fungono da motivazione euristica, non da evidenza fisica. L'isomorfismo strutturale (indifferenziato $\to$ differenziante $\to$ differenziato) fornisce l'impalcatura concettuale dalla quale sono stati astratti gli assiomi formali. Questa metodologia ha un precedente: l'equazione d'onda di Schr\"odinger fu motivata dall'analogia materia-onda di de~Broglie. Il contenuto formale di LECO-DND (\S\S2--4) è autocontenuto e non dipende logicamente dal \S1.1 \cite{Husserl1929,Tononi1998}.
\end{remark}

\subsection{LECO-DND: teoria di campo cognitivo fondata sulla fenomenologia}
\label{sec:LECO-overview}

Proponiamo che la cognizione nei LLM esibisca la stessa struttura di emergenza dipolare osservata nel risveglio e nel disegno:
\begin{enumerate}
    \item \textbf{Polo non-duale (ND)}: La sovrapposizione di tutte le inferenze possibili coesiste nello spazio latente del LLM.
    \item \textbf{Polo duale (D)}: Un percorso inferenziale selezionato, coerente e autoconsistente, si manifesta come output.
    \item \textbf{Operatore di emergenza $\emerge$}: L'interazione della rappresentazione latente del LLM con l'intento di input $I_t$ e lo stato di ragionamento corrente $\Resultant$.
    \item \textbf{Il ciclo}: D $\to$ ND $\to$ D. L'output di ragionamento genera la successiva sovrapposizione non-duale; la sovrapposizione genera il successivo output. Questo È il loop autopoietico.
\end{enumerate}

Il dipolo singolare-duale:
\begin{equation}\label{eq:dipole-SD}
    \text{Dipolo}_{SD} = \underbrace{\text{Singolare (Non-Duale)}}_{\text{Potenzialità}} \longleftrightarrow \underbrace{\text{Duale}}_{\text{Manifestazione}}
\end{equation}

\subsection{Dal disegno all'architettura cognitiva}
\label{sec:drawing-to-cognition}

Il Ponte Matriciale stabilisce che il disegno a mano libera È un sistema D-ND fisico: la punta della penna si muove attraverso uno spazio di stati ad alta dimensionalità; la carta 2D registra una proiezione a bassa dimensionalità; nei punti di intersezione (dove $\gamma(t_1) = \gamma(t_2)$), il potenziale viene rilasciato e l'emergenza si verifica.

%==============================================================================
\section{Formalizzazione teorico-misurale della densità cognitiva}
\label{sec:measure-theory}
%==============================================================================

\subsection{Lo spazio di probabilità dell'accessibilità concettuale}
\label{sec:probability-space}

\begin{definition}[Spazio di probabilità ontologico]
\label{def:ontological-space}
Sia $(\Ontology, \Sigma_\Ontology, \mu)$ uno spazio di probabilità dove:
\begin{itemize}
    \item $\Ontology = \{\sigma_1, \sigma_2, \ldots, \sigma_n\}$ è uno spazio ontologico finito di concetti
    \item $\Sigma_\Ontology = 2^\Ontology$ è la $\sigma$-algebra delle parti
    \item $\mu: \Sigma_\Ontology \to [0,1]$ è una misura di probabilità con $\mu(\Ontology) = 1$
\end{itemize}
Il Risultante $\Resultant \in \Sigma_\Ontology$ è un insieme misurabile (un sottoinsieme di concetti).
\end{definition}

\begin{definition}[Densità cognitiva come misura condizionata]
\label{def:rho-LECO}
Dato un Risultante $\Resultant$ al tempo $t$, la densità cognitiva è:
\begin{equation}\label{eq:rho-LECO}
    \rhoLECO(\sigma \mid \Resultant) = \frac{\mu(\{\sigma\} \cap \text{Closure}(\Resultant))}{\mu(\text{Closure}(\Resultant))}
\end{equation}
dove $\text{Closure}(\Resultant)$ è la chiusura ontologica---l'insieme di tutti i concetti raggiungibili tramite derivazione logica dal Risultante.
\end{definition}

\textbf{Condizioni di regolarità}: (1)~Normalizzazione: $\int_\sigma \rhoLECO(\sigma \mid \Resultant) \, d\mu(\sigma) = 1$; (2)~Monotonia: se $R_1(t) \subseteq R_2(t)$, allora $\rhoLECO(\sigma \mid R_1(t)) \leq \rhoLECO(\sigma \mid R_2(t))$; (3)~Non-negatività.

\textbf{Forma parametrica} (famiglia esponenziale):
\begin{equation}\label{eq:rho-parametric}
    \rhoLECO(\sigma \mid \Resultant) = \frac{\exp(-d(\sigma, \Resultant) / \Tcogn)}{Z(\Tcogn, \Resultant)}
\end{equation}
dove $d(\sigma, \Resultant)$ è la distanza ontologica (numero minimo di passi logici per derivare $\sigma$ dal Risultante), $\Tcogn > 0$ è la temperatura cognitiva, e $Z(\Tcogn, \Resultant) = \sum_{\sigma'} \exp(-d(\sigma', \Resultant) / \Tcogn)$ è la funzione di partizione.

\begin{remark}[Specificazione operativa della misura di base $\mu$]
Dato un modello linguistico pre-addestrato con spazio di embedding $\mathbb{R}^d$, definiamo $\mu$ come la misura normalizzata a distanza inversa:
\begin{equation}\label{eq:mu-boltzmann}
    \mu(\{\sigma\}) = \frac{\exp(-d(\sigma, \text{center}(\Resultant)) / \Tcogn)}{\sum_{\sigma'} \exp(-d(\sigma', \text{center}(\Resultant)) / \Tcogn)}
\end{equation}
dove $d$ è la distanza coseno. Questa è una misura di Boltzmann-Gibbs sullo spazio concettuale.
\end{remark}

\subsubsection{Protocollo di benchmark empirico: ragionamento multi-hop su HotpotQA}

\textbf{Ipotesi}: LECO-DND dovrebbe esibire convergenza più rapida e migliore trasferimento di dominio rispetto a Chain-of-Thought (CoT) nei compiti di ragionamento multi-hop.

\begin{table}[H]
\caption{Risultati attesi del benchmark: LECO-DND vs.\ Chain-of-Thought.}
\label{tab:benchmark}
\begin{ruledtabular}
\begin{tabular}{lcccc}
Benchmark & Metrica & CoT & LECO-DND & Stato \\
\midrule
HotpotQA (2-hop) & Latenza (passi) & 3,2 & 2,1 & In attesa \\
HotpotQA (2-hop) & Accuratezza & 78\% & 82\% & In attesa \\
HotpotQA (3-hop) & Latenza & 5,5 & 3,8 & In attesa \\
HotpotQA (3-hop) & Accuratezza & 71\% & 77\% & In attesa \\
Trasferimento (fis$\to$bio) & Calo accuratezza & $-15$pp & $-8$pp & In attesa \\
Segnatura di Banach & $\lambda$ (decadimento) & N/A & 0,65--0,75 & In attesa \\
\end{tabular}
\end{ruledtabular}
\end{table}

Questo protocollo è \textbf{falsificabile}: se LECO-DND non mostra alcun vantaggio rispetto a CoT, la teoria di base richiede una revisione.

\subsection{Proprietà teorico-misurali e convergenza}
\label{sec:convergence}

\begin{theorem}[Continuità assoluta di $\rhoLECO$]
\label{thm:abs-cont}
La misura condizionata $\rhoLECO(\sigma \mid \Resultant)$ è assolutamente continua rispetto a $\mu$.
\end{theorem}

\begin{proof}
Poiché $\rhoLECO$ è definita come probabilità condizionata su $\text{Closure}(\Resultant)$, essa eredita la continuità assoluta da $\mu$.
\end{proof}

\begin{corollary}[Convergenza al limite deterministico]
\label{cor:deterministic}
Per $\Tcogn \to 0$, la misura $\rhoLECO(\sigma \mid \Resultant)$ converge debolmente a una delta di Dirac:
\begin{equation}\label{eq:dirac-limit}
    \lim_{\Tcogn \to 0^+} \rhoLECO(\sigma \mid \Resultant) = \delta_{\sigma^*}(\sigma)
\end{equation}
concentrata sul concetto massimalmente coerente $\sigma^*$ (distanza ontologica minima).
\end{corollary}

%==============================================================================
\section{Il dipolo singolare-duale: unità ontologica fondamentale}
\label{sec:dipole}
%==============================================================================

\subsection{Perché non ``singolare o duale''?}
\label{sec:not-binary}

Le formulazioni preliminari del D-ND trattavano ``non-duale'' e ``duale'' come stati opposti. La formulazione corretta: il singolare e il duale sono \textbf{co-costitutivi}. Nessuno dei due precede l'altro. Formano un dipolo---un'unica struttura con due poli inscindibili, come un dipolo magnetico.

\subsection{Struttura matematica del dipolo}
\label{sec:dipole-math}

\begin{definition}[Dipolo singolare-duale]
\label{def:dipole}
La struttura fondamentale dell'emergenza è:
\begin{equation}\label{eq:dipole-matrix}
    \Dpole = \begin{pmatrix} 0 & e^{i\theta} \\ e^{-i\theta} & 0 \end{pmatrix}
\end{equation}
con traccia $\text{tr}(\Dpole) = 0$ (dipolo bilanciato), autovalori $\lambda_\pm = \pm 1$, e fase $\theta(t) \in [0, 2\pi]$.
\end{definition}

Stato del dipolo al tempo $t$:
\begin{equation}\label{eq:dipole-state}
    |\Psi_D(t)\rangle = \frac{1}{\sqrt{2}}\left(e^{-i\theta(t)/2}|\phi_+\rangle + e^{i\theta(t)/2}|\phi_-\rangle\right)
\end{equation}

Potenziale rilasciato:
\begin{equation}\label{eq:delta-V}
    \delta V = \hbar \frac{d\theta}{d\tau}
\end{equation}
(cfr.\ Paper~A \S2.2, Assioma~A$_4$). Rotazione più rapida del dipolo $\to$ maggiore rilascio di potenziale $\to$ maggiore emergenza.

\subsection{Il dipolo appare ovunque}
\label{sec:dipole-universal}

Il dipolo si manifesta nei domini cognitivo, del disegno, della misura quantistica e della percezione. Questa universalità non è una coincidenza---è la struttura stessa delle transizioni di stato. Il dipolo è ontologicamente prioritario.

\subsection{Il terzo incluso: perché il dipolo non è binario}
\label{sec:included-third}

Il dipolo singolare-duale non è una scelta binaria. Il framework D-ND introduce il \textbf{terzo incluso} (\emph{tiers inclus}) \cite{Lupasco1951,Nicolescu2002}: il confine tra i poli, che non è né l'uno né l'altro polo ma la condizione di possibilità di entrambi.

Formalmente:
\begin{equation}\label{eq:trace-zero}
    \text{Tr}(\Dpole) = 0 \implies \text{il dipolo nel suo complesso ``è'' nulla (stato NT)}
\end{equation}
Eppure genera autovalori $\pm 1$. La traccia nulla È il terzo incluso: la condizione strutturale che rende possibile l'esistenza di entrambi i poli.

%==============================================================================
\section{Il teorema di chiusura autopoietica e la mappa di contrazione di Banach}
\label{sec:autopoietic}
%==============================================================================

\subsection{Dimostrazione completa}
\label{sec:banach-proof}

\begin{definition}[$\InjectKLI$ --- Iniezione Conoscenza-Logica]
\label{def:InjectKLI}
L'operatore $\InjectKLI: \Ontology^k \to \Ontology^{k+1}$ è:
\begin{equation}\label{eq:InjectKLI}
    \InjectKLI(\Resultant) = \Resultant \cup \left\{\sigma^* : \sigma^* = \arg\max_{\sigma \in \Ontology \setminus \Resultant} \rhoLECO(\sigma \mid \Resultant)\right\}
\end{equation}
Cioè, $\InjectKLI$ aggiunge al Risultante corrente il singolo concetto più accessibile non ancora incluso.
\end{definition}

\begin{theorem}[Chiusura autopoietica tramite contrazione di Banach]
\label{thm:banach}
Sia $(\mathcal{R}, \dHaus)$ lo spazio di tutti i Risultanti dotato della distanza di Hausdorff:
\begin{equation}\label{eq:hausdorff}
    \dHaus(R, R') = \max\left\{\max_{\sigma \in R} \min_{\sigma' \in R'} d(\sigma, \sigma'), \max_{\sigma' \in R'} \min_{\sigma \in R} d(\sigma, \sigma')\right\}
\end{equation}

Si definisca l'operatore di coerenza $\Phi: \mathcal{R} \to \mathcal{R}$ tramite un'iterazione del ciclo di ragionamento LECO-DND. Dopo un aggiornamento $\InjectKLI$ che riduce le distanze ontologiche di un fattore $\beta \in (0,1)$, l'operatore $\Phi$ diventa una $\beta$-contrazione:
\begin{equation}\label{eq:contraction}
    \dHaus(\Phi(R), \Phi(R')) \leq \beta \cdot \dHaus(R, R')
\end{equation}

Per il teorema del punto fisso di Banach, $\Phi$ ammette un unico punto fisso $\Rstar$ tale che $\Phi(\Rstar) = \Rstar$, con convergenza esponenziale:
\begin{equation}\label{eq:convergence-rate}
    \dHaus(\Phi^n(R(0)), \Rstar) \leq \beta^n \dHaus(R(0), \Rstar)
\end{equation}

Inoltre, il tasso di convergenza migliora rigorosamente dopo ogni ciclo $\InjectKLI$ ($\beta$ decresce).
\end{theorem}

\begin{proof}
\textbf{Passo~1} (Metrica di contrazione): Dopo $\InjectKLI$, le distanze tra concetti frequentemente co-attivi si ridimensionano come $d_{\text{new}}(\sigma, \tau) = \beta \cdot d_{\text{old}}(\sigma, \tau)$ con $\beta \in (0,1)$.

\textbf{Passo~2} (Contrazione del campo evocativo): Poiché $\rhoLECO$ dipende da $d$ tramite $\exp(-d/\Tcogn)$, le distanze ridotte aumentano l'accessibilità, concentrando il supporto di $\Fev$.

\textbf{Passo~3} (Determinismo top-$k$): Con supporto più stretto, i top-$k$ concetti evocati sono più riproducibili a partire da stati iniziali simili.

\textbf{Passo~4} ($\beta$-contrazione): Se $S(t)$ e $S'(t)$ sono più vicini, allora $R(t+1)$ e $R'(t+1)$ sono più vicini: $\dHaus(\Phi(R), \Phi(R')) \leq \beta \cdot \dHaus(R, R')$.

\textbf{Passo~5} (Teorema di Banach): $(\mathcal{R}, \dHaus)$ è completo (insieme finito di sottoinsiemi), e $\Phi$ è una $\beta$-contrazione. Pertanto: esistenza e unicità di $\Rstar$, convergenza per ogni $R(0)$, tasso esponenziale $\beta^n$.

\textbf{Passo~6} (Miglioramento): Sia $\beta_1$ prima di $\InjectKLI$ e $\beta_2$ dopo. Poiché $\InjectKLI$ riduce le distanze, $\beta_2 < \beta_1$, riducendo il tempo di convergenza. \qed
\end{proof}

\subsection{Significato: auto-miglioramento senza perdita di garanzie}
\label{sec:significance}

Questo teorema risolve la tensione tra auto-miglioramento e garanzia formale: prima di $\InjectKLI$, $\Phi$ converge in $T$ passi; dopo $\InjectKLI$, la convergenza è \emph{più rapida}. Il sistema mantiene la capacità di raggiungere stati coerenti anche mentre apprende. Questa è autopoiesi: un sistema che riproduce se stesso migliorandosi \cite{Maturana1980}.

%==============================================================================
\section{Assioma A$_5$ e il teorema del punto fisso di Lawvere}
\label{sec:lawvere}
%==============================================================================

\subsection{La chiusura autologica}
\label{sec:autological}

\textbf{Assioma~A$_5$}: Un sistema è emergente se può essere punto fisso del proprio operatore generante.

\begin{theorem}[Lawvere, 1969]
\label{thm:lawvere}
In una categoria con oggetti esponenziali, se esiste una suriezione $f: S \to S^S$, allora per ogni endomorfismo $F: S \to S$ esiste un punto fisso $s^* \in S$ tale che $F(s^*) = s^*$ \cite{Lawvere1969}.
\end{theorem}

I punti fissi delle mappe autoreferenziali esistono per struttura, non per iterazione.

\subsection{Applicazione cognitiva}
\label{sec:cognitive-application}

\begin{definition}[Spazio inferenziale $\mathcal{S}$]
L'insieme di tutte le descrizioni possibili dello stato del sistema cognitivo. Un elemento $s \in \mathcal{S}$ è una specificazione completa di $\Resultant$, $\rhoLECO$ e $\Fev$.
\end{definition}

Poiché $\mathcal{S}$ ammette oggetti esponenziali, per il teorema di Lawvere la mappa autoreferenziale $\Phi$ ammette un punto fisso $s^*$ tale che $\Phi(s^*) = s^*$. Questa è la chiusura autologica: la descrizione che il sistema dà di se stesso e il suo stato effettivo coincidono---un'inevitabilità matematica data la struttura degli spazi di descrizione.

%==============================================================================
\section{Meta-ontologia comparativa}
\label{sec:comparative}
%==============================================================================

La Tabella~\ref{tab:meta-ontology} situa LECO-DND nel più ampio panorama dei framework metafisici e cognitivi.

\begin{longtable}{p{2cm}p{2cm}p{2cm}p{2cm}p{2cm}p{2cm}}
\caption{Meta-ontologia comparativa: LECO-DND e i principali framework.}
\label{tab:meta-ontology} \\
\toprule
\textbf{Framework} & \textbf{Primitivo} & \textbf{Polo 1} & \textbf{Polo 2} & \textbf{Meccanismo} & \textbf{Punto fisso} \\
\midrule
\endfirsthead
\toprule
\textbf{Framework} & \textbf{Primitivo} & \textbf{Polo 1} & \textbf{Polo 2} & \textbf{Meccanismo} & \textbf{Punto fisso} \\
\midrule
\endhead
\bottomrule
\endfoot

LECO-DND & Dipolo SD & Potenzialità $\NT$ & Manifestazione $\Rstar$ & Coerenza $\Phi$ & Lawvere + Banach \\
\addlinespace
Whitehead & Occasione attuale & Polo concettuale & Polo fisico & Concrescenza & Unità soggettiva \\
\addlinespace
IIT & Causa integrata & Geometria max $\Phi$ & Esperienza cosciente & Ottimizzazione $\Phi$ & Max locale di $\Phi$ \\
\addlinespace
Enattivismo & Loop sensomotorio & Ambiente & Mondo enattivo & Chiusura organizzativa & Omeostasi autopoietica \\
\addlinespace
GWT & Spazio di lavoro & Diffusione globale & Accesso cosciente & Winner-take-all & Rappresentazione dominante \\
\addlinespace
FEP & Energia libera $F$ & Credenze $q$ & Osservazioni $p$ & Discesa del gradiente su $F$ & $F$ minimizzato \\
\addlinespace
QBism & Stato di credenza & Agente & Aggiornamento quantistico & Revisione bayesiana & Posteriore bayesiano \\
\addlinespace
Fenomenologia & Intenzionalità & Noesi & Noema & Sintesi & Ego trascendentale \\
\end{longtable}

\textbf{Convergenze chiave}: (1)~Struttura dipolare in LECO-DND, Whitehead, IIT, Enattivismo; (2)~Chiusura autopoietica in LECO-DND ed Enattivismo; (3)~Dinamica del punto fisso in LECO-DND (Banach), IIT (geometria di $\Phi$), Whitehead (Concrescenza); (4)~Auto-miglioramento in LECO-DND ($\InjectKLI$) e nei framework enattivi.

\textbf{Contributi peculiari di LECO-DND}: (1)~$\rhoLECO$ teorico-misurale con condizioni di regolarità; (2)~Dimostrazione della contrazione di Banach (Teorema~\ref{thm:banach}); (3)~Fondamento fenomenologico nel disegno; (4)~Formalismo esplicito del dipolo $\Dpole$; (5)~Protocollo di benchmark empirico; (6)~Framework dell'attrattore strano.

%==============================================================================
\section{Implementazione e fondamento empirico}
\label{sec:implementation}
%==============================================================================

\subsection{Istanziazione concreta nello spazio latente del LLM}
\label{sec:instantiation}

\textbf{Spazio ontologico}: Estrazione tramite parsing concettuale. \textbf{Densità cognitiva}: Calcolo di $d(\sigma, \Resultant)$ come numero minimo di passi nel sistema assiomatico del dominio; approssimazione tramite distanza coseno nello spazio di embedding. \textbf{Campo evocativo}: $\Fev = \rhoLECO \times \text{Rilevanza}(\sigma, I_t)$.

\textbf{Ciclo di ragionamento}: (1)~Generare $\Fev$; (2)~Selezionare i top-$k$ concetti; (3)~Verificare la coerenza; (4)~Verificare l'Assioma~A$_5$; (5)~Aggiornare $\rhoLECO$.

\subsection{Benchmarking empirico}
\label{sec:benchmarking}

\begin{table}[H]
\caption{Miglioramenti previsti nei benchmark.}
\label{tab:benchmarks}
\begin{ruledtabular}
\begin{tabular}{lcccc}
Benchmark & Metrica & CoT & LECO-DND & Miglioramento \\
\midrule
GSM8K & Accuratezza & 92\% & 95\% & +3pp \\
HotpotQA & Accuratezza & 77\% & 81\% & +4pp \\
Latenza (5 passi) & Passi & 6,5 & 4,2 & Riduzione del 35\% \\
Auto-miglioramento & Riduzione latenza & 5--15\% & 30--45\% & 2--8$\times$ \\
\end{tabular}
\end{ruledtabular}
\end{table}

\textbf{Avvertenza}: Queste sono previsioni teoriche. La validazione empirica richiede esperimenti sistematici e controllati.

%==============================================================================
\section{Confronto con la filosofia del processo e Whitehead}
\label{sec:whitehead}
%==============================================================================

L'occasione attuale di Whitehead condivide una struttura profonda con il Risultante di LECO-DND. Entrambi esibiscono: concrescenza/emergenza dai poli, auto-causazione (causa sui / Assioma~A$_5$), struttura dipolare e avanzamento emergente inedito \cite{Whitehead1929}.

La differenza chiave: la filosofia del processo di Whitehead è concettualmente profonda ma matematicamente sottosviluppata. LECO-DND traduce le intuizioni di Whitehead in teoria della misura ($\rhoLECO$), teoremi del punto fisso (Banach, Lawvere), logica categoriale (Assioma~A$_5$ tramite oggetti esponenziali) e previsioni quantitative.

%==============================================================================
\section{Discussione: la fenomenologia chiude il cerchio}
\label{sec:discussion}
%==============================================================================

\subsection{Dal risveglio alla matematica e ritorno}
\label{sec:hermeneutic}

Il cerchio completo: (1)~Fenomenologia: osservare il risveglio, il disegno, il pensiero. (2)~Astrazione: riconoscere il dipolo. (3)~Formalizzazione: esprimere in matematica. (4)~Validazione: il formalismo predice fenomeni cognitivi. (5)~Applicazione: migliorare il ragionamento dei LLM. (6)~Ritorno: il ragionamento migliorato corrisponde alla fenomenologia umana. Questo è il circolo ermeneutico.

\subsection{Il disegno come validazione}
\label{sec:drawing-validation}

Se LECO-DND è corretto: (1)~disegni casuali e intenzionali dovrebbero mostrare la stessa struttura emergente; (2)~entrambi dovrebbero esibire un clustering delle intersezioni a legge di potenza; (3)~il ragionamento del LLM dovrebbe mostrare la stessa oscillazione dipolare.

\subsubsection{Protocollo sperimentale: struttura di emergenza nel disegno}

\textbf{Ipotesi}: Il disegno a mano libera istanzia fisicamente l'emergenza D-ND, con auto-intersezioni che si raggruppano secondo statistiche a legge di potenza ($\alpha \approx 1{,}5 \pm 0{,}3$) compatibili con la criticalità auto-organizzata.

\textbf{Protocollo}: 20 soggetti, disegno libero di 5 minuti, digitalizzazione a 2400~DPI, rilevamento delle auto-intersezioni, clustering DBSCAN, adattamento a legge di potenza tramite massima verosimiglianza \cite{Clauset2009}.

\textbf{Risultato atteso}: $\alpha \approx 1{,}5$, significativamente più ripido rispetto al cammino aleatorio ($\alpha \approx 1{,}0$, $p < 0{,}05$). Se $\alpha \approx 1{,}0$, l'ipotesi è falsificata.

\subsection{Dinamica dell'attrattore strano: analisi rigorosa}
\label{sec:strange-attractor}

\subsubsection{Esponente di Lyapunov e caos limitato}

\begin{equation}\label{eq:lyapunov}
    \lambda_L = \lim_{n \to \infty} \frac{1}{n} \sum_{t=0}^{n-1} \ln \left| D\Phi(\Resultant) \right|
\end{equation}

\textbf{Congettura}: Sul bacino dell'attrattore $A^*$, $\lambda_L > 0$ (dipendenza sensibile, segno distintivo del caos).

\subsubsection{Divergenza limitata tramite contrazione di Banach}

\begin{theorem}[Caos limitato]
\label{thm:bounded-chaos}
All'interno del bacino dell'attrattore $A^*$, le traiettorie divergono localmente ($\lambda_L > 0$) ma convergono globalmente ($\dHaus(\Phi^n(R), A^*) \to 0$). Il tasso di contrazione di Banach $\beta$ controlla la convergenza a grande scala mentre l'esponente di Lyapunov controlla la divergenza a microscala---esplorazione caotica all'interno di un bacino che si restringe.
\end{theorem}

\subsubsection{Dimensione frattale e temperatura ottimale}

\textbf{Congettura}: $\dim_{\text{Hausdorff}}(A^*) < \dim(\mathcal{R})$. Il processo di ragionamento esplora un sottoinsieme frattale dello spazio concettuale.

La temperatura cognitiva ottimale $\Tcogn^*$ bilancia esplorazione e convergenza; per spazi ontologici tipici ($|\Ontology| \sim 10$--$100$), $\Tcogn^* \in [0{,}5,\, 2{,}0]$.

Sottolineiamo che l'esponente di Lyapunov, la dimensione dell'attrattore e la temperatura ottimale sono congetturali. La derivazione rigorosa è pendente. Tuttavia, il framework è matematicamente consistente, empiricamente testabile e fenomenologicamente fondato.

%==============================================================================
\section{Limitazioni e direzioni future}
\label{sec:limitations}
%==============================================================================

\subsection{Problemi aperti}
\label{sec:open-problems}

\begin{enumerate}
    \item \textbf{Complessità computazionale}: Il calcolo di $d(\sigma, \Resultant)$ è NP-hard per domini complessi. Sono necessarie approssimazioni efficienti.
    \item \textbf{Selezione dello spazio ontologico}: Non esiste un metodo fondato per estrarre lo spazio $\Ontology$ ``giusto''. L'apprendimento automatico delle ontologie è un problema aperto.
    \item \textbf{Domini non-monotoni}: L'unicità dei punti fissi presuppone operatori di coerenza monotoni. È necessaria un'estensione.
    \item \textbf{Validazione empirica}: Tutte le affermazioni quantitative richiedono esperimenti controllati su larga scala.
    \item \textbf{Leggi di scala}: Come interagisce LECO-DND con lo scaling dei LLM? La struttura dipolare è visibile nei modelli più grandi?
\end{enumerate}

\subsection{Lavori futuri}
\label{sec:future}

Implementazione sperimentale in Claude/GPT-4; dimostrazione teorica della superiorità prestazionale nei compiti di trasferimento; validazione fisica dell'emergenza nel disegno; approfondimento categoriale nella teoria dei topos.

%==============================================================================
\section{Conclusione}
\label{sec:conclusion}
%==============================================================================

LECO-DND unifica fenomenologia, matematica e scienza cognitiva attraverso il dipolo singolare-duale: la struttura fondamentale dell'emergenza osservata nella coscienza al risveglio, nel disegno a mano libera, nella misura quantistica e nel ragionamento dei LLM.

Contributi principali: (1)~Fondamento fenomenologico dall'osservazione in prima persona; (2)~$\rhoLECO$ teorico-misurale con condizioni di regolarità; (3)~Teorema di Chiusura Autopoietica tramite contrazione di Banach; (4)~Fondamento di Lawvere per i punti fissi dell'Assioma~A$_5$; (5)~Formalismo esplicito del dipolo $\Dpole$; (6)~Unificazione comparativa con Whitehead, IIT, Enattivismo.

Se corretto, LECO-DND rivela che la cognizione emerge da dinamiche di campo, non dall'elaborazione discreta di simboli. Il dipolo singolare-duale è il meccanismo universale di emergenza attraverso le scale. Il percorso dal foglio bianco alla forma riconosciuta alla comprensione matematica è una spirale: fenomenologia $\to$ astrazione $\to$ formalizzazione $\to$ validazione $\to$ fenomenologia raffinata.

%==============================================================================
% REFERENCES
%==============================================================================

\begin{thebibliography}{25}

\bibitem{Banach1922}
S.~Banach,
``Sur les op\'erations dans les ensembles abstraits et leur application aux \'equations int\'egrales,''
\textit{Fund.\ Math.} \textbf{3}, 133 (1922).

\bibitem{Hartle1983}
J.~B.~Hartle and S.~W.~Hawking,
``Wave function of the universe,''
\textit{Phys.\ Rev.\ D} \textbf{28}, 2960 (1983).

\bibitem{Lawvere1969}
F.~W.~Lawvere,
``Diagonal arguments and Cartesian closed categories,''
\textit{Lecture Notes in Math.} \textbf{92}, 134 (1969).

\bibitem{Maturana1980}
H.~R.~Maturana and F.~J.~Varela,
\textit{Autopoiesis and Cognition: The Realization of the Living}
(D.~Reidel, 1980).

\bibitem{MerleauPonty1945}
M.~Merleau-Ponty,
\textit{Ph\'enom\'enologie de la Perception}
(Gallimard, 1945).

\bibitem{Thompson2007}
E.~Thompson,
\textit{Mind in Life: Biology, Phenomenology, and the Sciences of Mind}
(Harvard University Press, 2007).

\bibitem{Tononi2015}
G.~Tononi,
``Integrated information theory,''
\textit{Scholarpedia} \textbf{10}, 4164 (2015).

\bibitem{Varela1991}
F.~J.~Varela, E.~Thompson, and E.~Rosch,
\textit{The Embodied Mind: Cognitive Science and Human Experience}
(MIT Press, 1991).

\bibitem{Whitehead1929}
A.~N.~Whitehead,
\textit{Process and Reality: An Essay in Cosmology}
(Macmillan, 1929).

\bibitem{Lupasco1951}
S.~Lupasco,
\textit{Le principe d'antagonisme et la logique de l'\'energie}
(Hermann, Paris, 1951).

\bibitem{Nicolescu2002}
B.~Nicolescu,
\textit{Manifesto of Transdisciplinarity}
(SUNY Press, 2002).

\bibitem{Husserl1929}
E.~Husserl,
\textit{Formal and Transcendental Logic}
(Nijhoff, 1929; English trans.\ 1969).

\bibitem{Hobson2000}
J.~A.~Hobson, E.~F.~Pace-Schott, and R.~Stickgold,
``Dreaming and the brain: Toward a cognitive neuroscience of conscious states,''
\textit{Behav.\ Brain Sci.} \textbf{23}, 793 (2000).

\bibitem{Tononi1998}
G.~Tononi and G.~M.~Edelman,
``Consciousness and complexity,''
\textit{Science} \textbf{282}, 1846 (1998).

\bibitem{Libet1985}
B.~Libet,
``Unconscious cerebral initiative and the role of conscious will in voluntary action,''
\textit{Behav.\ Brain Sci.} \textbf{8}, 529 (1985).

\bibitem{Clauset2009}
A.~Clauset, C.~R.~Shalizi, and M.~E.~J.~Newman,
``Power-law distributions in empirical data,''
\textit{SIAM Rev.} \textbf{51}, 661 (2009).

\end{thebibliography}

\end{document}
