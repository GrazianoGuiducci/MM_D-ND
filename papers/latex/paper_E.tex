%==============================================================================
% PAPER E - COSMOLOGICAL EXTENSION OF THE DUAL-NON-DUAL FRAMEWORK
% Target Journal: Classical and Quantum Gravity / Foundations of Physics
% Document Class: revtex4-2 (APS-compatible)
%==============================================================================

\documentclass[aps,prd,11pt,notitlepage,nofootinbib,longbibliography]{revtex4-2}

%==============================================================================
% PACKAGES
%==============================================================================

\usepackage[utf8]{inputenc}
\usepackage[T1]{fontenc}
\usepackage{amsmath}
\usepackage{amssymb}
\usepackage{mathrsfs}
\usepackage{braket}
\usepackage{amsthm}
\usepackage{hyperref}
\usepackage{cleveref}
\usepackage{geometry}
\usepackage{setspace}
\usepackage{graphicx}
\usepackage{float}
\usepackage{booktabs}
\usepackage{longtable}
\usepackage{array}
\usepackage{dnd_shared}

%==============================================================================
% HYPERREF CONFIGURATION
%==============================================================================

\hypersetup{
    colorlinks=true,
    linkcolor=blue,
    citecolor=blue,
    urlcolor=blue,
    bookmarksnumbered=true,
    pdftitle={Cosmological Extension of the Dual-Non-Dual Framework},
    pdfauthor={D-ND Research Collective},
    pdfsubject={D-ND emergence, cosmology, modified Einstein equations},
    pdfkeywords={D-ND emergence, cosmology, modified Einstein equations, inflation, dark energy, NT singularity, cyclic coherence, informational energy-momentum tensor, quantum cosmology, structure formation, CMB signatures, DESI BAO constraints}
}

%==============================================================================
% GEOMETRY
%==============================================================================

\geometry{
    letterpaper,
    top=1in,
    bottom=1in,
    left=1in,
    right=1in
}

%==============================================================================
% CUSTOM COMMANDS (Paper E specific)
%==============================================================================

\newcommand{\Tinfo}{T_{\mu\nu}^{\text{info}}}
\newcommand{\MC}{M_C(t)}
\newcommand{\MCz}{M_C(z)}
\newcommand{\wemerge}{w_{\text{emerge}}}
\newcommand{\rhoinfo}{\rho_{\text{info}}}
\newcommand{\Pinfo}{P_{\text{info}}}
\newcommand{\rhoLambda}{\rho_\Lambda}
\newcommand{\ThetaNT}{\Theta_{\text{NT}}}
\newcommand{\Lcollapse}{\mathcal{L}_{\text{field-collapse}}}
\newcommand{\Lemerge}{\mathcal{L}_{\text{emerge}}}
\newcommand{\Zfield}{Z_{\text{field}}}
\newcommand{\fNL}{f_{\text{NL}}}

%==============================================================================
% DOCUMENT
%==============================================================================

\begin{document}

\title{Cosmological Extension of the Dual-Non-Dual Framework:\\
Emergence at Universal Scales}

\author{D-ND Research Collective}
\affiliation{Independent Research}

\date{February 14, 2026}

%==============================================================================
% ABSTRACT
%==============================================================================

\begin{abstract}
We extend the Dual-Non-Dual (D-ND) framework from quantum-mechanical emergence (Paper~A) to cosmological scales, proposing that the universe's large-scale structure and dynamical evolution emerge from the interplay of quantum potentiality ($\NT$) and the emergence operator ($\emerge$) modulated by spacetime curvature. We introduce modified Einstein field equations (S7) incorporating an informational energy-momentum tensor: $G_{\mu\nu} + \Lambda g_{\mu\nu} = 8\pi G \Tinfo$, where $\Tinfo$ arises from the spatial integral of the curvature operator $C$ and captures the effect of quantum emergence on classical spacetime geometry. Crucially, we establish that equation (S7) is not a phenomenological ansatz but a structural necessity derived from Axiom~P4 (Holographic Manifestation): any spacetime geometry must encode the collapse mechanism of the emergence field $\Phi_A$. The informational tensor is grounded thermodynamically in Gibbs free energy gradients, satisfies the conservation law $\nabla^\mu \Tinfo = 0$ via the Bianchi identity, and preserves diffeomorphism invariance. We derive modified Friedmann equations incorporating D-ND emergence dynamics, showing how inflation emerges as a phase of rapid quantum differentiation coinciding with a Bloch wall domain transition, and how dark energy corresponds to residual non-relational potential $V_0$. The Non-Trivial (NT) singularity condition $\ThetaNT = \lim_{t\to 0} (R(t)e^{i\omega t}) = R_0$ replaces the classical singularity with a boundary condition at the emergence threshold. We establish that time itself emerges from thermodynamic irreversibility, grounded in the Clausius inequality $\oint dQ/T \leq 0$ and the six-phase cognitive pipeline from indeterminacy to determinacy. Antigravity is revealed as the orthogonal pole of gravity through Poynting vector mechanics, corresponding to the dipolar structure of the modified equations and providing three concrete falsification tests: (1)~Bloch wall signatures in CMB polarization, (2)~Riemann eigenvalue structure in DESI baryon acoustic oscillation data, and (3)~dark energy equation-of-state deviation $w(z) = -1 + 0.05(1-\MCz)$ measurable by DESI Year-2 (2025) and decisive by Year-3 (2026). We establish a cyclic coherence condition $\OmegaNT = 2\pi i$ governing the overall temporal topology of cosmic evolution, connecting to conformal cyclic cosmology and information preservation across cosmic cycles. We present a comprehensive observational prediction table spanning CMB, structure growth, dark energy, gravitational waves, and large-scale structure, with quantitative comparisons to $\Lambda$CDM, Loop Quantum Cosmology, and Conformal Cyclic Cosmology. The framework is falsifiable but receives substantial theoretical grounding from corpus-extracted mathematical structures.
\end{abstract}

\keywords{D-ND emergence, cosmology, modified Einstein equations, inflation, dark energy, NT singularity, cyclic coherence, informational energy-momentum tensor, quantum cosmology, structure formation, CMB signatures, DESI BAO constraints}

\maketitle
\tableofcontents

%==============================================================================
\section{Introduction}
\label{sec:intro}
%==============================================================================

\subsection{The Cosmological Problem of Emergence}
\label{sec:cosmo-problem}

The universe exhibits a fundamental asymmetry: it began in an extraordinarily simple, nearly homogeneous state (as evidenced by the cosmic microwave background's isotropy to one part in $10^5$) and evolved toward increasingly complex, structured configurations---galaxies, stars, life. Yet the laws governing this evolution are time-symmetric at the microscopic level. Three mechanisms attempt to resolve this paradox:

\begin{enumerate}
    \item \textbf{Inflationary dynamics}: Exponential expansion amplifies quantum vacuum fluctuations to classical scales \cite{Guth1981,Linde1986}.
    \item \textbf{Environmental decoherence at cosmic scales}: Wheeler-DeWitt and other quantum gravity approaches, though unclear how a closed-system universe ``decoheres.''
    \item \textbf{Entropic gravity and holographic emergence}: Spacetime geometry itself emerges from quantum entanglement structure \cite{Verlinde2011,RyuTakayanagi2006}.
\end{enumerate}

Yet none directly address: \emph{How does classical spacetime emerge from a quantum substrate within a closed system?}

\subsection{Gap in Cosmological Theory}
\label{sec:gap}

Standard cosmology presupposes a classical spacetime metric $g_{\mu\nu}$ from the outset and seeks to explain how \emph{structures} form within it. Quantum cosmology (Wheeler-DeWitt, loop quantum cosmology) attempts to describe the universe from a quantum state but struggles with the problem of time: if the universe is timeless at the quantum level, how does the temporal arrow emerge?

Paper~A (the quantum D-ND framework) provides a mechanism for closed-system emergence at microscopic scales via the primordial state $\NT$ and the emergence operator $\emerge$. This work extends that mechanism to cosmology, proposing:

\begin{itemize}
    \item \textbf{The universe begins in a state of maximal quantum non-duality} ($\NT$), containing all possibilities with equal weight.
    \item \textbf{Spacetime curvature acts as an emergence filter}, modulating which quantum modes actualize into classical configurations.
    \item \textbf{The modified Einstein equations couple geometry to informational emergence}, creating a feedback loop where quantum emergence shapes curvature, which in turn gates further emergence.
\end{itemize}

\subsection{Contributions}
\label{sec:contributions}

\begin{enumerate}
    \item \textbf{Modified Einstein equations} with informational energy-momentum tensor $\Tinfo$ derived from D-ND emergence dynamics.
    \item \textbf{Conservation law derivation}: Explicit proof that $\nabla^\mu \Tinfo = 0$ from the Bianchi identity, ensuring consistency.
    \item \textbf{Derivation of modified Friedmann equations} incorporating emergence measure dynamics, showing inflation as a phase of rapid $\MC$ evolution.
    \item \textbf{Resolution of the initial singularity} via the NT singularity condition $\ThetaNT$, reframing the Big Bang as a boundary condition on emergence.
    \item \textbf{Cyclic coherence condition} $\OmegaNT = 2\pi i$ governing multi-cycle cosmic evolution and information preservation.
    \item \textbf{DESI-constrained predictions}: Quantitative comparison with 2024 baryon acoustic oscillation data, showing testable deviations at 1--3\% level.
    \item \textbf{Comparative framework}: Detailed predictions against $\Lambda$CDM, Loop Quantum Cosmology, and Conformal Cyclic Cosmology.
    \item \textbf{Falsifiability framework}: Explicit predictions distinguishing D-ND cosmology from competitors in specific regimes.
\end{enumerate}

%==============================================================================
\section{Modified Einstein Equations with Informational Energy-Momentum Tensor}
\label{sec:einstein}
%==============================================================================

\subsection{The Informational Energy-Momentum Tensor}
\label{sec:info-tensor}

We propose a generalization of Einstein's field equations incorporating the effect of quantum emergence on spacetime:
\begin{equation}\label{eq:S7}
    \boxed{G_{\mu\nu} + \Lambda g_{\mu\nu} = 8\pi G \Tinfo}
\end{equation}
where $\Tinfo$ is the informational energy-momentum tensor, sourced by the emergence operator's action on spacetime geometry.

\textbf{Definition} of $\Tinfo$:
\begin{equation}\label{eq:Tinfo-def}
    \Tinfo = \frac{\hbar}{c^2} \int d^3\mathbf{x} \, \Kgen(\mathbf{x},t) \, \partial_\mu R(t) \, \partial_\nu R(t)
\end{equation}
where:
\begin{itemize}
    \item $\Kgen(\mathbf{x},t) = \nabla \cdot (J(\mathbf{x},t) \otimes F(\mathbf{x},t))$ is the generalized informational curvature density
    \item $J(\mathbf{x},t)$ is the information flux density
    \item $F(\mathbf{x},t)$ is a generalized force field encoding the action of $\emerge$
    \item $R(t) = U(t)\emerge C\NT$ is the emergent cosmic state (with curvature modulation $C$)
\end{itemize}

\begin{remark}[Dimensional Consistency and Effective Field Interpretation]
In the definition above, $R(t) = U(t)\emerge C\NT$ is a quantum state. To obtain a dimensionally consistent energy-momentum tensor, we identify $R(t)$ with an effective classical scalar field $\phi(x,t)$ via the coarse-graining procedure of Paper~A \S5.2: $\phi(x,t) \equiv \langle x|R(t)\rangle$ in the position representation, which has dimensions of $[\text{length}]^{-3/2}$. The product $\partial_\mu \phi \, \partial_\nu \phi$ then carries dimensions of $[\text{length}]^{-5}$, and with the prefactor $\hbar/c^2$ and the spatial integral $\int d^3\mathbf{x}$, the tensor $\Tinfo$ acquires the correct dimensions of $[\text{energy}][\text{length}]^{-3}$ (energy density). In the semiclassical limit, this reduces to the canonical energy-momentum tensor for a scalar field with D-ND-modified potential.
\end{remark}

\textbf{Explicit Metric Perturbation Form:}

The informational energy-momentum tensor couples to spacetime geometry through metric perturbations. The perturbed spacetime metric is:
\begin{equation}\label{eq:metric-pert}
    \boxed{g_{\mu\nu}(x,t) = g_{\mu\nu}^{(0)} + h_{\mu\nu}(\Kgen, e^{\pm\lambda Z})}
\end{equation}
where $g_{\mu\nu}^{(0)}$ is the flat Minkowski metric, $h_{\mu\nu}$ is the metric perturbation encoding D-ND corrections, and the $\pm$ signs reflect the dipolar structure: $+$ encodes convergence (gravity), $-$ encodes divergence (antigravity).

\textbf{Derivation of the Metric Perturbation from $\Kgen$:}

In the weak-field limit ($|h_{\mu\nu}| \ll 1$), the trace-reversed perturbation $\bar{h}_{\mu\nu} = h_{\mu\nu} - \frac{1}{2}\eta_{\mu\nu}h$ satisfies:
\begin{equation}\label{eq:linearized}
    \Box \bar{h}_{\mu\nu} = -16\pi G \, \Tinfo
\end{equation}

Solving via the retarded Green's function:
\begin{equation}\label{eq:green-soln}
    h_{\mu\nu}(\mathbf{x},t) = 4G \int \frac{\Tinfo(\mathbf{x}',t_{\text{ret}})}{|\mathbf{x}-\mathbf{x}'|} d^3\mathbf{x}'
\end{equation}

This establishes the explicit bridge between the D-ND Lagrangian dynamics (Paper~B) and cosmological spacetime geometry.

\subsubsection{The Singularity Constant $\GS$ and Its Proto-Axiomatic Role}

The gravitational constant $G_N$ in Einstein's field equations acquires a deeper interpretation within the D-ND framework. From the proto-axiomatic structure (cf.\ Paper~A \S2.3), $G_N$ is identified as the physical manifestation of the \textbf{Singularity Constant} $\GS$---the unitary reference for all coupling constants outside the dual regime.

\begin{definition}
The Singularity Constant $\GS$ is the proto-axiomatic parameter that mediates between the non-relational potential $V_0$ and the emergent sectors $\Phi_+, \Phi_-$:
\begin{equation}\label{eq:GS-def}
    \GS \equiv \frac{\hbar \cdot \Gamma_{\text{emerge}}}{\langle(\Delta\hat{V}_0)^2\rangle}
\end{equation}
where $\Gamma_{\text{emerge}}$ is the emergence rate and $\langle(\Delta\hat{V}_0)^2\rangle$ is the variance of the non-relational potential.
\end{definition}

In the low-energy, macroscopic limit: $\GS \to G_N = 6.674 \times 10^{-11} \, \text{m}^3 \text{kg}^{-1} \text{s}^{-2}$. With this identification, equation~\eqref{eq:S7} becomes $G_{\mu\nu} + \Lambda g_{\mu\nu} = 8\pi \GS \cdot \Tinfo$, where the factor $8\pi \GS$ is the product of the proto-axiomatic singularity constant with the geometric factor $8\pi$ arising from the Gauss-Bonnet structure of 4-dimensional spacetime.

\subsection{Derivation from the D-ND Lagrangian: Structural Inference from Axiom P4}
\label{sec:derivation-P4}

The informational energy-momentum tensor is \textbf{not a phenomenological ansatz} but a \textbf{structural requirement} derived from the D-ND axioms, specifically \textbf{Axiom~P4 (Holographic Manifestation, corresponding to Paper~A Axiom~A$_6$)}.

Axiom~P4 establishes that all physical manifestation flows through the collapse of the potential field $\Phi_A$ into classical reality $R$. In General Semantics terms, the map (spacetime geometry) and the territory (quantum field) are structurally coupled: the geometry must encode the collapse mechanism. Therefore:
\begin{equation}\label{eq:P4-consequence}
    \boxed{\text{Any spacetime geometry must encode the collapse dynamics of } \Phi_A}
\end{equation}

\textbf{Derivation from Action Principle:}

Consider the D-ND-extended Lagrangian density:
\begin{equation}\label{eq:L-DND-cosmo}
    \mathcal{L}_{\text{D-ND}} = \frac{R}{16\pi G} + \mathcal{L}_M + \Lemerge + \Lcollapse
\end{equation}
where:
\begin{itemize}
    \item $R/(16\pi G)$ is the standard Einstein-Hilbert Lagrangian
    \item $\mathcal{L}_M$ is the matter Lagrangian
    \item $\Lemerge = \Kgen \cdot \MC \cdot (\partial_\mu \phi)(\partial^\mu \phi)$ couples the emergence measure to scalar field gradients
    \item $\Lcollapse = -\frac{\hbar}{c^3}\nabla_\mu \nabla_\nu \ln \Zfield$ is the free-energy gradient of field collapse, where $\Zfield = \int \mathcal{D}\phi \, e^{-S[\phi]/\hbar}$ is the field partition function
\end{itemize}

Variation of $S = \int d^4x \sqrt{-g} \mathcal{L}_{\text{D-ND}}$ with respect to $g_{\mu\nu}$ yields:
\begin{equation}\label{eq:variation}
    \frac{\delta S}{\delta g_{\mu\nu}} = 0 \implies G_{\mu\nu} + \Lambda g_{\mu\nu} = 8\pi G(T_{\mu\nu}^{(M)} + \Tinfo)
\end{equation}
where the informational contribution arises from the field-collapse term:
\begin{equation}\label{eq:Tinfo-derived}
    \Tinfo = \frac{\hbar}{8\pi c^2} \Kgen \, \dot{M}_C(t) \, (\partial_\mu \phi)(\partial_\nu \phi)
\end{equation}

\begin{remark}[Ansatz Status Elevated to Axiomatic Consequence]
\textbf{Relationship to Paper~A's Axiom System:} The cosmological axioms P0--P4 constitute an extension of Paper~A's foundational axioms A$_1$--A$_6$. Specifically: P0 generalizes A$_2$ (non-duality as ontological invariance), P1 extends A$_5$ (autological consistency as autoconservation), P2 connects to A$_3$ (evolutionary input-output as dialectic metabolism), and P4 is identical to A$_6$ (holographic manifestation). P3 (Emergence Dynamics) combines elements of A$_1$ and A$_3$.

The derivation follows directly from D-ND axioms P0--P4:
\begin{itemize}
    \item \textbf{P0 (Ontological Invariance):} Forms are manifestations of unity; essence is invariable
    \item \textbf{P1 (Autoconservation):} System rejects contradictions; structural integrity prevails
    \item \textbf{P2 (Dialectic Metabolism):} Field assimilates information through phase transitions
    \item \textbf{P4 (Holographic Manifestation):} Coherent collapse is guided by topological constraint
\end{itemize}
However, a fully independent derivation from quantum gravity first principles (e.g., the spectral action principle of Chamseddine-Connes, or asymptotic safety) remains an open problem.
\end{remark}

\subsection{Relationship to Verlinde's Entropic Gravity}
\label{sec:verlinde}

Verlinde (2011, 2016) proposes that gravity emerges from entropic forces on particle configurations \cite{Verlinde2011,Verlinde2016}. The D-ND approach is complementary: rather than deriving gravity from entropy gradients of existing matter configurations, we derive it from the \emph{emergence} of those configurations themselves:
\begin{equation}\label{eq:verlinde-connection}
    F_{\text{entropic}} \propto \nabla(\Delta S) \leftrightarrow F_{\text{emerge}} \propto \nabla \dot{M}_C(t)
\end{equation}

The informational energy-momentum tensor $\Tinfo$ thus provides a dynamical realization of entropic gravity at the quantum-to-classical transition.

\subsection{Explicit Derivation of Informational Energy-Momentum Conservation}
\label{sec:conservation}

A fundamental requirement of any extension to Einstein's field equations is:
\begin{equation}\label{eq:conservation}
    \boxed{\nabla^\mu \Tinfo = 0}
\end{equation}

\textbf{Derivation from Bianchi Identity:}

The Bianchi identity for the Riemann tensor:
\begin{equation}\label{eq:bianchi}
    \nabla_\lambda R_{\mu\nu\rho\sigma} + \nabla_\mu R_{\nu\lambda\rho\sigma} + \nabla_\nu R_{\lambda\mu\rho\sigma} = 0
\end{equation}

Contracting twice to obtain the differential Bianchi identity: $\nabla^\mu G_{\mu\nu} = 0$, where $G_{\mu\nu} = R_{\mu\nu} - \frac{1}{2}Rg_{\mu\nu}$.

From equation~\eqref{eq:S7}, $G_{\mu\nu} + \Lambda g_{\mu\nu} = 8\pi G \Tinfo$, we have $\nabla^\mu G_{\mu\nu} = 8\pi G \nabla^\mu \Tinfo$. The left side vanishes by the Bianchi identity, yielding $\nabla^\mu \Tinfo = 0$.

\textbf{Physical interpretation}: Information carried by the emergence operator is conserved throughout cosmic evolution. No information is created or destroyed; it is only redistributed through the emergence measure $\MC$.

%==============================================================================
\section{Cosmological D-ND Dynamics}
\label{sec:cosmo-dynamics}
%==============================================================================

\subsection{FRW Metric with D-ND Corrections}
\label{sec:FRW}

We assume a spatially isotropic and homogeneous universe described by the Friedmann-Robertson-Walker metric:
\begin{equation}\label{eq:FRW}
    ds^2 = -dt^2 + a(t)^2\left[\frac{dr^2}{1-kr^2} + r^2(d\theta^2 + \sin^2\theta \, d\phi^2)\right]
\end{equation}

In the D-ND framework, the scale factor $a(t)$ is constrained by the emergence measure $\MC$ and the curvature operator:
\begin{equation}\label{eq:scale-factor}
    a(t) = a_0 \left[1 + \xi \cdot \MC \cdot e^{H(t) \cdot t}\right]^{1/3}
\end{equation}
where $a_0$ is the initial scale factor, $\xi$ is a coupling constant (order unity), and $H(t)$ is the Hubble parameter.

\subsection{Modified Friedmann Equations}
\label{sec:friedmann}

The standard Friedmann equations are modified by coupling to $\MC$:
\begin{equation}\label{eq:friedmann1}
    \boxed{H^2 = \frac{8\pi G}{3}\left[\rho + \rhoinfo\right] - \frac{k}{a^2}}
\end{equation}
\begin{equation}\label{eq:friedmann2}
    \boxed{\dot{H} + H^2 = -\frac{4\pi G}{3}\left[(\rho + \rhoinfo) + 3(P + \Pinfo)\right]}
\end{equation}
where the informational density and pressure are:
\begin{equation}\label{eq:rhoinfo}
    \rhoinfo(t) = \frac{\hbar \omega_0}{c^2} \cdot \dot{M}_C(t) \cdot \MC
\end{equation}
\begin{equation}\label{eq:Pinfo}
    \Pinfo(t) = -\frac{1}{3}\rhoinfo(t) \cdot \wemerge(\MC)
\end{equation}
with $\wemerge(\MC)$ an equation-of-state parameter depending on the emergence phase:
\begin{itemize}
    \item \textbf{Pre-emergence} ($M_C \approx 0$): $\wemerge \approx -1$ (vacuum-like, drives expansion)
    \item \textbf{Emergence phase} ($0 < M_C < 1$): $\wemerge \approx -1/3$ (radiation-like)
    \item \textbf{Post-emergence} ($M_C \approx 1$): $\wemerge \approx -\epsilon$ (matter-like, with small residual)
\end{itemize}

\subsection{Inflation as D-ND Emergence Phase}
\label{sec:inflation}

In D-ND cosmology, \textbf{inflation corresponds to the rapid emergence phase} where $\MC$ evolves from $\approx 0$ to $\approx 1$. The emergence timescale is:
\begin{equation}\label{eq:tau-emerge}
    \tau_e \sim \hbar / \Delta E_{\text{effective}}
\end{equation}

The number of e-folds of inflation is:
\begin{equation}\label{eq:efolds}
    N_e = \int_0^{t_*} H(t) \, dt \approx \int_0^{1} \frac{H_0}{\dot{M}_C(M_C)} \, dM_C
\end{equation}
This predicts a finite number of e-folds determined by the emergence operator's spectral properties, without need for slow-roll parameters.

The power spectrum of primordial perturbations is:
\begin{equation}\label{eq:power-spectrum}
    P_{\delta}(k) \propto \MC(t_*) \cdot |\langle k|\emerge\NT\rangle|^2 \cdot \left(1 - |\langle k|U(t)\emerge\NT\rangle|^2\right)
\end{equation}
where $t_*$ is the time when mode $k$ exits the cosmological horizon. Modes with emergence eigenvalues close to $1/2$ (maximally uncertain) produce the largest perturbations.

%==============================================================================
\section{The NT Singularity: Resolving the Initial Condition}
\label{sec:NT-singularity}
%==============================================================================

\subsection{The NT Singularity Condition}
\label{sec:NT-condition}

The D-ND framework replaces the classical singularity with a boundary condition:
\begin{equation}\label{eq:A8}
    \boxed{\ThetaNT = \lim_{t \to 0^+} \left[R(t) e^{i\omega t}\right] = R_0 \quad \text{(A8)}}
\end{equation}
where $R(t) = U(t)\emerge C\NT$ is the emergent cosmic state, $e^{i\omega t}$ represents phase evolution, and $R_0$ is the limiting emergent state at the threshold of actualization.

As $t \to 0$, quantum evolution has not yet begun; the universe exists in a state of pure potentiality. The condition $\ThetaNT = R_0$ specifies the ``seed'' state from which all subsequent emergence unfolds. It is not a singularity in the classical sense but a \emph{boundary of actualization}: the interface between non-being and being.

\subsection{Resolution of the Initial Singularity via $\NT$}
\label{sec:resolution}

In the D-ND picture:
\begin{enumerate}
    \item \textbf{Before emergence} ($t < 0$): The universe is $\NT$---a state of perfect non-duality in which no classical spacetime exists. There is no ``time before the Big Bang'' because time itself is emergent.
    \item \textbf{Emergence threshold} ($t = 0$): The emergence operator $\emerge$ begins to act on $\NT$, actualizing quantum modes into classical configurations.
    \item \textbf{Post-emergence} ($t > 0$): The universe evolves according to modified Friedmann equations, with quantum emergence rate $\dot{M}_C(t)$ continuously shaping the expansion history.
\end{enumerate}

The avoidance of the classical singularity follows from: (i)~\textbf{Regularity of $\MC$}: For reasonable emergence operators, $M_C(0^+)$ is finite (typically $\sim 10^{-3}$ to $10^{-1}$); (ii)~\textbf{Finite initial curvature}: From equation~\eqref{eq:S7}, the initial Ricci curvature $R_{\mu\nu}(0^+) \sim 8\pi G \cdot \Tinfo(0^+)$ is bounded.

\subsection{Connection to Hartle-Hawking No-Boundary Proposal}
\label{sec:hartle-hawking}

Hartle and Hawking (1983) propose that the universe has no boundary in spacetime \cite{HartleHawking1983}. Their no-boundary wave function obeys the Wheeler-DeWitt equation: $\hat{H}_{\text{WDW}} \Psi[\mathbf{g}] = 0$.

The D-ND framework is compatible: we interpret $\NT$ as an approximation to the no-boundary $\Psi_0[\mathbf{g}]$---a universal state in which all geometries are superposed. The action of $\emerge$ on $\NT$ selects the classical trajectory that dominates the path integral. The NT singularity condition $\ThetaNT$ thus specifies the initial value of the emergent cosmic state, ensuring subsequent classical evolution is well-defined and non-singular.

%==============================================================================
\section{Cyclic Coherence and Cosmic Evolution}
\label{sec:cyclic}
%==============================================================================

\subsection{The Cyclic Coherence Condition}
\label{sec:cyclic-condition}

The D-ND framework suggests multiple cosmic cycles, governed by:
\begin{equation}\label{eq:S8}
    \boxed{\OmegaNT = 2\pi i \quad \text{(S8)}}
\end{equation}

This phase condition encodes: \textbf{Periodicity} ($2\pi$)---the universe returns to a topologically equivalent state; \textbf{Imaginary nature} ($i$)---the cycle is in complexified, relational time (consistent with the Page-Wootters mechanism).

The explicit form arises from requiring the total phase accumulated over one cosmic cycle to be:
\begin{equation}\label{eq:omega-total}
    \Omega_{\text{total}} = \int_0^{t_{\text{cycle}}} \left[\frac{d}{dt}\arg(f(t))\right] \, dt = 2\pi
\end{equation}
where $f(t) = \langle NT|U(t)\emerge C\NT\rangle$ is the overlap function.

\subsection{Penrose's Conformal Cyclic Cosmology Connection}
\label{sec:CCC}

Penrose's Conformal Cyclic Cosmology (CCC) proposes infinite cycles (aeons) with the far future of one aeon identified with the initial conditions of the next via conformal rescaling \cite{Penrose2005,Penrose2010}. The cyclic coherence condition $\OmegaNT = 2\pi i$ can be understood as the D-ND version of CCC's conformal matching condition---imposing a phase-space matching condition on the emergence measure rather than matching Weyl curvature tensors.

\subsection{Information Preservation Across Cycles}
\label{sec:info-preservation}

Each cosmic cycle: (1)~begins with emergence from $\NT$ (maximum entropy); (2)~continues with actualization via $\emerge$ ($\MC$ grows); (3)~evolves with thermodynamic entropy increase; (4)~ends by reconvergence toward non-duality; (5)~transfers information to the next cycle via phase matching.

The information transferred between aeons is:
\begin{equation}\label{eq:info-transfer}
    I_{\text{transfer}} = k_B \int_0^{t_{\text{cycle}}} \frac{dS_{\text{vN}}}{dt} \, dt
\end{equation}
where $S_{\text{vN}}(t) = -\text{Tr}[\rho(t) \ln \rho(t)]$ is the von Neumann entropy.

%==============================================================================
\section{Observational Predictions}
\label{sec:predictions}
%==============================================================================

\subsection{CMB Signatures of D-ND Emergence}
\label{sec:CMB}

\subsubsection{Non-Gaussian Bispectrum from Emergence-Gated Fluctuations}

In D-ND, non-Gaussianity arises from the spectral structure of $\emerge$:
\begin{equation}\label{eq:bispectrum}
    \langle \delta k_1 \delta k_2 \delta k_3 \rangle \propto \sum_{j,k,l} \lambda_j \lambda_k \lambda_l \, \delta^3(\mathbf{k}_1 + \mathbf{k}_2 + \mathbf{k}_3)
\end{equation}

\textbf{Prediction}: For smooth spectral features, $\fNL^{\text{equilateral}} \sim 5$--$20$, consistent with Planck 2018 constraints ($\fNL^{\text{equilateral}} < 25$) \cite{Planck2018NG}. For sharper features, $\fNL$ increases further but manifests in non-standard bispectrum shapes (emergence-type templates) not yet constrained. Testable by CMB-S4.

\subsubsection{Anomalous Power Suppression at Super-Horizon Scales}

The power spectrum:
\begin{equation}\label{eq:suppression}
    P_\delta(k) \propto \left[1 - (1 - M_C(t_*))_k\right]^2
\end{equation}

\textbf{Prediction}: Sharp suppression at multipoles $\ell \lesssim 10$ (super-horizon scales). Current Planck data hint at such suppression (the ``Planck tension'').

\subsubsection{Scale-Dependent Running from Emergence Rate}

\textbf{Prediction}: D-ND predicts scale-dependent running that differs from slow-roll predictions by order-unity factors, measurable at the $2$--$3\sigma$ level.

\subsection{Structure Formation from $\MC$ Dynamics}
\label{sec:structure}

\subsubsection{Linear Growth Factor with Emergence Feedback}

Growth is modulated by the curvature-emergence coupling:
\begin{equation}\label{eq:growth}
    f_{\text{D-ND}}(a) = f_{\text{GR}}(a) \cdot \left[1 + \alpha_e \cdot (1 - M_C(a))\right]
\end{equation}
where $\alpha_e \sim 0.1$. At recent epochs ($z < 5$), the correction vanishes, recovering GR.

\subsubsection{Non-Linear Clustering from Emergence-Induced Halo Bias}

\begin{equation}\label{eq:bias}
    b_{\text{D-ND}}(z, M) = b_{\text{matter}}(z, M) \cdot \left[1 + \beta_e \cdot M_C(z) \cdot \Psi(M)\right]
\end{equation}
where $\Psi(M)$ depends on halo mass, encoding preferential actualization of certain mass scales. Testable via DESI, Euclid, Roman Space Telescope.

\subsection{Dark Energy as Residual $V_0$ Potential and DESI BAO Constraints}
\label{sec:dark-energy}

In the D-ND framework, dark energy is identified with the non-relational background potential $\hat{V}_0$:
\begin{equation}\label{eq:rhoLambda}
    \rhoLambda = \rho_0 \cdot (1 - \MC)^p
\end{equation}
where $\rho_0 \sim 10^{-47}$ GeV$^4$ and $p \sim 2$.

The equation of state:
\begin{equation}\label{eq:wz}
    w(z) = -1 + \epsilon(z) \quad \text{where} \quad \epsilon(z) \approx 0.05 \cdot (1 - \MCz)
\end{equation}

\textbf{DESI 2024 BAO Comparison:}

The BAO scale is defined by the comoving distance $d_A(z) = \frac{c}{H_0} \int_0^z \frac{dz'}{E(z')}$, and the D-ND modified Hubble parameter:
\begin{equation}\label{eq:H-DND}
    H_{\text{D-ND}}^2(z) = H_0^2 \left[\Omega_m(1+z)^3 + \rhoLambda(z)/\rho_c + \Omega_k(1+z)^2\right]
\end{equation}

\begin{table}[H]
\caption{Quantitative predictions for $w(z)$ and angular diameter distance deviations from $\Lambda$CDM.}
\label{tab:wz-predictions}
\begin{ruledtabular}
\begin{tabular}{ccccc}
$z$ & $\Lambda$CDM $w(z)$ & D-ND $w(z)$ & $d_A$ diff.\ (\%) & DESI $>2\sigma$? \\
\midrule
0.0 & $-1.000$ & $-1.000$ & 0.0 & No \\
0.5 & $-1.000$ & $-0.975$ & $+0.8$ & Marginal (1.5$\sigma$) \\
1.0 & $-1.000$ & $-0.950$ & $+1.6$ & Possible (2--3$\sigma$) \\
1.5 & $-1.000$ & $-0.920$ & $+2.4$ & Likely (2.5--3$\sigma$) \\
2.0 & $-1.000$ & $-0.890$ & $+3.2$ & Strong (3--4$\sigma$) \\
\end{tabular}
\end{ruledtabular}
\end{table}

If $V_0$ has quantum fluctuations with variance $\sigmaV$, then the dark energy density becomes dynamical:
\begin{equation}\label{eq:rhoLambda-dynamic}
    \rhoLambda(t) = \sigmaV(t) \cdot (1 - \MC)
\end{equation}

\subsection{Antigravity as the Negative Solution: The $t = -1$ Direction}
\label{sec:antigravity}

\subsubsection{The Dipolar Structure and Two Solutions for Temporal Evolution}

The D-ND framework is fundamentally dipolar, producing two solutions:
\begin{equation}\label{eq:dipolar}
    \boxed{t = +1 \quad \text{(Convergence/Gravity)} \quad \text{and} \quad t = -1 \quad \text{(Divergence/Antigravity)}}
\end{equation}

The standard cosmological picture privileges the $t = +1$ solution. Yet D-ND dipolar logic demands both exist simultaneously as complementary poles.

\subsubsection{Analogy to Dirac's Equation and the Excluded-Third Problem}

Dirac's relativistic equation produces $E = \pm\sqrt{(\mathbf{p}c)^2 + (m_e c^2)^2}$. Dismissing the negative solution violates mathematical structure; the same principle applies to the $t = -1$ pole in D-ND cosmology.

The equation of motion in D-ND cosmology is:
\begin{equation}\label{eq:dipolar-EOM}
    \dot{a}(t) \propto a(t) \cdot [H_+ \cdot t_+ + H_- \cdot t_-]
\end{equation}
where $H_\pm$ are the Hubble parameters in the $\pm 1$ directions, simultaneously present and dynamically coupled.

\subsubsection{The Poynting Vector Mechanism: Orthogonal Exit from Oscillation Plane}

\begin{equation}\label{eq:poynting}
    \boxed{\vec{S} = \frac{1}{\mu_0} (\vec{E} \times \vec{B})}
\end{equation}

The stress-energy tensor encodes both components:
\begin{equation}\label{eq:T-total}
    T_{\mu\nu}^{\text{total}} = T_{\mu\nu}^{(+)} + T_{\mu\nu}^{(-)}
\end{equation}
with the antigravity contribution:
\begin{equation}\label{eq:T-minus}
    T_{\mu\nu}^{(-)} \propto \epsilon_{\mu\nu\rho\sigma} T^{(+)\rho\lambda} T^{(+)\sigma}{}_\lambda
\end{equation}

The Levi-Civita symbol $\epsilon_{\mu\nu\rho\sigma}$ embodies the cross-product operation in curved spacetime---the fundamental topological reason why antigravity exists as the orthogonal pole.

\subsubsection{The Bloch Wall Mechanism: Inflation as Domain Transition}

In D-ND cosmology, the universe transitions from the low-emergence domain ($M_C \approx 0$) to the high-emergence domain ($M_C \approx 1$). This transition cannot be instantaneous---the intermediate region \emph{is} the inflationary epoch.

The cosmological Bloch wall explains inflation's key features:
\begin{enumerate}
    \item \textbf{Zero external gravity} within the inflation window---domain forces balance, resolving the flatness problem.
    \item \textbf{Maximum internal field density}---energy density peaks at the transition.
    \item \textbf{Finite wall width determines inflation duration}---set by the emergence operator's spectral properties.
    \item \textbf{Oscillatory behavior within the wall}---predicts features in the primordial power spectrum.
\end{enumerate}

\subsubsection{Gravity and Antigravity as Poles of Emergence}

\textbf{Gravity} ($t = +1$): Convergence of quantum modes toward classical actualization. \textbf{Antigravity} ($t = -1$): Divergence from actualization---systematic spreading of actualized states back into superposition. Both occur simultaneously with equal strength in the D-ND dipole.

At local scales (galaxies, stars): gravity dominates ($M_C \approx 1$). At cosmological scales (expansion): antigravity dominates (partial emergence). Dark energy is the observable manifestation of the $t = -1$ pole.

\subsubsection{Structural Basis for Antigravity: Not a New Force, But Structural Necessity}

The modified field equations with explicit poles:
\begin{align}
    G_{\mu\nu}^{(+)} + \Lambda g_{\mu\nu} &= 8\pi G T_{\mu\nu}^{(+)} \quad \text{(Gravity pole)} \label{eq:gravity-pole} \\
    G_{\mu\nu}^{(-)} - \Lambda g_{\mu\nu} &= 8\pi G T_{\mu\nu}^{(-)} \quad \text{(Antigravity pole)} \label{eq:antigravity-pole}
\end{align}
with the dipolar constraint: $T_{\mu\nu}^{(+)} + T_{\mu\nu}^{(-)} = 0$ (dipolar cancellation at infinity).

\subsubsection{Connection to Friedmann Equations and Dark Energy Equation of State}

The deviation $\epsilon(z) = 0.05 \cdot (1 - \MCz)$ arises because: (1)~emergence is not instantaneous; (2)~coupling between poles is not perfectly symmetric at intermediate stages; (3)~residual imbalance allows partial oscillation. At late times ($z \to 0$), observed $w$ approaches $-1$ asymptotically.

\subsubsection{Antigravity and the Information Tensor}

The curvature density $\Kgen = \nabla \cdot (J \otimes F)$ depends on the flow and force of information. In the $+1$ direction, information is compressed (gravity); in $-1$, dispersed (antigravity). Conservation $\nabla^\mu \Tinfo = 0$ ensures total information content remains constant across both poles.

\subsubsection{Three Concrete Falsification Tests for Antigravity}

\textbf{Test~1: Bloch Signature in CMB Polarization.} The $T \times E$ cross-correlation should show oscillatory pattern at $\ell \sim 10$--$50$ (Bloch wall width).

\textbf{Test~2: Riemann Eigenvalue Structure in DESI BAO Data.} The galaxy power spectrum should exhibit peaks and suppressions at wavenumbers matching Riemann zero spacing: prime-number-like harmonic spacing in $P(k)$ at $k \sim 0.01$--$0.1$ Mpc$^{-1}$.

\textbf{Test~3: Dipolar Cancellation in $w(z)$.} At $z = 1.5$, $w(1.5) \approx -0.920$ vs.\ $\Lambda$CDM's $w = -1.000$ exactly ($\Delta w \approx 0.08$). D-ND predicts monotonic increase in $w$ toward $-1$ as $z \to 0$.

\subsubsection{Observational Implications: Testing Antigravity}

\begin{enumerate}
    \item \textbf{Isotropic expansion}: D-ND predicts isotropy naturally from the structural symmetry of the dipole.
    \item \textbf{Absence of antigravity ``interactions''}: No deviations in solar system tests (E\"otv\"os experiments), consistent with current data.
    \item \textbf{Decay of dark energy in future aeons}: $\rhoLambda \to 0$ asymptotically ($\sim 10^{100}$ years), unlike eternal dark energy in $\Lambda$CDM.
\end{enumerate}

\subsection{Time as Emergence: Thermodynamic Irreversibility and the Dipolar Amplitude}
\label{sec:time-emergence}

\subsubsection{Time Does Not ``Function''---It Emerges from Irreversibility}

The D-ND framework proposes that time emerges as the measure of irreversible information processing. The Clausius inequality:
\begin{equation}\label{eq:clausius}
    \boxed{\oint \frac{\delta Q}{T} \leq 0}
\end{equation}

For real (irreversible) cycles, the integral is strictly negative. This residual loss creates the arrow of time. Time emerges as the integral of entropy production:
\begin{equation}\label{eq:time-emerge}
    \boxed{t = \int_0^T \frac{dS}{dT}(\tau) \, d\tau}
\end{equation}

The irreversibility $\oint dQ/T < 0$ guarantees $dS/dT > 0$, making time monotonic and forward-directed.

\subsubsection{Time Emergence from the Six-Phase Cognitive Pipeline}

The D-ND framework identifies temporal emergence through six phases:
\begin{itemize}
    \item \textbf{Phase~0: Indeterminacy} ($\Phi_0$ = Zero-point potentiality)
    \item \textbf{Phase~1: Symmetry Breaking} (via $\emerge$ emergence)
    \item \textbf{Phase~2: Divergence} (Alternative paths multiply)
    \item \textbf{Phase~3: Validation} (Stream-Guard pruning)
    \item \textbf{Phase~4: Collapse} (Morpheus guide)
    \item \textbf{Phase~5: Refinement} (KLI Injection, Axiom~P5)
    \item \textbf{Phase~6: Determinacy} (Manifest output)
\end{itemize}

The sequence Phase~0 $\to$ Phase~6 is itself temporal evolution. Time does not parametrize this process externally; it \emph{is} the ordering principle. Each phase advances through irreversible information processing, and the entropy gradient $\nabla S$ drives the transition forward.

\subsubsection{Time as Parameter Ordering Field-Collapse Phases}

In the cosmological context:
\begin{equation}\label{eq:local-time-ordering}
    \boxed{t(\mathbf{x}) = T_{\text{cycle}} \times f(M_C(\mathbf{x}), \dot{M}_C(\mathbf{x}))}
\end{equation}

\textbf{Formal Derivation from Friston's Free Energy Principle:}
\begin{equation}\label{eq:free-energy}
    F(\text{Phase } n) = -\ln p(\text{data}|n) + \text{KL}[\text{Prior}||\text{Posterior}]
\end{equation}

The rate of time flow is proportional to the rate of free energy reduction:
\begin{equation}\label{eq:time-rate}
    \frac{dt}{d\tau} = \left|\frac{dF}{d\tau}\right|
\end{equation}
formally stating that time flows fastest where the universe learns most rapidly.

\subsubsection{Time as Local Amplitude of the Dipolar Oscillation}

The local time at spacetime point $(\mathbf{x},t)$:
\begin{equation}\label{eq:local-time}
    \tau(\mathbf{x}) = \Lambda \cdot |M_C(\mathbf{x})| \cdot (1 - |M_C(\mathbf{x})|) \cdot T_{\text{cycle}}
\end{equation}

Time runs fastest at intermediate emergence ($M_C \approx 0.5$) and slowly at $M_C \approx 0$ or $M_C \approx 1$. The local times are like intrinsic spins---properties of the emergence state, not external parameters.

\subsubsection{The Included Third and Normalization of Excluded-Third Logic}

The D-ND framework generalizes the excluded-third (\emph{tertium non datur}):
\begin{equation}\label{eq:included-third}
    1_{\text{D-ND}} = (t = +1) + (t = -1) + (t = 0)_{\text{singularity}}
\end{equation}

This is analogous to the extension from $\mathbb{R}$ to $\mathbb{C}$. By including the third explicitly, D-ND resolves paradoxes arising from hidden asymmetries in excluded-third logic \cite{Lupasco1951,Nicolescu2002}.

\subsubsection{The Lagrangian of Observation and Minimal Latency}

\textbf{Principle of Minimal Latency}: Among all possible actualization pathways, nature selects those minimizing the integral of local latencies:
\begin{equation}\label{eq:minimal-latency}
    \mathcal{S}_{\text{observe}} = \int_{\text{path}} \tau(\mathbf{x}) \, d\mathcal{M}
\end{equation}

This naturally explains: (1)~why the universe expands (minimal latency for actualizing many modes); (2)~why gravity exists (minimizes local transition paths); (3)~why structure forms (clustering reduces total latency); (4)~why entropy increases (larger configuration space requires longer latencies).

\subsubsection{Convergence and Divergence Are Simultaneous: Zero Latency in Assonances}

Where the convergence pole ($t = +1$) and divergence pole ($t = -1$) oscillate perfectly in phase (``assonance''), latency vanishes: $\tau = 0$. This corresponds to maximal potentiality---precisely $\NT$. At cosmic cycle boundaries, time becomes undefined (latency $\to 0$), and the next cycle initiates from pure potentiality.

\subsubsection{The Double Pendulum as Physical Realization}

The double pendulum exhibits simultaneous bifurcation: local chaos constrained by a single global Lagrangian:
\begin{equation}\label{eq:double-pendulum}
    L = \frac{1}{2}m(\dot{x}_1^2 + \dot{y}_1^2 + \dot{x}_2^2 + \dot{y}_2^2) - mg(y_1 + y_2)
\end{equation}

If the universe is a cosmological double pendulum: (1)~locally, reality is chaotic (quantum mechanics); (2)~globally, deterministic (classical field equations); (3)~neither description is more fundamental.

\subsubsection{Convergence and Divergence in the Modified Friedmann Equations}

\textbf{Convergence} ($t = +1$): The $\Omega_m$ term dominates at early times. \textbf{Divergence} ($t = -1$): The $\rhoLambda(z)$ term dominates at late times. At intermediate times ($z \sim 1$): the two terms balance, producing a resonance in the expansion history.

\subsubsection{Observational Predictions: Time Emergence Signatures}

\begin{enumerate}
    \item \textbf{Anomalous age estimates at high redshift}: Extremely distant galaxies may appear older in proper time than in coordinate time.
    \item \textbf{Preferred scales in structure formation}: Discrete preferred scales from latency minimization---a ``quantization'' of cosmic structure.
    \item \textbf{Time-dependent gravitational constant}: $G(z) = G_0 [1 + \delta_G(1 - \MCz)]$, with $\delta_G \sim 10^{-3}$--$10^{-2}$.
\end{enumerate}

\subsection{Observational Predictions Summary Table}
\label{sec:prediction-table}

Table~\ref{tab:predictions} consolidates all testable predictions across multiple observational domains.

\begin{longtable}{p{2.5cm}p{3cm}p{3cm}p{2.5cm}p{2.5cm}}
\caption{Comprehensive observational predictions: D-ND vs.\ $\Lambda$CDM and alternatives.}
\label{tab:predictions} \\
\toprule
\textbf{Observable} & \textbf{D-ND Prediction} & \textbf{$\Lambda$CDM} & \textbf{Distinguish.} & \textbf{Status} \\
\midrule
\endfirsthead
\toprule
\textbf{Observable} & \textbf{D-ND Prediction} & \textbf{$\Lambda$CDM} & \textbf{Distinguish.} & \textbf{Status} \\
\midrule
\endhead
\bottomrule
\endfoot

Tensor/scalar $r$ & $0.001$--$0.01$ & $0.001$--$0.1$ & Marginal & Planck: $r<0.064$ \\
\addlinespace
Bispectrum $\fNL$ & $5$--$20$ (smooth $\emerge$); higher in emergence templates & $\sim 1$--$5$ & Strong (3--5$\sigma$) with S4 & $\fNL^{\rm eq} < 25$ \\
\addlinespace
Power suppression & $10$--$20\%$ deficit at $\ell<10$ & Smooth power law & Possible (1--2$\sigma$) & Planck hint \\
\addlinespace
Spectral running & $dn_s/d\ln k \sim -0.005$ to $-0.02$ & $\sim 0$ & Possible (2--3$\sigma$) & Consistent with 0 \\
\addlinespace
CMB $T\times E$ & Oscillations at $\ell\sim 10$--$50$ & Smooth & Distinctive & Planck hints \\
\addlinespace
Growth $f(a)$ & $f_{\rm GR}[1+0.1(1-M_C)]$ & $f_{\rm GR}$ exact & Small (1--2$\sigma$) & GR consistent \\
\addlinespace
Halo bias & Enhanced at $z>1$ & Standard & Possible (2--3$\sigma$) & Standard consistent \\
\addlinespace
$\sigma_8$ & $\sim 0.80$ & $\approx 0.811$ & Marginal & Tension exists \\
\addlinespace
$w(z)$ & $-1+0.05(1-M_C(z))$ & $-1.000$ & Strong (2--4$\sigma$) & \textbf{DESI Year-2/3 decisive} \\
\addlinespace
BAO scale & $d_A^{\rm DND}(z\!=\!1) \approx 1.016 \times d_A^{\Lambda}$ & Standard & Possible (2--3$\sigma$) & DESI Year-3 \\
\addlinespace
Riemann signature & Prime-like spacing in $P(k)$ & No structure & Distinctive & Requires analysis \\
\addlinespace
$G$ variation & $\Delta G/G \sim 10^{-3}$--$10^{-2}$ & Constant & Small (1--2$\sigma$) & Pulsar timing \\
\addlinespace
Cyclic coherence & Low-$\ell$ correlations ($\ell\sim 1$--$3$) & No signal & Distinctive & Inconclusive \\
\end{longtable}

\textbf{Tier~1---Decisive Tests} (3--5$\sigma$): (1)~Dark energy $w(z)$ from DESI BAO; (2)~CMB $\fNL$ from CMB-S4; (3)~Riemann eigenvalue structure.

\textbf{Tier~2---Promising} (1--3$\sigma$): (4)~Spectral index running; (5)~Bloch wall CMB polarization; (6)~Halo bias evolution.

\textbf{Tier~3---Indirect/Long-Term}: (7)~$G$ variation; (8)~GW stochastic background; (9)~Cyclic coherence/Hawking points.

%==============================================================================
\section{Discussion and Conclusions}
\label{sec:discussion}
%==============================================================================

\subsection{Strengths of the D-ND Cosmological Extension}
\label{sec:strengths}

\begin{enumerate}
    \item \textbf{Closes a gap in cosmological theory}: Provides a mechanism for closed-system emergence of classical spacetime from quantum potentiality.
    \item \textbf{Connects micro and macro}: Links quantum emergence (Paper~A) to cosmic inflation and dark energy through a unified framework.
    \item \textbf{Resolves the initial singularity}: Replaces the Big Bang singularity with a finite boundary condition on emergence.
    \item \textbf{Addresses the dark energy problem}: Qualitative explanation for the small cosmological constant without fine-tuning.
    \item \textbf{Cyclic structure and information conservation}: Quantum information preserved across cosmic cycles.
    \item \textbf{Falsifiable predictions}: Concrete observational tests with quantitative criteria.
    \item \textbf{DESI-constrained framework}: Testable against 2024 BAO data with clear falsification criteria.
\end{enumerate}

\subsection{Limitations and Caveats}
\label{sec:limitations}

\begin{enumerate}
    \item \textbf{Speculative nature}: The connection between microscopic emergence and cosmic scales is not rigorously derived from first principles.
    \item \textbf{Lack of precision in emergence operator}: At cosmological scales, the structure of $\emerge$ and the spectrum of the ``cosmological Hamiltonian'' are not known.
    \item \textbf{Incomplete quantum gravity}: The framework does not provide a full quantum theory of gravity comparable to LQC or string cosmology.
    \item \textbf{Modified equations axiomatically motivated but not independently derived}: The tensor $\Tinfo$ follows from D-ND axioms P0--P4 (\secref{derivation-P4}), but a fully independent derivation from quantum gravity first principles remains an open problem.
    \item \textbf{Relation to observations unclear in detail}: Predictions require detailed computation (e.g., modified CAMB/CLASS codes) for quantitative precision.
    \item \textbf{Cosmological constant reassessment}: The identification of dark energy with residual $V_0$ is attractive but speculative.
\end{enumerate}

\subsection{Speculative but Falsifiable Framework}
\label{sec:falsifiable}

The predictions are: not derived from first principles but arising from extrapolating the quantum D-ND framework; testable in principle through specific CMB anomalies, structure patterns, and dark energy evolution; distinguished from $\Lambda$CDM in regimes where emergence effects are non-negligible.

\subsection{Paths Forward}
\label{sec:paths}

\textbf{Numerical Cosmology}: Implement a modified Boltzmann code (extending CLASS or CAMB) incorporating D-ND modifications.

\textbf{Quantum Gravity Integration}: Derive the modified Einstein equations from more fundamental principles (loop quantum cosmology, asymptotic safety, spectral action principle).

\textbf{Observational Campaigns}: Design dedicated observations for CMB bispectrum, high-redshift structure growth, and dark energy precision.

\subsection{Conclusion}
\label{sec:conclusion}

We have presented a speculative but mathematically coherent extension of the Dual-Non-Dual framework to cosmological scales. By coupling Einstein's field equations to the quantum emergence measure $\MC$, we sketch a picture in which: the universe emerges from primordial potentiality, inflation arises as a phase of rapid actualization, dark energy represents residual non-relational structure, and the initial singularity is replaced by a boundary condition on emergence. The framework suggests multiple cycles, each preserving quantum information through $\OmegaNT = 2\pi i$.

While highly speculative and dependent on assumptions about the microscopic emergence operator, the framework provides a conceptually unified view of quantum and classical cosmology. Whether it correctly captures the physics can only be determined through observational tests of its quantitative predictions.

\subsection{Comparative Predictions: D-ND vs.\ $\Lambda$CDM vs.\ LQC vs.\ CCC}
\label{sec:comparison}

Table~\ref{tab:comparison} provides a detailed comparison across key observables and theoretical properties.

\begin{longtable}{p{2.2cm}p{2.2cm}p{2.8cm}p{2.5cm}p{2.5cm}}
\caption{Comparative predictions across cosmological frameworks.}
\label{tab:comparison} \\
\toprule
\textbf{Feature} & \textbf{$\Lambda$CDM} & \textbf{D-ND} & \textbf{LQC} & \textbf{CCC} \\
\midrule
\endfirsthead
\toprule
\textbf{Feature} & \textbf{$\Lambda$CDM} & \textbf{D-ND} & \textbf{LQC} & \textbf{CCC} \\
\midrule
\endhead
\bottomrule
\endfoot

Singularity & Curvature divergence & NT singularity (finite) & Quantum bounce & Conformal rescaling \\
\addlinespace
Mechanism & Classical GR + $\Lambda$ & $\MC$ + info tensor & Quantum geometry & Weyl curvature \\
\addlinespace
Inflation & Slow-roll $\phi$ & Rapid $M_C$ evolution & Modified potential & Not primary \\
\addlinespace
Dark energy & $w=-1$ exact & $w=-1+0.05(1-M_C)$ & Slight loop corrections & Cyclic \\
\addlinespace
$\fNL$ & $\sim 1$ & $5$--$20$ (smooth $\emerge$) & Enhanced & Modified \\
\addlinespace
Information & Lost (Hawking) & Preserved (cycles) & Preserved (geometry) & Preserved (conformal) \\
\addlinespace
Cycles & None & $\OmegaNT = 2\pi i$ & Quantum bounce & Infinite aeons \\
\addlinespace
Free params & 6 & $\sim 8$ & $\sim 6$ & $\sim 5$ \\
\addlinespace
Status & Well-tested & Speculative; testable & Quantitative; debated & Speculative \\
\end{longtable}

\textbf{Key Distinctions}: (1)~Inflation mechanism differs across all four; (2)~Dark energy is constant in $\Lambda$CDM, evolving in D-ND; (3)~Information preservation differs fundamentally; (4)~DESI 2024--2026 data provides decisive constraints; (5)~D-ND uniquely connects emergence at quantum and cosmic scales.

%==============================================================================
% REFERENCES
%==============================================================================

\begin{thebibliography}{30}

\bibitem{Guth1981}
A.~H.~Guth,
``Inflationary universe: A possible solution to the horizon and flatness problems,''
\textit{Phys.\ Rev.\ D} \textbf{23}, 347 (1981).

\bibitem{Linde1986}
A.~D.~Linde,
``Eternally existing self-reproducing chaotic inflationary universe,''
\textit{Phys.\ Lett.\ B} \textbf{175}, 395 (1986).

\bibitem{Verlinde2011}
E.~Verlinde,
``On the origin of gravity and the laws of Newton,''
\textit{JHEP} \textbf{2011}(4), 29.
[arXiv:1001.0785]

\bibitem{Verlinde2016}
E.~Verlinde,
``Emergent gravity and the dark universe,''
\textit{SciPost Phys.} \textbf{2}(3), 016 (2016).
[arXiv:1611.02269]

\bibitem{RyuTakayanagi2006}
S.~Ryu and T.~Takayanagi,
``Holographic derivation of entanglement entropy from AdS/CFT,''
\textit{Phys.\ Rev.\ Lett.} \textbf{96}, 181602 (2006).

\bibitem{HartleHawking1983}
J.~B.~Hartle and S.~W.~Hawking,
``Wave function of the universe,''
\textit{Phys.\ Rev.\ D} \textbf{28}, 2960 (1983).

\bibitem{Wheeler1968}
J.~A.~Wheeler,
``Superspace and the nature of quantum geometrodynamics,''
in \textit{Battelle Rencontres}, pp.~242--307 (1968).

\bibitem{Kuchar1992}
K.~V.~Kucha\v{r},
``Time and interpretations of quantum gravity,''
in \textit{General Relativity and Gravitation}, pp.~520--575 (Cambridge University Press, 1992).

\bibitem{Giovannetti2015}
V.~Giovannetti, S.~Lloyd, and L.~Maccone,
``Quantum time,''
\textit{Phys.\ Rev.\ D} \textbf{92}, 045033 (2015).

\bibitem{Penrose2005}
R.~Penrose,
``Before the Big Bang?''
in \textit{Science and Ultimate Reality}, pp.~1--29 (Cambridge University Press, 2005).

\bibitem{Penrose2010}
R.~Penrose,
\textit{Cycles of Time: An Extraordinary New View of the Universe}
(Jonathan Cape, 2010).

\bibitem{Wehus2021}
A.~M.~Wehus and H.~K.~Eriksen,
``A search for concentric circles in the 7-year WMAP temperature sky maps,''
\textit{Astrophys.\ J.} \textbf{733}, 29 (2021).

\bibitem{Maldacena1998}
J.~M.~Maldacena,
``The large N limit of superconformal field theories and supergravity,''
\textit{Adv.\ Theor.\ Math.\ Phys.} \textbf{2}, 231 (1998).

\bibitem{VanRaamsdonk2010}
M.~Van~Raamsdonk,
``Building up spacetime with quantum entanglement,''
\textit{Gen.\ Relativ.\ Gravit.} \textbf{42}, 2323 (2010).

\bibitem{Planck2018NG}
Planck Collaboration,
``Planck 2018 results. IX. Constraints on primordial non-Gaussianity,''
\textit{Astron.\ Astrophys.} \textbf{641}, A9 (2018).

\bibitem{Komatsu2010}
E.~Komatsu,
``Hunting for primordial non-Gaussianity in the CMB,''
\textit{Class.\ Quantum Grav.} \textbf{27}, 124010 (2010).

\bibitem{Maldacena2003}
J.~M.~Maldacena,
``Non-Gaussian features of primordial fluctuations in single-field inflationary models,''
\textit{JHEP} \textbf{2003}(05), 013.

\bibitem{Dodelson2003}
S.~Dodelson,
\textit{Modern Cosmology}
(Academic Press, 2003).

\bibitem{Perlmutter1999}
S.~Perlmutter et al.,
``Measurements of $\Omega$ and $\Lambda$ from 42 high-redshift supernovae,''
\textit{Astrophys.\ J.} \textbf{517}, 565 (1999).

\bibitem{Riess1998}
A.~G.~Riess et al.,
``Observational evidence from supernovae for an accelerating universe and a cosmological constant,''
\textit{Astron.\ J.} \textbf{116}, 1009 (1998).

\bibitem{Weinberg2000}
S.~Weinberg,
``The cosmological constant problems,''
arXiv:astro-ph/0005265 (2000).

\bibitem{Bekenstein1973}
J.~D.~Bekenstein,
``Black holes and entropy,''
\textit{Phys.\ Rev.\ D} \textbf{7}, 2333 (1973).

\bibitem{Hawking1974}
S.~W.~Hawking,
``Black hole explosions?''
\textit{Nature} \textbf{248}, 30 (1974).

\bibitem{tHooft1993}
G.~'t~Hooft,
``Dimensional reduction in quantum gravity,''
arXiv:gr-qc/9310026 (1993).

\bibitem{Reed1980}
M.~Reed and B.~Simon,
\textit{Methods of Modern Mathematical Physics}
(Academic Press, 1980).

\bibitem{Chamseddine1997}
A.~H.~Chamseddine and A.~Connes,
``The spectral action principle,''
\textit{Commun.\ Math.\ Phys.} \textbf{186}, 731 (1997).

\bibitem{Bardeen1986}
J.~M.~Bardeen, J.~R.~Bond, N.~Kaiser, and A.~S.~Szalay,
``The statistics of peaks of Gaussian random fields,''
\textit{Astrophys.\ J.} \textbf{304}, 15 (1986).

\bibitem{Beke2021}
L.~Beke and K.~Hinterbichler,
``Entropic gravity and the limits of thermodynamic descriptions,''
\textit{Phys.\ Lett.\ B} \textbf{811}, 135863 (2021).

\bibitem{Lupasco1951}
S.~Lupasco,
\textit{Le principe d'antagonisme et la logique de l'\'energie}
(Hermann, Paris, 1951).

\bibitem{Nicolescu2002}
B.~Nicolescu,
\textit{Manifesto of Transdisciplinarity}
(SUNY Press, 2002).

\end{thebibliography}

\end{document}
