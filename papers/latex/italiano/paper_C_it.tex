%==============================================================================
% PAPER C - GEOMETRIA DELL'INFORMAZIONE E CONNESSIONE ZETA
% Traduzione italiana — per lettura personale
% Originale: paper_C.tex (Journal of Mathematical Physics / Commun. Math. Phys.)
%==============================================================================

\documentclass[aps,prl,11pt,notitlepage,nofootinbib,longbibliography]{revtex4-2}

%==============================================================================
% PACKAGES
%==============================================================================

\usepackage[utf8]{inputenc}
\usepackage[T1]{fontenc}
\usepackage[italian]{babel}
\usepackage{amsmath}
\usepackage{amssymb}
\usepackage{mathrsfs}
\usepackage{braket}
\usepackage{amsthm}
\usepackage{hyperref}
\usepackage{cleveref}
% natbib loaded by revtex4-2 automatically
\usepackage{geometry}
\usepackage{setspace}
\usepackage{graphicx}
\usepackage{float}
\usepackage{booktabs}
\usepackage{dnd_shared}

%==============================================================================
% HYPERREF CONFIGURATION
%==============================================================================

\hypersetup{
    colorlinks=true,
    linkcolor=blue,
    citecolor=blue,
    urlcolor=blue,
    bookmarksnumbered=true,
    pdftitle={Geometria dell'informazione e struttura numero-teorica nel framework D-ND},
    pdfauthor={D-ND Research Collective},
    pdfsubject={Geometria dell'informazione, Zeta di Riemann, Carica topologica, Emergenza},
    pdfkeywords={geometria dell'informazione, zeta di Riemann, carica topologica, emergenza, Berry-Keating, Gauss-Bonnet, metrica di Fisher}
}

%==============================================================================
% GEOMETRY
%==============================================================================

\geometry{
    margin=1in,
    includehead=true,
    includefoot=true
}

%==============================================================================
% CUSTOM MACROS FOR PAPER C
%==============================================================================

% Curvature and topology
\newcommand{\Kc}{K_c}
\newcommand{\Pposs}{P_{\text{poss}}}
\newcommand{\Llat}{L_{\text{lat}}}
\newcommand{\xicomplete}{\xi}

% Hilbert space
\newcommand{\Hilbert}{\mathcal{H}}

% Partial derivatives
\newcommand{\pd}[2]{\frac{\partial #1}{\partial #2}}
\newcommand{\pdd}[3]{\frac{\partial^2 #1}{\partial #2 \partial #3}}

% Conjecture environment (not in dnd_shared.sty)
\theoremstyle{plain}
\newtheorem{conjecture}[theorem]{Congettura}

%==============================================================================
% BEGIN DOCUMENT
%==============================================================================

\begin{document}

%==============================================================================
% TITLE AND METADATA
%==============================================================================

\title{Geometria dell'informazione e struttura numero-teorica\\nel framework D-ND}

\author{D-ND Research Collective}
\affiliation{Ricerca indipendente}

\date{\today}

%==============================================================================
% ABSTRACT
%==============================================================================

\begin{abstract}

Stabiliamo una connessione tra la curvatura informazionale del framework di emergenza Duale-Non-Duale (D-ND) e gli zeri della funzione zeta di Riemann. Definiamo una curvatura informazionale generalizzata $\Kgen(x,t) = \nabla_M \cdot (J(x,t) \otimes F(x,t))$ sul landscape di emergenza, dove $J$ rappresenta il flusso informazionale e $F$ denota il campo di forze generalizzato. \textbf{La congettura centrale} è che i valori critici di questa curvatura corrispondano agli zeri della zeta di Riemann sulla retta critica: $\Kgen(x,t) = \Kc \Leftrightarrow \zeta(1/2 + it) = 0$. Costruiamo una carica topologica $\chiDND = (1/2\pi)\oint_M \Kgen \, dA$ (un invariante di tipo Gauss--Bonnet), congetturiamo che sia quantizzata ($\chiDND \in \mathbb{Z}$), e la connettiamo alla coerenza ciclica $\OmegaNT = 2\pi i$. Deriviamo la funzione zeta di Riemann come somma spettrale sugli autovalori di emergenza e stabiliamo corrispondenze strutturali con la congettura di Berry--Keating. Testando contro i primi 100 zeri di Riemann verificati su tre distinti spettri dell'operatore di emergenza, troviamo che la correlazione curvatura-zeta emerge in modo forte ed esclusivo sotto spaziatura logaritmica degli autovalori (Pearson $r = 0.921$, $p \approx 10^{-42}$), coerentemente con l'ipotesi spettrale di Berry--Keating. Un'analisi complementare del gap spettrale rivela che la spaziatura lineare degli autovalori riproduce al meglio la statistica locale dei gap (KS $= 0.152$, $p = 0.405$), suggerendo una struttura a due scale. Verifichiamo numericamente la quantizzazione di $\chiDND$ sul landscape di emergenza D-ND, e specifichiamo condizioni matematiche precise che proverebbero o confuterebbero definitivamente la connessione.

\end{abstract}

%==============================================================================
% SEZIONE 1: INTRODUZIONE
%==============================================================================

\section{Introduzione}
\label{sec:introduction}

\subsection{Geometria dell'informazione in fisica}
\label{subsec:info_geo}

La geometria dell'informazione~\cite{amari2016,amari2007} studia la struttura geometrico-differenziale delle distribuzioni di probabilità. La metrica di informazione di Fisher,
\begin{equation}
\label{eq:fisher_metric}
g_{ij} = \int \pd{\ln p(x|\theta)}{\theta_i} \pd{\ln p(x|\theta)}{\theta_j} p(x|\theta) \, dx,
\end{equation}
definisce una geometria riemanniana sullo spazio delle distribuzioni di probabilità. La curvatura della geometria dell'informazione misura la non-linearità di una famiglia di modelli.

La geometria si è rivelata fondamentale in fisica: la curvatura dello spaziotempo codifica la gravità~\cite{einstein1915}, la curvatura di gauge determina le forze nucleari~\cite{yangmills1954}, l'hessiano dell'entropia definisce la stabilità~\cite{balian2007}, e la metrica di Fisher governa la criticalità quantistica~\cite{zanardi2006}. Sorge una domanda naturale: \emph{La curvatura di un landscape di emergenza può essere connessa a strutture fondamentali della teoria dei numeri?}

\subsection{Teoria dei numeri e meccanica quantistica}
\label{subsec:number_theory}

L'ipotesi di Riemann~\cite{riemann1859} asserisce che tutti gli zeri non banali di $\zeta(s) = \sum_{n=1}^\infty n^{-s}$ giacciono sulla retta critica $\text{Re}(s) = 1/2$. La verifica numerica si estende a trilioni di zeri~\cite{platt2021}, ma una dimostrazione resta elusiva.

Berry e Keating~\cite{berry1999,berry2008} hanno congetturato che gli zeri corrispondano ad autovalori di un hamiltoniano quantistico sconosciuto
\begin{equation}
\label{eq:berry_keating}
\hat{H}_{\text{BK}} = \frac{1}{2}\left(\hat{p} \ln \hat{x} + \ln \hat{x} \, \hat{p}\right) + \text{correzioni}.
\end{equation}
Connes~\cite{connes1999} ha proposto un approccio di geometria non commutativa in cui lo spettro codifica gli zeri di Riemann. Sierra e Townsend~\cite{sierra2011} hanno connesso questo all'AdS/CFT.

La nostra proposta collega questi framework: l'operatore di emergenza $\emerge$ e la sua curvatura $\Kgen$ codificano dati spettrali che corrispondono agli zeri della zeta.

\subsection{La connessione D-ND}
\label{subsec:dnd_connection}

Dalla fondazione D-ND (Paper~A), l'operatore di curvatura è:
\begin{equation}
\label{eq:curvature_operator}
C = \int d^4x \, \Kgen(x,t) \ket{x}\bra{x}
\end{equation}
dove $\Kgen(x,t) = \nabla \cdot (J(x,t) \otimes F(x,t))$ è la curvatura informazionale generalizzata.

\textbf{Congettura centrale}: i valori critici di $\Kgen$ (dove $\Kgen = \Kc$) corrispondono a transizioni di fase nel landscape di emergenza che si allineano con gli zeri di $\zeta$ sulla retta critica.

\subsection{Contributi}
\label{subsec:contributions}

\begin{enumerate}
    \item Definizione rigorosa di $\Kgen$ e sua relazione con la metrica di Fisher e la curvatura di Ricci.
    \item Formulazione della congettura D-ND/zeta: $\Kgen(x,t) = \Kc \Leftrightarrow \zeta(1/2 + it) = 0$.
    \item Classificazione topologica tramite la carica topologica di Gauss--Bonnet $\chiDND$, con calcolo esplicito in 2D.
    \item Interpretazione spettrale: derivazione di $\zeta$ dai dati spettrali D-ND.
    \item Connessione di $\OmegaNT = 2\pi i$ al numero di avvolgimento.
    \item Struttura a curva ellittica degli stati di emergenza stabili.
    \item Evidenza numerica da tre test computazionali contro i primi 100 zeri della zeta, che rivela una struttura a due scale.
    \item Criteri espliciti di falsificabilità.
\end{enumerate}

%==============================================================================
% SEZIONE 2: CURVATURA INFORMAZIONALE
%==============================================================================

\section{Curvatura informazionale nel framework D-ND}
\label{sec:curvature}

\subsection{Curvatura informazionale generalizzata}
\label{subsec:kgen_def}

Sia $M$ il landscape di emergenza---una varietà liscia parametrizzata dallo spazio delle configurazioni e dal tempo. Definiamo:

\textbf{Flusso informazionale}: la corrente di probabilità
\begin{equation}
\label{eq:info_flow}
J(x,t) = \text{Im}\left[\psi^*(x,t) \nabla \psi(x,t)\right]
\end{equation}

\textbf{Campo di forze generalizzato}: il gradiente del potenziale efficace
\begin{equation}
\label{eq:force_field}
F(x,t) = -\nabla V_{\text{eff}}(x,t) - \frac{\hbar^2}{2m}\nabla(\log\rho(x,t))
\end{equation}

\textbf{Curvatura informazionale generalizzata}:
\begin{equation}
\label{eq:kgen}
\Kgen(x,t) = \nabla_M \cdot (J(x,t) \otimes F(x,t))
\end{equation}

In rappresentazione di coordinate con metrica $g$:
\begin{equation}
\label{eq:kgen_coords}
\Kgen = \nabla_\mu (J^\mu F^\nu g_{\nu\alpha} n^\alpha) = \frac{1}{\sqrt{g}} \partial_\mu \left(\sqrt{g} \, (J \otimes F)^{\mu}{}_{\nu} n^\nu \right)
\end{equation}
dove $n^\nu$ è la normale unitaria agli insiemi di livello del potenziale di emergenza. Nel caso semplificato 1D, questo si riduce a $\Kgen = \partial_x(J \cdot F)$.

\subsection{Relazione con la metrica di Fisher e la curvatura di Ricci}
\label{subsec:fisher_relation}

La metrica di informazione di Fisher su $\{p(x|\theta)\}$ è:
\begin{equation}
g_{ij}(\theta) = \mathbb{E}_{p}\left[\pd{\ln p}{\theta_i} \pd{\ln p}{\theta_j}\right]
\end{equation}

\begin{proposition}[Informale]
\label{prop:kgen_ricci}
La curvatura informazionale generalizzata $\Kgen$ è legata alla curvatura di Ricci della metrica di Fisher da $\Kgen = \mathcal{R} + \text{(termini di drift geometrico)}$ per una scelta opportuna della metrica su $M$.
\end{proposition}

\subsection{$\Kgen$ come generalizzazione della curvatura di Fisher}
\label{subsec:kgen_generalization}

\begin{proposition}[Generalizzazione di $\Kgen$]
\label{prop:kgen_gen}
La curvatura informazionale generalizzata $\Kgen$ estende la curvatura indotta dalla metrica di Fisher $\mathcal{R}_F$ all'intero landscape di emergenza:
\begin{equation}
\label{eq:kgen_unified}
\Kgen = \mathcal{R}_F + \frac{1}{Z} \nabla \cdot (J \otimes F)
\end{equation}
dove $Z$ è una costante di normalizzazione che garantisce la coerenza dimensionale.
\end{proposition}

%==============================================================================
% SEZIONE 3: CLASSIFICAZIONE TOPOLOGICA
%==============================================================================

\section{Classificazione topologica via Gauss--Bonnet}
\label{sec:topology}

\subsection{Carica topologica come integrale della curvatura}
\label{subsec:topo_charge}

Definiamo la carica topologica D-ND:
\begin{equation}
\label{eq:chi_dnd}
\chiDND = \frac{1}{2\pi} \oint_{\partial M} \Kgen \, dA
\end{equation}

Questa è una formula di tipo Gauss--Bonnet. Il teorema classico di Gauss--Bonnet afferma: per una varietà riemanniana compatta bidimensionale $M$ senza bordo,
\begin{equation}
\label{eq:gauss_bonnet}
\int_M K \, dA = 2\pi \chi(M)
\end{equation}
dove $K$ è la curvatura gaussiana e $\chi(M)$ è la caratteristica di Eulero.

\subsection{Quantizzazione: $\chiDND \in \mathbb{Z}$}
\label{subsec:quantization}

\begin{conjecture}[Quantizzazione topologica]
\label{conj:quantization}
Se $\Kgen$ deriva dall'operatore di emergenza $\emerge$ con spettro discreto $\{\lambda_k\}$, allora $\chiDND \in \mathbb{Z}$.
\end{conjecture}

\textit{Motivazione}: per il teorema dell'indice di Atiyah--Singer~\cite{atiyah1963}, la carica totale è:
\begin{equation}
\chiDND = \sum_{k=1}^M n_k
\end{equation}
dove $n_k$ è il grado topologico associato all'autovalore $\lambda_k$.

\subsection{Calcolo esplicito in 2D}
\label{subsec:2d_computation}

Abbiamo calcolato $\chiDND$ sul landscape di emergenza D-ND a doppia buca $V(Z) = Z^2(1-Z)^2 + \lambda \theta_{\text{NT}} Z(1-Z)$ su una griglia $200 \times 200$, con accoppiamento $\lambda \in [0.1, 0.9]$.

Risultati (Figg.~\ref{fig:C7}--\ref{fig:C8}):
\begin{itemize}
    \item $\chiDND$ resta entro $0.043$ dall'intero $0$ su tutti i 100 passi temporali.
    \item Il 100\% dei campioni rientra nella distanza $0.1$ da un intero.
    \item Distanza media dall'intero più vicino: $0.027$.
\end{itemize}

L'integrale di volume quasi nullo indica una distribuzione simmetrica della curvatura (le regioni positive e negative si cancellano), coerente con un landscape ricco di selle prodotto dal potenziale a doppia buca.

\subsection{Estensione a dimensioni superiori}
\label{subsec:higher_dim}

Il teorema di Chern--Gauss--Bonnet si applica a varietà compatte di dimensione pari. Per varietà di dimensione dispari (inclusa la 3D), la caratteristica di Eulero via Gauss--Bonnet è identicamente nulla.

Per una varietà di emergenza 4D $M_4$:
\begin{equation}
\chi(M_4) = \frac{1}{32\pi^2} \int_{M_4} \left(|W|^2 - 2|E|^2 + \frac{R^2}{6}\right) \sqrt{g} \, d^4x
\end{equation}
dove $W$ è il tensore di Weyl, $E$ il tensore di Ricci a traccia nulla, e $R$ la curvatura scalare.

In alternativa, per una varietà 3D parametrizzata da $(x, y, t)$, si studia la famiglia di sezioni 2D $M_2(t)$ e si traccia $\chiDND(t)$ come funzione di $t$. Le discontinuità in $\chiDND(t)$ segnalano biforcazioni topologiche.

\subsection{Coerenza ciclica e numero di avvolgimento}
\label{subsec:winding}

La coerenza ciclica $\OmegaNT = 2\pi i$ si connette al numero di avvolgimento:
\begin{equation}
\label{eq:winding}
w = \frac{1}{2\pi i} \oint_C d(\ln f(z))
\end{equation}

La coerenza ciclica eguaglia il numero di avvolgimento di $\zeta$ attorno all'origine, connettendo: (1)~la struttura topologica $\chiDND$, (2)~il comportamento di avvolgimento di $\zeta$, e (3)~la fase quantistica $\OmegaNT$.

%==============================================================================
% SEZIONE 4: LA CONNESSIONE ZETA
%==============================================================================

\section{La connessione zeta}
\label{sec:zeta}

\subsection{Formulazione spettrale}
\label{subsec:spectral}

L'operatore di emergenza $\emerge = \sum_{k=1}^M \lambda_k \ket{e_k}\bra{e_k}$ con $\lambda_k \in [0,1]$ ammette una rappresentazione spettrale connessa a $\zeta$:
\begin{equation}
\label{eq:formula_a6}
\zeta(s) \approx \int \left(\rho(x) e^{-sx} + \Kgen\right) dx
\end{equation}
dove $\rho(x)$ è una densità possibilistica e $\Kgen$ è la curvatura.

\subsection{Congettura centrale}
\label{subsec:conjecture}

\begin{conjecture}[Connessione D-ND/Zeta]
\label{conj:zeta}
Per $t \in \mathbb{R}$:
\begin{equation}
\label{eq:conjecture}
\Kgen(x_c, t) = \Kc \Leftrightarrow \zeta(1/2 + it) = 0
\end{equation}
dove $x_c = x_c(t)$ è il punto spaziale di curvatura critica e $\Kc$ è la soglia critica.
\end{conjecture}

\textbf{Avvertenza sullo stato}: questa congettura è speculativa. L'operatore di emergenza $\emerge$ è fenomenologico (Paper~A), pertanto $\Kgen$ ne eredita l'indeterminatezza. Un test rigoroso richiede: (1)~una derivazione indipendente dai principi primi di $\emerge$, (2)~il calcolo numerico di $\Kgen$ su un dominio specificato, e (3)~un confronto pre-registrato con gli zeri noti della zeta.

\subsection{Argomento di coerenza strutturale}
\label{subsec:consistency}

Il framework D-ND è \emph{coerente} con l'ipotesi di Riemann:

\textbf{Allineamento delle simmetrie.} La simmetria dipolare D-ND (Assioma~1: $D(x,x') = D(x',x)$) si manifesta come $\mathcal{L}_R(t) = \mathcal{L}_R(-t)$. L'equazione funzionale $\xicomplete(s) = \xicomplete(1-s)$ ha la stessa struttura.

\textbf{Struttura estremale.} Sotto struttura spettrale logaritmica, i valori $|K_c^{(n)}|$ ai tempi degli zeri della zeta sono fortemente correlati con le posizioni degli zeri ($r = 0.921$).

\textbf{Zeri fuori dalla retta e rottura di simmetria.} Uno zero a $\sigma \neq 1/2$ romperebbe la simmetria $\xicomplete(s) = \xicomplete(1-s)$. Nel framework D-ND, questo corrisponderebbe alla violazione della simmetria dipolare. Questo argomento è \emph{condizionale} alla stessa corrispondenza D-ND/zeta.

\begin{remark}[Fondamenti logici]
\label{rem:logic}
Il framework D-ND opera con il \emph{terzo incluso} (cfr.\ Lupasco~\cite{lupasco1951}, Nicolescu~\cite{nicolescu2002}): stati contraddittori possono coesistere a diversi livelli di realtà. La matematica classica---inclusi Gauss--Bonnet, equazioni funzionali e teoria spettrale---opera sotto il terzo escluso. Questo articolo utilizza strumenti classici come linguaggio matematico, mentre il framework che descrive potrebbe richiedere una fondazione logica estesa. Dove emerge una tensione, la segnaliamo esplicitamente.
\end{remark}

\subsection{Confronto numerico con i primi 100 zeri della zeta}
\label{subsec:numerical}

Utilizzando mpmath (precisione a 30 cifre), abbiamo calcolato i primi 100 zeri non banali $\zeta(1/2 + it_n) = 0$, da $t_1 \approx 14.1347$ a $t_{100} \approx 236.5242$.

Abbiamo costruito un modello di emergenza a $N = 100$ livelli:
\begin{itemize}
    \item $\NT = (1/\sqrt{N}) \sum_{k=1}^{N} \ket{k}$
    \item $\emerge = \sum_k \lambda_k \ket{e_k}\bra{e_k}$ con tre pattern di autovalori: lineare ($\lambda_k = k/N$), primo ($\lambda_k \propto 1/p_k$), logaritmico ($\lambda_k = \log(k{+}1)/\log N$)
    \item $H = \text{diag}(2\pi \lambda_k)$, $R(t) = e^{-iHt} \emerge \NT$
\end{itemize}

\begin{table}[h]
\centering
\caption{Correlazione tra valori di curvatura critica e posizioni degli zeri della zeta.}
\label{tab:correlations}
\begin{tabular}{lcccc}
\toprule
Pattern & Pearson $r$ & $p$-value & Spearman $\rho$ & Monotonicità \\
\midrule
Lineare & $-0.233$ & $1.96 \times 10^{-2}$ & $-0.221$ & 54.5\% \\
Primo & $-0.030$ & $7.64 \times 10^{-1}$ & $-0.063$ & 49.5\% \\
\textbf{Logaritmico} & $\mathbf{0.921}$ & $\mathbf{5.6 \times 10^{-42}}$ & $\mathbf{0.891}$ & \textbf{76.8\%} \\
\bottomrule
\end{tabular}
\end{table}

La correlazione emerge in modo forte ed esclusivo sotto spaziatura logaritmica, corrispondente alla struttura hamiltoniana di Berry--Keating~\eqref{eq:berry_keating}. Ciò costituisce una conferma indipendente dell'ipotesi spettrale di Berry--Keating da un framework di geometria dell'informazione.

\subsection{Stime del gap spettrale}
\label{subsec:spectral_gaps}

Abbiamo calcolato gli autovalori dell'operatore di Laplace--Beltrami sulla varietà di emergenza con metrica di Fisher e potenziale a doppia buca:
\begin{equation}
H_{\text{emergence}} = \Delta_{\mathcal{M}} + V(Z)
\end{equation}

\begin{table}[h]
\centering
\caption{Test di Kolmogorov--Smirnov di confronto tra i gap spettrali e i gap degli zeri della zeta.}
\label{tab:ks_test}
\begin{tabular}{lccc}
\toprule
Pattern & Statistica KS & $p$-value & $\text{Var}(\Delta\lambda)$ \\
\midrule
\textbf{Lineare} & $\mathbf{0.152}$ & $\mathbf{0.405}$ & $0.250$ \\
Logaritmico & $0.281$ & $0.010$ & $0.650$ \\
Primo & $0.723$ & $< 10^{-6}$ & $6.755$ \\
\bottomrule
\end{tabular}
\end{table}

Un pattern complementare: gli spettri lineari riproducono al meglio la statistica dei gap (compatibile con GUE~\cite{berry1999}), gli spettri logaritmici codificano le posizioni globali. Questa struttura a due scale suggerisce che l'operatore di emergenza completo richieda un crossover da logaritmico a lineare.

\subsection{Autovalori di Laplace--Beltrami e Hilbert--P\'olya}
\label{subsec:hilbert_polya}

La congettura di Hilbert--P\'olya propone che gli zeri di Riemann corrispondano ad autovalori di un operatore autoaggiunto. Lo identifichiamo con l'operatore di Laplace--Beltrami sulla varietà di emergenza:
\begin{equation}
\Delta_{\mathcal{M}} \Phi = g^{\mu\nu} \nabla_\mu \nabla_\nu \Phi
\end{equation}

L'hamiltoniano di Berry--Keating è identificato con $\hat{H}_{\text{zeta}} = \Delta_{\mathcal{M}} + \text{(correzioni di curvatura)}$. Il processo di emergenza definisce la varietà; la geometria della varietà definisce l'operatore; lo spettro dell'operatore produce gli zeri della zeta.

\subsection{Relazioni di simmetria}
\label{subsec:symmetry}

La funzione zeta di Riemann soddisfa:
\begin{equation}
\label{eq:functional_eq}
\xicomplete(s) = \xicomplete(1-s), \quad \xicomplete(s) = \frac{1}{2} s(s-1) \pi^{-s/2} \Gamma(s/2) \zeta(s)
\end{equation}

La simmetria D-ND $\mathcal{L}_R(t) = \mathcal{L}_R(-t)$ ne è l'analogo informazionale: entrambe esprimono il principio per cui il sistema appare identico dai poli opposti di un dipolo.

%==============================================================================
% SEZIONE 5: DENSITÀ POSSIBILISTICA E CURVE ELLITTICHE
%==============================================================================

\section{Densità possibilistica e curve ellittiche}
\label{sec:elliptic}

\subsection{Struttura a curva ellittica}
\label{subsec:elliptic_structure}

Associamo al landscape di emergenza una famiglia di curve ellittiche:
\begin{equation}
\label{eq:elliptic_curve}
E_t: y^2 = x^3 - \frac{3}{2}\langle K \rangle(t) \cdot x + \frac{1}{3}\langle K^3 \rangle(t)
\end{equation}
con discriminante $\Delta \neq 0$.

Per il teorema di Mordell--Weil~\cite{silverman2009}, $E_t(\mathbb{Q}) \cong E_t(\mathbb{Q})_{\text{torsion}} \times \mathbb{Z}^r$ dove il rango $r$ misura i gradi di libertà negli stati razionali (classici).

\subsection{Densità possibilistica}
\label{subsec:possibilistic}

Definiamo:
\begin{equation}
\label{eq:possibilistic}
\rho(x,y,t) = |\braket{\psi_{x,y}}{\Psi}|^2
\end{equation}

Quando $(x,y)$ è un punto razionale, $\rho$ presenta tipicamente dei picchi---gli stati razionali sono più probabili. Questo connette la dinamica dell'emergenza all'aritmetica delle curve ellittiche.

\subsection{Chiusura NT e stabilità informazionale}
\label{subsec:closure}

L'emergenza stabile è caratterizzata da:
\begin{equation}
\label{eq:stability}
\oint_{NT} (\Kgen \cdot \Pposs - \Llat) \, dt = 0
\end{equation}

\begin{conjecture}[Chiusura NT]
\label{conj:closure}
Il continuum NT raggiunge la chiusura topologica se e solo se tre condizioni valgono simultaneamente:
\begin{enumerate}
    \item \textbf{La latenza si annulla}: $\Llat \to 0$.
    \item \textbf{Degenerazione ellittica}: $\Delta(t_c) \to 0$ (la curva $E_t$ acquisisce una singolarità).
    \item \textbf{Ortogonalità}: $\nabla_M \Kgen \cdot \nabla_M \Pposs = 0$.
\end{enumerate}
\end{conjecture}

Quando tutte e tre le condizioni valgono, l'integrale di contorno produce:
\begin{equation}
\label{eq:closure_integral}
\oint_{\text{NT}} \frac{\Kgen(Z) \cdot \Pposs(Z)}{Z} \, dZ = 2\pi i \cdot \text{Res}_{Z=0}[\Kgen \cdot \Pposs / Z]
\end{equation}

Per il teorema dei residui, quando le condizioni di chiusura normalizzano il residuo all'unità, questo produce $\OmegaNT = 2\pi i$---la stessa fase quantistica che appare nel numero di avvolgimento di $\zeta$.

%==============================================================================
% SEZIONE 6: DISCUSSIONE
%==============================================================================

\section{Percorsi verso la dimostrazione o la confutazione}
\label{sec:discussion}

\subsection{Cosa dimostrerebbe la congettura}
\label{subsec:proof}

\begin{enumerate}
    \item \textbf{Corrispondenza esatta}: mappa biiettiva $\Kgen(x_c(t), t) = \Kc \Leftrightarrow \zeta(1/2 + it) = 0$.
    \item \textbf{Identità spettrale}: lo spettro di $C$ eguaglia $\{t_n\}$.
    \item \textbf{Realizzazione hamiltoniana}: hamiltoniano esplicito $\hat{H}_{\text{emergence}}$ con autovalori che coincidono con $t_n$ con errore relativo $< 10^{-10}$.
    \item \textbf{Isomorfismo categoriale}: equivalenza tra landscape di emergenza e funzioni~L.
\end{enumerate}

\subsection{Cosa confuterebbe la congettura}
\label{subsec:disproof}

\begin{enumerate}
    \item Controesempio: $\zeta(1/2 + it_0) = 0$ ma nessuna curvatura critica a $t_0$.
    \item Fallimento della corrispondenza spettrale per modelli di emergenza espliciti.
    \item Incompatibilità topologica tra $\chiDND$ e le molteplicità degli zeri della zeta.
    \item Tassi di crescita incompatibili di $K_c^{(n)}$ rispetto a $t_n$.
\end{enumerate}

%==============================================================================
% SEZIONE 7: RELAZIONE CON BERRY-KEATING
%==============================================================================

\section{Relazione con la congettura di Berry--Keating}
\label{sec:berry_keating}

Il framework D-ND fornisce una candidata realizzazione fisica di Berry--Keating:

\begin{enumerate}
    \item \textbf{Identificazione geometrica}: l'operatore di curvatura $C$~\eqref{eq:curvature_operator} è un candidato naturale per $\hat{H}_{\text{zeta}}$.
    \item \textbf{Corrispondenza spettrale}: lo spettro di $C$ include i valori critici $\Kc$.
    \item \textbf{Fondamento fisico}: mentre Berry--Keating è astratto, il D-ND si connette all'emergenza fisica.
\end{enumerate}

\begin{table}[h]
\centering
\caption{Confronto tra gli approcci Berry--Keating e D-ND.}
\label{tab:comparison}
\begin{tabular}{lll}
\toprule
Aspetto & Berry--Keating & D-ND \\
\midrule
Hamiltoniano & Logaritmico astratto & Operatore di curvatura $C$ \\
Base & Spazio delle fasi classico & Landscape di emergenza \\
Connessione con la zeta & Assunta & Congetturata dalla curvatura \\
Falsificabilità & Limitata & Testabile (Tabella~\ref{tab:correlations}) \\
\bottomrule
\end{tabular}
\end{table}

%==============================================================================
% SEZIONE 8: CONCLUSIONI
%==============================================================================

\section{Conclusioni}
\label{sec:conclusions}

Questo articolo stabilisce un framework matematico che connette geometria dell'informazione, teoria dell'emergenza D-ND e funzione zeta di Riemann. Il risultato centrale è una \emph{congettura} secondo cui i valori critici della curvatura informazionale corrispondono agli zeri della zeta sulla retta critica.

Contributi principali:
\begin{enumerate}
    \item Definizione rigorosa di $\Kgen$ e sua derivazione dalla metrica di Fisher.
    \item Classificazione topologica via Gauss--Bonnet, con $\chiDND \in \mathbb{Z}$ verificata numericamente.
    \item Rappresentazione spettrale di $\zeta$ a partire dagli autovalori di emergenza.
    \item Evidenza numerica: gli spettri logaritmici codificano le posizioni degli zeri ($r = 0.921$), gli spettri lineari codificano la statistica dei gap (KS $= 0.152$).
    \item Criteri espliciti di falsificabilità.
\end{enumerate}

La connessione D-ND/zeta richiede una struttura spettrale specifica (logaritmica, compatibile con Berry--Keating) per manifestarsi. Questa selettività rafforza la congettura vincolandola.

Lavoro futuro: estensione a $N$ più elevati, derivazione dai principi primi dello spettro dell'operatore di emergenza, dimostrazioni rigorose tramite teorema dell'indice, indagine sulla struttura a due scale come firma di crossover.

%==============================================================================
% REFERENCES
%==============================================================================

\begin{thebibliography}{99}

\bibitem{amari2016} Amari, S., \emph{Information Geometry and Its Applications} (Springer, 2016).

\bibitem{amari2007} Amari, S. and Nagaoka, H., \emph{Methods of Information Geometry} (AMS, 2007).

\bibitem{zanardi2006} Zanardi, P. and Paunkov\'ic, N., Phys.\ Rev.\ E \textbf{74}, 031123 (2006).

\bibitem{balian2007} Balian, R., \emph{From Microphysics to Macrophysics}, Vol.~2 (Springer, 2007).

\bibitem{einstein1915} Einstein, A., Sitzungsber.\ Preuss.\ Akad.\ Wiss.\ Berlin, 844 (1915).

\bibitem{yangmills1954} Yang, C.~N. and Mills, R.~L., Phys.\ Rev.\ \textbf{96}, 191 (1954).

\bibitem{riemann1859} Riemann, B., Monatsber.\ K\"onigl.\ Preuss.\ Akad.\ Wiss.\ Berlin, 671 (1859).

\bibitem{titchmarsh1986} Titchmarsh, E.~C., \emph{The Theory of the Riemann Zeta-Function}, 2nd ed.\ (Oxford, 1986).

\bibitem{platt2021} Platt, D. and Robles, N., arXiv:2004.09765 [math.NT] (2021).

\bibitem{berry1999} Berry, M.~V. and Keating, J.~P., SIAM Rev.\ \textbf{41}, 236 (1999).

\bibitem{berry2008} Berry, M.~V. and Keating, J.~P., Proc.\ R.\ Soc.\ A \textbf{437}, 437 (2008).

\bibitem{connes1999} Connes, A., Selecta Math.\ \textbf{5}, 29 (1999).

\bibitem{sierra2011} Sierra, G. and Townsend, P.~K., J.\ High Energy Phys.\ \textbf{2011}(3), 91 (2011).

\bibitem{chamseddine1997} Chamseddine, A.~H. and Connes, A., Commun.\ Math.\ Phys.\ \textbf{186}, 731 (1997).

\bibitem{silverman2009} Silverman, J.~H., \emph{The Arithmetic of Elliptic Curves}, 2nd ed.\ (Springer, 2009).

\bibitem{atiyah1963} Atiyah, M.~F. and Singer, I.~M., Ann.\ Math.\ \textbf{87}, 484 (1963).

\bibitem{vanraamsdonk2010} Van Raamsdonk, M., Gen.\ Rel.\ Grav.\ \textbf{42}, 2323 (2010).

\bibitem{ryu2006} Ryu, S. and Takayanagi, T., Phys.\ Rev.\ Lett.\ \textbf{96}, 181602 (2006).

\bibitem{lupasco1951} Lupasco, S., \emph{Le principe d'antagonisme et la logique de l'\'energie} (Hermann, 1951).

\bibitem{nicolescu2002} Nicolescu, B., \emph{Manifesto of Transdisciplinarity} (SUNY Press, 2002).

\bibitem{priest2006} Priest, G., \emph{In Contradiction}, 2nd ed.\ (Oxford, 2006).

\bibitem{paperA} D-ND Research Collective, ``Quantum Emergence from Primordial Potentiality: The D-ND Framework,'' Draft~3.0 (2026).

\end{thebibliography}

%==============================================================================
% END DOCUMENT
%==============================================================================

\end{document}
