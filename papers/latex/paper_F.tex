%==============================================================================
% PAPER F - D-ND QUANTUM INFORMATION ENGINE
% Target Journal: Quantum / Physical Review A / Quantum Science and Technology
% Document Class: revtex4-2 (APS-compatible)
%==============================================================================

\documentclass[aps,pra,11pt,notitlepage,nofootinbib,longbibliography]{revtex4-2}

%==============================================================================
% PACKAGES
%==============================================================================

\usepackage[utf8]{inputenc}
\usepackage[T1]{fontenc}
\usepackage{amsmath}
\usepackage{amssymb}
\usepackage{mathrsfs}
\usepackage{braket}
\usepackage{amsthm}
\usepackage{hyperref}
\usepackage{cleveref}
% natbib loaded by revtex4-2 automatically
\usepackage{geometry}
\usepackage{setspace}
\usepackage{graphicx}
\usepackage{float}
\usepackage{booktabs}
\usepackage{listings}
\usepackage{dnd_shared}

%==============================================================================
% HYPERREF CONFIGURATION
%==============================================================================

\hypersetup{
    colorlinks=true,
    linkcolor=blue,
    citecolor=blue,
    urlcolor=blue,
    bookmarksnumbered=true,
    pdftitle={D-ND Quantum Information Engine: Modified Quantum Gates and Computational Framework},
    pdfauthor={D-ND Research Collective},
    pdfsubject={Quantum Computing, D-ND Gates, Possibilistic Density},
    pdfkeywords={possibilistic quantum information, D-ND gates, universal gate sets, IFS, quantum error correction, emergence-assisted computing}
}

%==============================================================================
% GEOMETRY
%==============================================================================

\geometry{
    letterpaper,
    top=1in,
    bottom=1in,
    left=1in,
    right=1in
}

%==============================================================================
% LISTINGS STYLE (for pseudocode)
%==============================================================================

\lstset{
    basicstyle=\small\ttfamily,
    frame=single,
    breaklines=true,
    numbers=left,
    numberstyle=\tiny,
    xleftmargin=2em,
    framexleftmargin=1.5em
}

%==============================================================================
% CUSTOM COMMANDS (Paper F specific)
%==============================================================================

\newcommand{\Mdist}{M_{\text{dist}}}
\newcommand{\Ment}{M_{\text{ent}}}
\newcommand{\Mproto}{M_{\text{proto}}}
\newcommand{\HDND}{H_{\text{DND}}}
\newcommand{\CNOTDND}{\text{CNOT}_{\text{DND}}}
\newcommand{\PDND}{P_{\text{DND}}}
\newcommand{\Rlinear}{R_{\text{linear}}}
\newcommand{\Remit}{R_{\text{emit}}}
\newcommand{\CNOTstd}{\text{CNOT}_{\text{std}}}

% Conjecture environment
\theoremstyle{plain}
\newtheorem{conjecture}[theorem]{Conjecture}
\newtheorem{openproblem}[theorem]{Open Problem}

%==============================================================================
% DOCUMENT
%==============================================================================

\begin{document}

\title{D-ND Quantum Information Engine:\\Modified Quantum Gates and Computational Framework}

\author{D-ND Research Collective}
\affiliation{Independent Research}
\date{February 14, 2026}

\begin{abstract}
We formalize the quantum-computational aspects of the D-ND (Dual-Non-Dual) framework
by introducing a possibilistic quantum information architecture that generalizes standard
quantum mechanics. Rather than pure probabilistic superposition, D-ND quantum states are
characterized by a \emph{possibilistic density} measure $\rhoDND$ incorporating emergence
structure, nonlocal coupling, and topological invariants. We define four modified quantum
gates---$\HDND$, $\CNOTDND$, $\PDND$, and Shortcut$_{\text{DND}}$---that preserve D-ND
structure while enabling practical computation. We prove that
$\{\HDND, \CNOTDND, \PDND\}$ form a universal gate set in the perturbative regime by
deriving arbitrary SU$(2^n)$ unitaries from gate compositions. A complete circuit model
with error analysis and coherence preservation guarantees is presented. We develop a
simulation framework based on Iterated Function Systems (IFS) with pseudocode and
polynomial complexity analysis. We position D-ND computation within known quantum advantage
results (BQP vs.\ BPP), showing how emergence-assisted error suppression provides a distinct
pathway to quantum speedup. Applications to quantum search algorithms and topological quantum
computing are discussed. This work bridges quantum information theory and emergence-theoretic
dynamics, establishing D-ND as a viable computational paradigm for near-term hybrid
quantum-classical algorithms.
\end{abstract}

\keywords{Possibilistic quantum information, D-ND gates, universal gate sets, Iterated
Function Systems, quantum error correction, emergence-assisted computing, BQP complexity,
topological quantum computing}

\maketitle

%==============================================================================
\section{Introduction}
\label{sec:intro}
%==============================================================================

Quantum computing has achieved remarkable theoretical and experimental progress, yet
fundamental limitations persist: decoherence, measurement collapse, and the Born rule's
strict probabilistic interpretation constrain the space of algorithms and applications.
The D-ND framework (developed in Papers A--E) proposes that quantum systems need not be
purely probabilistic; instead, \emph{possibility} can coexist with probability, mediated
through emergence and nonlocal coupling.

\subsection{Notation Clarification}
\label{sec:notation}

Throughout this paper, the emergence coupling coefficient $\lambda$ (without subscript)
represents the linear approximation parameter quantifying the strength of D-ND quantum
gate modifications relative to standard quantum operations. This is distinguished from:
Paper~A's $\lambdak$: eigenvalues of the emergence operator in the quantum substrate;
Paper~B's $\lambdaDND$: potential coupling constant in the dual-non-dual Hamiltonian;
Paper~D's $\lambdaauto$: autological convergence rate in observer dynamics;
Paper~E's $\lambdacosmo$: cosmological emergence coupling.
The notation is clarified further in \S\ref{sec:paperA-connection} where
$\lambda = M(t)$ (the emergence measure) during the linear approximation regime.

\subsection{Motivations}

\begin{enumerate}
\item \textbf{Beyond Probabilism}: Standard quantum mechanics treats all information as
probabilistic amplitudes. D-ND permits possibilistic states---superpositions where some
branches may be ``proto-actual'' (not yet fully actualized) or ``suppressed'' by emergence
dynamics.

\item \textbf{Nonlocal Emergence}: Rather than viewing nonlocality as spooky action at a
distance, D-ND models it as structure in the emergence field $\emerge$. Quantum gates can
be designed to exploit this structure.

\item \textbf{Topological Robustness}: D-ND incorporates topological invariants (homological
cycles, Betti numbers) that provide natural error correction and gate fidelity improvements.

\item \textbf{Hybrid Classical-Quantum}: The linear simulation framework allows efficient
classical emulation of certain D-ND circuits, reducing hardware requirements.

\item \textbf{Quantum Advantage Through Emergence}: Unlike standard approaches that rely
solely on quantum superposition, D-ND offers emergence-assisted error suppression, a novel
pathway to quantum advantage.
\end{enumerate}

\subsection{Paper Structure}

Section~\ref{sec:framework} introduces the possibilistic density measure and its relationship
to standard quantum states. Section~\ref{sec:gates} defines the four core modified gates with
rigorous composition rules. Section~\ref{sec:circuit} develops the circuit model and error
analysis. Section~\ref{sec:simulation} presents the IFS-based simulation framework with
pseudocode. Section~\ref{sec:applications} sketches applications, compares with known quantum
advantage results, and establishes a computational bridge to the THRML/Omega-Kernel library
by Extropic AI. Section~\ref{sec:conclusions} concludes. Appendices~\ref{app:prop22} and
\ref{app:prop43} provide proofs of key propositions.

%==============================================================================
\section{D-ND Quantum Information Framework}
\label{sec:framework}
%==============================================================================

\subsection{Possibilistic Density $\rhoDND$}
\label{sec:rho-dnd}

In standard quantum mechanics, the state of a system is given by a density matrix
$\rho \in \mathcal{L}(\mathcal{H})$, where $\mathcal{L}(\mathcal{H})$ is the space of
bounded linear operators on Hilbert space $\mathcal{H}$. D-ND generalizes this to a
\emph{possibilistic density} by incorporating emergence.

\begin{definition}[Possibilistic Density --- Formula B10]
\label{def:rho-dnd}
Let $\Mdist$, $\Ment$, $\Mproto$ be three non-negative real-valued measures on the
Hilbert space basis states:
\begin{itemize}
\item $\Mdist$: \emph{distributive capacity} (how ``spread'' the state is across basis elements)
\item $\Ment$: \emph{entanglement strength} (degree of nonlocal correlation structure)
\item $\Mproto$: \emph{proto-actualization measure} (how ``ready'' a branch is to become classical)
\end{itemize}
Then the \textbf{possibilistic density} is:
\begin{equation}
\rhoDND = \frac{\Mdist + \Ment + \Mproto}{\sum_{\text{all states}} (\Mdist + \Ment + \Mproto)}
= \frac{M}{\Sigma M}
\label{eq:rho-dnd}
\end{equation}
where $M = \Mdist + \Ment + \Mproto$ and $\Sigma M$ is the total measure across the system.
\end{definition}

\textbf{Interpretation:}
Each component of $M$ represents a different aspect of ``being available to computation'':
$\Mdist$ accounts for superposition breadth (analogous to Shannon entropy in the possibility space);
$\Ment$ captures nonlocal structure (branches participating in long-range correlations have higher
$\Ment$); and $\Mproto$ measures how close a branch is to classical actuality.

\textbf{Remark on Measure Independence and Operational Content.}
Definition~\ref{def:rho-dnd} requires three measures whose definitions must be operationally grounded:
\begin{enumerate}
\item \textbf{$\Mdist$ (Distributive Capacity)}: The Shannon entropy of the probability distribution
over basis states,
\begin{equation}
\Mdist = -\sum_i p_i \log p_i
\end{equation}
where $p_i = |\langle i|\psi\rangle|^2$ are the basis state probabilities.

\item \textbf{$\Ment$ (Entanglement Strength)}: For bipartite systems, the negativity
(Vidal \& Werner, 2002~\cite{Vidal2002}):
\begin{equation}
\Ment = \max(0, \text{Neg}(\rho_{AB})) = \max_k(0, -\lambda_k)
\end{equation}
where $\lambda_k$ are eigenvalues of the partial transpose.

\item \textbf{$\Mproto$ (Proto-Actualization Measure)}: Defined from Paper~A's emergence measure:
\begin{equation}
\Mproto(t) = 1 - M(t) = |\langle\text{NT}|U(t)\emerge|\text{NT}\rangle|^2
\end{equation}
\end{enumerate}

With these identifications, $\rhoDND$ is a genuine extension of standard density matrices,
carrying information---the proto-actualization trajectory $\Mproto(t)$---that standard quantum
states discard.

\subsection{Connection to Standard Quantum States}
\label{sec:standard-connection}

\begin{proposition}[Hilbert Space Embedding]
\label{prop:hilbert-embedding}
If $\Mproto \equiv 0$ (no proto-actualization, pure quantum regime) and $\mathcal{H}$ is
separable, then $\rhoDND$ defines a valid density operator via:
\begin{equation}
\hat{\rho}_{\text{DND}} = \sum_i \frac{M(i)}{\Sigma M} |i\rangle\langle i|
\end{equation}
where $M(i) = \Mdist(i) + \Ment(i)$ and $\Sigma M = \sum_i M(i)$. This satisfies:
(i)~$\text{Tr}[\hat{\rho}_{\text{DND}}] = 1$, (ii)~$\hat{\rho}_{\text{DND}} \geq 0$,
(iii)~$\hat{\rho}_{\text{DND}} = \hat{\rho}_{\text{DND}}^\dagger$.
The inner product
$\langle\psi|\phi\rangle_{\text{DND}} = \text{Tr}[|\psi\rangle\langle\phi|\hat{\rho}_{\text{DND}}]
= \sum_i a_i^* b_i \rho_{\text{DND}}(i)$
(where $|\psi\rangle = \sum_i a_i|i\rangle$, $|\phi\rangle = \sum_i b_i|i\rangle$) defines a
weighted Hilbert space structure that reduces to the standard inner product when $M(i)$ is uniform.
\end{proposition}

\emph{Proof}: See Appendix~\ref{app:prop22}.

\subsection{Connection to Paper A Emergence Measure}
\label{sec:paperA-connection}

Paper~A establishes the fundamental emergence measure
$M(t) = 1 - |\langle\text{NT}|U(t)\emerge|\text{NT}\rangle|^2$,
which quantifies the degree of state differentiation from the non-localized state $\NT$.

\begin{proposition}[$M(t)$ and Proto-Actualization]
\label{prop:M-proto}
The proto-actualization measure $\Mproto$ can be identified with the complement of the
Paper~A emergence measure:
\begin{equation}
\Mproto(t) = 1 - M(t) = |\langle\text{NT}|U(t)\emerge|\text{NT}\rangle|^2
\end{equation}
\end{proposition}

\textbf{Interpretation:}
When $M(t) = 0$ (early emergence): $\Mproto = 1$, meaning all modes remain proto-actual.
When $M(t) = 1$ (late emergence): $\Mproto = 0$, meaning all modes fully actualized.
The transition regime ($0 < M(t) < 1$) is the D-ND window where hybrid quantum-classical
behavior dominates.

\begin{proposition}[Distributive and Entanglement Measures]
\label{prop:M-decomp}
The three components satisfy the constraint:
\begin{equation}
\Mdist(t) + \Ment(t) = M(t), \qquad \Mproto(t) = 1 - M(t)
\end{equation}
so that $\Mdist(t) + \Ment(t) + \Mproto(t) = 1$. The emergence measure $M(t)$ governs
the partition: as emergence progresses, weight transfers from $\Mproto$ to $\Mdist + \Ment$.
\end{proposition}

\begin{proposition}[Reduction to Standard Quantum States]
\label{prop:reduction}
When $M(t) \to 1$ (equivalently $\Mproto \to 0$), the possibilistic density $\rhoDND$
reduces to a standard quantum state:
\begin{equation}
\lim_{M(t) \to 1} \rhoDND = \rho_{\text{standard}} =
\frac{\Mdist + \Ment}{\sum_{\text{states}}(\Mdist + \Ment)}
\end{equation}
which satisfies the Born rule probabilities under measurement.
\end{proposition}

\textbf{Remark on Circuit Implications:} In practical D-ND circuits, $\lambda = M(t)$.
Hence the linear approximation $\Rlinear(t) = P(t) + \lambda \cdot \Remit(t)$ is valid
during early emergence ($M(t) < 0.5$), where proto-actualization is dominant and the
classical component $P(t)$ is small.

%==============================================================================
\section{Modified Quantum Gates}
\label{sec:gates}
%==============================================================================

We define four fundamental gates adapted to the D-ND framework. Each gate:
(1)~preserves the structure of $\rhoDND$;
(2)~incorporates feedback from the emergence field $\emerge$;
(3)~reduces to standard gates when $\Mproto \to 0$.

\subsection{Hadamard$_{\text{DND}}$ (Formula C1)}
\label{sec:hadamard}

The standard Hadamard $H$ creates equal superposition: $H|0\rangle = (|0\rangle + |1\rangle)/\sqrt{2}$.

\begin{definition}
\label{def:hadamard-dnd}
The \textbf{Hadamard$_{\text{DND}}$} gate modifies the redistribution of density by coupling
to graph-theoretic emergence structure:
\begin{equation}
\HDND |v\rangle = \frac{1}{\mathcal{N}_v}
\sum_{u \in \text{Nbr}(v)} w_u \cdot \delta V_u \, |u\rangle
\label{eq:hadamard-dnd}
\end{equation}
where $v$ is a vertex in the emergence graph (state label), $\delta V_u$ is the
emergence-field potential gradient at neighbor $u$, $w_u$ is the emergence weight
(eigenvalue of $\emerge$ at $u$), $\text{Nbr}(v)$ is the neighborhood of $v$,
and $\mathcal{N}_v = \sqrt{\sum_{u \in \text{Nbr}(v)} |w_u \cdot \delta V_u|^2}$
is the normalization factor ensuring unitarity.
\end{definition}

\textbf{Physical Interpretation:}
Rather than creating uniform superposition, $\HDND$ weights each neighbor according to its
emergence ``readiness'' ($w_u$) and the local potential gradient. High $\delta V$ concentrates
the superposition; low $\delta V$ allows fuller spread. The normalization $\mathcal{N}_v$
ensures $\|\HDND|v\rangle\| = 1$.

\textbf{Remark on unitarity:} When the emergence field is static and the graph is regular,
$\HDND$ reduces to the standard Hadamard. For general emergence graphs, $\HDND$ is unitary
by construction but is not generally self-adjoint. $\HDND^2 = I$ holds only in the symmetric case.

\subsection{CNOT$_{\text{DND}}$ with Nonlocal Emergence (Formula C2)}
\label{sec:cnot}

\begin{definition}
\label{def:cnot-dnd}
The \textbf{CNOT$_{\text{DND}}$} gate incorporates nonlocal emergence coupling:
\begin{equation}
\CNOTDND = \CNOTstd \cdot e^{-i\,s\,\ell^*}
\label{eq:cnot-dnd}
\end{equation}
where $\CNOTstd = \left(\begin{smallmatrix} I & 0 \\ 0 & X \end{smallmatrix}\right)$ is the standard
CNOT gate, $s = \frac{1}{n}\sum_{i \neq j} |\langle i|H|j\rangle|$ is the nonlocal spreading
parameter, and $\ell^* = 1 - \delta V$ is the emergence-coherence factor with
$\delta V = \|\nabla\emerge\|/\|\emerge\| \in [0,1]$.
\end{definition}

\textbf{Effect:}
The phase factor $e^{-is\ell^*}$ applies a global nonlocal phase depending on both the
spreading rate $s$ and the coherence factor $\ell^*$. When $\delta V$ is high (strong emergence),
$\ell^*$ is small and the gate approaches standard CNOT. When $\delta V$ is low, the full
nonlocal phase is applied.

\textbf{Composition:} $\CNOTDND^2 = e^{-2is\ell^*} \cdot I$ (involutory up to a global phase).

\subsection{Phase$_{\text{DND}}$ with Potential Fluctuation Coupling (Formula C3)}
\label{sec:phase}

\begin{definition}
\label{def:phase-dnd}
The \textbf{Phase$_{\text{DND}}$} gate couples phase dynamics to emergence potential:
\begin{equation}
\PDND(\phi)|v\rangle = e^{-i(1 - \phi_{\text{phase}} \cdot \delta V)} |v\rangle
\label{eq:phase-dnd}
\end{equation}
where $\phi_{\text{phase}}$ is the classical phase parameter and $\delta V$ is the emergence
potential gradient at $v$.
\end{definition}

\textbf{Interpretation:}
The effective phase depends on the emergence potential. In strong emergence ($\delta V \to 1$),
the phase is suppressed. In weak emergence, the full phase is applied. This creates a
potential-dependent phase landscape exploitable for topological computation.

\subsection{Shortcut$_{\text{DND}}$ for Topological Operations (Formula C4)}
\label{sec:shortcut}

\begin{definition}[Circuit Depth Reduction Principle]
\label{def:shortcut-dnd}
Given a target entanglement structure on $m$ qubits (normally requiring $|E|$ CNOT operations),
the topological compression factor $\chi \in (0, 1]$ derived from the first Betti number of
the emergence graph determines the reduced gate count:
\begin{equation}
m_{\text{reduced}} = \lceil \chi \cdot |E| \rceil, \qquad
\chi = \frac{\beta_1(G_\emerge)}{\beta_1(G_\emerge) + |E|}
\label{eq:shortcut}
\end{equation}
where $\beta_1(G_\emerge)$ is the first Betti number of the emergence graph.
\end{definition}

\textbf{Remark:} Shortcut$_{\text{DND}}$ is not a single unitary gate but a circuit compilation
strategy: it specifies how to rearrange $\CNOTDND$ gates using topological information to reduce
circuit depth. The resulting circuit implements the same entanglement structure with fewer gates.

\subsection{Gate Universality (Perturbative Regime)}
\label{sec:universality}

\begin{proposition}[Gate Universality]
\label{prop:universality}
In the weak-emergence regime ($\delta V \ll 1$), the set $\{\HDND, \CNOTDND, \PDND\}$ forms
a universal quantum gate set for D-ND circuits: for any unitary $U \in \text{SU}(2^n)$, there
exists a finite sequence of gates from this set that approximates $U$ to arbitrary precision.
\end{proposition}

\begin{proof}
\textbf{Standard universality:} $\{H, \text{CNOT}, P(\pi/4)\}$ forms a universal gate set
(Nielsen \& Chuang~\cite{Nielsen2010}; Kitaev-Solovay theorem).

\textbf{Limiting reduction:} When $\delta V \to 0$, the D-ND gates reduce to standard gates:
$\HDND \to H$, $\CNOTDND \to \text{CNOT}$, $\PDND \to P(\phi)$ (from
Definitions~\ref{def:hadamard-dnd}--\ref{def:phase-dnd}).

\textbf{Perturbative extension:} For small $\delta V > 0$, each D-ND gate differs from its
standard counterpart by $O(\delta V)$: $\|G_{\text{DND}} - G_{\text{standard}}\| = O(\delta V)$.
The composition of $N$ gates accumulates error at most $N \cdot O(\delta V)$. Since the standard
gate set is universal and the perturbations are smooth, the D-ND gate set remains dense in
SU$(2^n)$ for sufficiently small $\delta V$.

\textbf{Error bound:} For a circuit of $N$ gates at emergence strength $\delta V$:
$\varepsilon_{\text{approx}} \leq N \cdot C \cdot \delta V$, where $C$ depends on gate
geometry. Choosing $\delta V < \varepsilon_{\text{target}}/(N \cdot C)$ achieves the desired
precision.
\end{proof}

\begin{openproblem}[Strong-Emergence Universality]
Whether $\{\HDND, \CNOTDND, \PDND\}$ remains universal for arbitrary $\delta V \in (0, 1]$
is an open question. A constructive proof would require explicit parametric families of
universal gate decompositions over $\delta V$, or a topological argument showing the gate set
generates a dense subgroup of SU$(2^n)$ for all $\delta V$.
\end{openproblem}

%==============================================================================
\section{Circuit Model}
\label{sec:circuit}
%==============================================================================

\subsection{D-ND Circuit Composition Rules}

A \textbf{D-ND circuit} $C$ is a sequence of gates $\{G_1, G_2, \ldots, G_k\}$ acting on
$\rhoDND$, with composition:
\begin{equation}
C(\rhoDND) = G_k \circ G_{k-1} \circ \cdots \circ G_1(\rhoDND)
\end{equation}

\textbf{Constraint 4.1 (Emergence Consistency):} Between consecutive gates $G_i$ and $G_{i+1}$,
the emergence field $\emerge$ must satisfy:
\begin{equation}
\text{spec}(\emerge_i) \cap \text{spec}(\emerge_{i+1}) \neq \emptyset
\end{equation}
ensuring continuity of the emergence landscape.

\textbf{Constraint 4.2 (Coherence Preservation):} The total coherence loss is bounded:
\begin{equation}
\sum_{i=1}^{k} (1 - \ell_i^*) \leq \Lambda_{\max}
\end{equation}
where $\Lambda_{\max}$ is the maximum allowed coherence budget.

\subsection{Error Model and Coherence Preservation}
\label{sec:error-model}

\begin{proposition}[Emergence-Assisted Error Suppression]
\label{prop:error-suppression}
Let $C$ be a D-ND circuit of $k$ gates with emergence-dependent Lindblad operators
$L_k^{\text{DND}}(t) = L_k \cdot (1 - M(t))$. Then the per-gate error rate is suppressed
linearly:
\begin{equation}
\varepsilon(t) = \varepsilon_0 \cdot (1 - M(t))
\label{eq:error-suppression}
\end{equation}
and the total circuit fidelity satisfies:
\begin{equation}
F_{\text{total}} = \prod_{i=1}^{k} [1 - \varepsilon_0(1 - M(t_i))]
\geq (1 - \varepsilon_0)^{k(1-\bar{M})}
\label{eq:fidelity}
\end{equation}
where $\bar{M} = (1/k)\sum_i M(t_i)$ is the average emergence factor.
\end{proposition}

\emph{Proof}: See Appendix~\ref{app:prop43}.

\textbf{Implication:} D-ND circuits with strong average emergence ($\bar{M}$ close to 1) achieve
significant fidelity improvement over standard circuits. The suppression is linear per gate but
compounds favorably over deep circuits. This complements standard quantum error correction.

%==============================================================================
\section{Simulation Framework}
\label{sec:simulation}
%==============================================================================

\subsection{IFS (Iterated Function System) Approach}
\label{sec:ifs}

When emergence is strong, an Iterated Function System approximation becomes viable.

\begin{definition}
\label{def:ifs}
Let $\{f_1, f_2, \ldots, f_n\}$ be contraction maps on the space of densities
(Definition~\ref{def:rho-dnd}), with contraction factors $\{\lambda_1, \ldots, \lambda_n\}$
(each $\lambda_i < 1$). An IFS is:
\begin{equation}
\rho_{\text{DND}}^{(n+1)} = \sum_{i=1}^{n} p_i \, f_i(\rho_{\text{DND}}^{(n)})
\end{equation}
where $p_i$ are weights determined by the emergence graph structure.
\end{definition}

\textbf{Scope limitations:} The IFS framework applies specifically to D-ND circuits in the
linear emergence regime ($M(t) < 0.5$, $\lambda < 0.5$). We do \emph{not} claim that arbitrary
quantum circuits can be simulated polynomially classically. For full quantum circuits ($M(t) \to 1$),
standard BQP-hard simulation applies. The IFS structure emerges naturally from D-ND dynamics because
the emergence operator creates self-similar branching structures (Paper~C \S3.1). See
Barnsley~\cite{Barnsley1988} for the mathematical foundations of IFS.

\subsection{Linear Approximation $\Rlinear = P + \lambda \cdot R(t)$ (Formula C7)}
\label{sec:linear-approx}

For practical implementation:
\begin{equation}
\Rlinear(t) = P(t) + \lambda \cdot \Remit(t)
\label{eq:linear-approx}
\end{equation}
where $P(t)$ is the probabilistic component (standard quantum simulation with $\Mproto = 0$),
$\lambda$ is the emergence-coupling coefficient, and:
\begin{equation}
\Remit(t) = \int_0^t M(s) \, e^{-\gamma(t-s)} \, ds
\end{equation}
where $\gamma$ is the emergence-memory decay rate.

\subsection{Pseudocode for D-ND IFS Simulation}
\label{sec:pseudocode}

\begin{lstlisting}[caption={D-ND Quantum Circuit Simulation via IFS},label={lst:ifs}]
Input: rho_0, circuit C, time T, lambda, gamma, epsilon
Output: rho_final, measurement_stats

1. INITIALIZE
   P(0) <- rho_0
   M(0) <- ComputeEmergenceMeasure(rho_0)
   t <- 0, dt <- T / NumSteps

2. FOR each gate G_i in C:
   3. P(t+dt) <- StandardQuantumSim(P(t), G_i, dt)
   4. M(t+dt) <- M(t) + dt * dM/dt(t)
   5. R_emit(t+dt) <- exp(-gamma*dt)*R_emit(t)
                     + dt*M(t)
   6. dU_corr <- ExponentialMap(dV, lambda, ell*)
      P(t+dt) <- dU_corr * P(t+dt) * dU_corr_dag
   7. epsilon_eff <- eps_0 * (1 - M(t+dt))
   8. rho_DND(t+dt) <- P(t+dt) + lambda*R_emit(t+dt)
      Renormalize
   9. t <- t + dt

10. RETURN rho_DND(T), measurements
\end{lstlisting}

\textbf{Complexity:} $O(n^3 \cdot T)$ when $\lambda < 0.3$ (weak emergence),
$O(n^4 \cdot T)$ for moderate emergence, and $O(2^n \cdot T)$ for strong emergence
(standard simulation regime).

\subsection{Error Analysis of Linear Approximation}
\label{sec:error-analysis}

\begin{proposition}[Error Bound for Linear Approximation]
\label{prop:error-bound}
Let $R_{\text{exact}}(t)$ be the exact D-ND state evolution and
$\Rlinear(t) = P(t) + \lambda \cdot \Remit(t)$. Then:
\begin{equation}
\|R_{\text{exact}}(t) - \Rlinear(t)\| \leq C \cdot \lambda^2 \cdot \|\Remit(t)\|^2
\label{eq:error-bound}
\end{equation}
where $C \approx T \cdot \log(n) \cdot \rho_{\max}$ for a circuit of depth $T$ on $n$ qubits
with emergence spectrum bounded by $\rho_{\max}$.
\end{proposition}

\emph{Proof sketch:} The exact evolution satisfies
$R_{\text{exact}} = \mathcal{U}_{\text{full}} R(0)$ while the linear approximation uses
$\mathcal{U}_{\text{linear}} = \mathcal{U}_{\text{standard}} + \lambda\mathcal{U}_{\text{correction}}$.
The error $\Delta = \mathcal{U}_{\text{full}} - \mathcal{U}_{\text{linear}} = O(\lambda^2)$
by perturbation theory.

\textbf{Validity regime:} $M(t) < 0.5$ (early to mid-stage emergence). For $\lambda < 0.3$,
relative error remains below 1.2\%, suitable for NISQ applications. For $\lambda \geq 0.5$,
the linear approximation breaks down and full quantum simulation is required.

\subsection{Comparison with Standard Quantum Simulation}
\label{sec:comparison}

\begin{table}[H]
\centering
\begin{tabular}{lll}
\toprule
\textbf{Aspect} & \textbf{Standard} & \textbf{D-ND Linear} \\
\midrule
Time Complexity & $O(2^n \cdot T)$ & $O(n^3 \cdot T)$ when $\lambda < 0.3$ \\
Memory & $O(2^n)$ & $O(n^2)$ \\
Accuracy (low em.) & Perfect & $\sim$99\% \\
Hardware & Quantum processor & Classical + emergence oracle \\
Error handling & Circuit-level QEC & Emergence-assisted suppression \\
\bottomrule
\end{tabular}
\caption{Comparison of simulation approaches.}
\label{tab:comparison}
\end{table}

%==============================================================================
\section{Applications and Quantum Advantage}
\label{sec:applications}
%==============================================================================

\subsection{Quantum Search with Emergent Speedup}
\label{sec:quantum-search}

\textbf{Problem:} Search for a marked item in an unsorted database of size $N$.
Standard Grover's algorithm achieves $O(\sqrt{N})$ speedup.

\textbf{D-ND Enhancement:} By using $\HDND$ gates that preferentially weight
high-emergence branches, we can concentrate the possibilistic density on the marked item.

\begin{conjecture}
\label{conj:search}
For circuits where emergence is controlled ($\Mproto \propto t$), D-ND quantum search may
achieve a constant-factor improvement over standard Grover, with query complexity
$O(\sqrt{N}/\alpha)$ where $\alpha \geq 1$ is an emergence-amplification factor.
\end{conjecture}

\textbf{Remark on lower bounds:} The BBBV theorem (Bennett et al.~\cite{Bennett1997})
establishes that any quantum search requires $\Omega(\sqrt{N})$ oracle queries. Any D-ND
speedup beyond this bound would require a fundamentally different oracle model. The improvement
claimed here is a constant factor $\alpha$ within the standard oracle model.

\subsection{Topological Quantum Computing}
\label{sec:topological}

D-ND is naturally suited to topological quantum computing:
(1)~states are protected by topological invariants (homological cycles in the emergence graph);
(2)~braiding via Shortcut$_{\text{DND}}$ implements nonabelian anyon exchange efficiently;
(3)~the emergence field provides additional topological protection beyond intrinsic error suppression.

For moderate emergence, overhead reduction is:
\begin{equation}
\text{Overhead reduction} = 1 - \frac{\Mproto}{\Mdist + \Ment}
\end{equation}

\subsection{Positioning Within BQP vs.\ BPP}
\label{sec:bqp}

D-ND provides a distinct mechanism for quantum speedup:
\begin{enumerate}
\item \textbf{Emergence-Assisted Complexity:} The emergence measure $M(t)$ provides a
continuously controllable resource.
\item \textbf{Hybrid Complexity Class:} Define BQP$_{\text{DND}}$ as problems solvable by
D-ND circuits with polynomial emergence overhead.
\item \textbf{Error Suppression Advantage:} Proposition~\ref{prop:error-suppression} shows
$\varepsilon(t) = \varepsilon_0(1 - M(t))$, enabling deeper circuits with strong emergence.
\end{enumerate}

\subsection{Open Problem: Quantum Advantage via D-ND Amplitude Amplification}
\label{sec:open-problem}

\begin{openproblem}
Prove or disprove that D-ND quantum circuits can achieve superpolynomial speedup for a natural
problem class, using emergence-modulated amplitude amplification distinct from standard Grover.
\end{openproblem}

\textbf{Candidate approach:} Initialize to $\NT$. Apply emergence-modulated oracle
$O_{\text{DND}}(t) = I - (1 + M(t))|x^*\rangle\langle x^*|$ and diffusion operator
$D_{\text{DND}}(t) = (1-M(t)) D_{\text{Grover}} + M(t) D_{\text{random}}$. The iteration
count becomes:
\begin{equation}
T_{\text{DND}} \sim \frac{\sqrt{N/k}}{\sqrt{1 + \lambda\Psi_C}}
\end{equation}
where $\Psi_C$ is a coherence enhancement factor derived from the circuit structure. The total
query complexity remains $\Omega(\sqrt{N/k})$ by BBBV, so this should be understood as a
constant-factor improvement for fixed $n$.

\subsection{Connection to Thermodynamic Sampling: The THRML/Omega-Kernel Bridge}
\label{sec:thrml}

Recent developments in thermodynamic computing by Extropic AI provide a direct experimental
validation pathway for D-ND quantum information theory. The THRML/Omega-Kernel library
implements probabilistic graphical model sampling through thermodynamic principles, with
architecture isomorphic to the D-ND framework.

\subsubsection{SpinNode as D-ND Dipole}

The THRML SpinNode with states $\{-1, +1\}$ corresponds to the D-ND singular-dual dipole:
\begin{equation}
\text{SpinNode} \in \{-1, +1\} \leftrightarrow
\text{D-ND dipole} \in \{|\varphi_+\rangle, |\varphi_-\rangle\}
\end{equation}
The Ising energy-based model $E = -\sum_{i,j} J_{ij}s_is_j - \sum_i h_is_i$ maps to
the D-ND effective potential $V_{\text{eff}}$ with $J_{ij}$ as interaction Hamiltonian and
$h_i$ as single-particle potential.

\subsubsection{Block Gibbs Sampling as Iterative Emergence}

THRML's block Gibbs sampling---dividing the graph into alternating blocks with conditional
updates---is isomorphic to the D-ND emergence process: initial random state $\leftrightarrow$
sampling from $\NT$; each Gibbs sweep $\leftrightarrow$ one application of $\emerge$;
convergence to equilibrium $\leftrightarrow$ full emergence with $M \approx 1$.

Each Gibbs sweep samples $p(s_B|s_{B^c}) \propto \exp(-\beta E(s_B, s_{B^c}))$, where the
Boltzmann factor $\exp(-\beta E)$ corresponds to the emergence operator's selective amplification
of high-coherence branches.

\subsubsection{Gate Correspondence}

The four D-ND gates map to THRML operations:
$\HDND \leftrightarrow$ block redistribution;
$\CNOTDND \leftrightarrow$ inter-block conditional update;
$\PDND \leftrightarrow$ temperature/bias modulation;
Shortcut$_{\text{DND}} \leftrightarrow$ multi-block simultaneous update.

\subsubsection{Significance for Experimental Validation}

The THRML framework provides a direct experimental validation pathway:
(1)~existing running codebase (JAX, GPU-accelerated);
(2)~thermodynamic hardware roadmap (Extropic AI processors);
(3)~hybrid classical-quantum bridge;
(4)~emergence verification via conditional probability distributions;
(5)~algorithm compatibility (Grover, VQE, QAOA variants).

\subsection{Simulation Metrics from D-ND Hybrid Framework}
\label{sec:metrics}

Four key metrics quantify the hybrid quantum-classical transition:

\textbf{Coherence Measure:}
$C(t) = |\langle\Psi(t)|\Psi(0)\rangle|^2 = \text{Tr}[\rho(t)\rho(0)]$.
When $C(t) = 1$: perfect coherence. When $C(t) \to 0$: complete decoherence.

\textbf{Tension Measure:}
$T(t) = \|\partial\rho/\partial t\|^2 = \text{Tr}[(\dot{\rho})^\dagger\dot{\rho}]$.
High $T(t)$: active emergence. Low $T(t)$: equilibrium.

\textbf{Emergence Rate:}
$dM/dt = (d/dt)[1 - |\langle\text{NT}|U(t)\emerge|\text{NT}\rangle|^2]$.
Fast $dM/dt$: strong emergence coupling.

\textbf{Convergence Criterion:}
$|C(t) - C(t-1)| < \varepsilon$ for user-specified tolerance $\varepsilon$.

%==============================================================================
\section{Conclusions}
\label{sec:conclusions}
%==============================================================================

We have formalized the quantum-computational aspects of the D-ND framework:

\begin{enumerate}
\item \textbf{Possibilistic Density $\rhoDND$} unifies quantum superposition with emergence
structure, enabling a richer information space.

\item \textbf{Four Modified Gates} ($\HDND$, $\CNOTDND$, $\PDND$, Shortcut$_{\text{DND}}$)
provide a complete gate set adapted to D-ND dynamics.

\item \textbf{Gate Universality} (Proposition~\ref{prop:universality}) proves the gate set can
approximate arbitrary SU$(2^n)$ unitaries in the perturbative regime. Strong-emergence universality
remains an open problem.

\item \textbf{Emergence-Assisted Error Suppression} (Proposition~\ref{prop:error-suppression})
shows fidelity improvement with $\varepsilon(t) = \varepsilon_0(1-M(t))$, complementary to
standard QEC.

\item \textbf{Linear Simulation Framework} enables polynomial-time classical approximation
when $\lambda < 0.3$, reducing hardware requirements.

\item \textbf{Applications} to quantum search (constant-factor improvement), topological QC
(reduced overhead), and the THRML thermodynamic computing bridge are demonstrated.
\end{enumerate}

\textbf{Future Directions:}
Hardware implementation on superconducting qubits;
D-ND algorithm library for optimization and machine learning;
efficient emergence oracle realization;
integration with variational quantum algorithms;
experimental validation on NISQ devices.

%==============================================================================
\section*{Acknowledgments}
%==============================================================================

This work builds on the D-ND theoretical framework developed in Papers A--E. The authors
thank the quantum information and emergence dynamics research communities for foundational insights.

%==============================================================================
% APPENDICES
%==============================================================================

\appendix

\section{Proof of Proposition~\ref{prop:hilbert-embedding}}
\label{app:prop22}

\begin{proof}
\textbf{Density operator construction:} When $\Mproto = 0$, we have $M(i) = \Mdist(i) + \Ment(i) \geq 0$
for each basis state $|i\rangle$. Define $\Sigma M = \sum_i M(i) > 0$. Then:
\begin{equation}
\hat{\rho}_{\text{DND}} = \sum_i \frac{M(i)}{\Sigma M} |i\rangle\langle i|
\end{equation}

\textbf{Density matrix properties:}
(i)~$\text{Tr}[\hat{\rho}_{\text{DND}}] = \sum_i M(i)/\Sigma M = 1$;
(ii)~all eigenvalues $M(i)/\Sigma M \geq 0$;
(iii)~$\hat{\rho}_{\text{DND}}$ is diagonal in a real basis, hence self-adjoint.

\textbf{Weighted inner product:} For $|\psi\rangle = \sum_i a_i|i\rangle$ and
$|\phi\rangle = \sum_j b_j|j\rangle$:
\begin{equation}
\langle\psi|\phi\rangle_{\text{DND}} = \text{Tr}[|\psi\rangle\langle\phi|\hat{\rho}_{\text{DND}}]
= \sum_i a_i^* b_i \frac{M(i)}{\Sigma M}
\end{equation}

\textbf{Hilbert space verification:}
Sesquilinearity follows from linearity of trace and sum.
Conjugate symmetry: $\langle\psi|\phi\rangle^*_{\text{DND}} = \sum_i a_i b_i^* M(i)/\Sigma M
= \langle\phi|\psi\rangle_{\text{DND}}$.
Positive-definiteness: $\langle\psi|\psi\rangle_{\text{DND}} = \sum_i |a_i|^2 M(i)/\Sigma M \geq 0$,
with equality iff $a_i = 0$ for all $i$ in the support of $M$.

\textbf{Born rule recovery:} $P(i) = \langle i|\hat{\rho}_{\text{DND}}|i\rangle = M(i)/\Sigma M$.
When $M(i)$ is uniform, the weighted inner product reduces to the standard inner product.
\end{proof}

\section{Proof of Proposition~\ref{prop:error-suppression}}
\label{app:prop43}

\begin{proof}
\textbf{Emergence-dependent Lindblad equation:}
The evolution with decoherence and emergence coupling follows:
\begin{equation}
\frac{d\rho}{dt} = -\frac{i}{\hbar}[H, \rho] + \mathcal{D}_{\text{DND}}[\rho]
\end{equation}
where $\mathcal{D}_{\text{DND}}[\rho] = \sum_k (L_k^{\text{DND}} \rho (L_k^{\text{DND}})^\dagger
- \tfrac{1}{2}\{(L_k^{\text{DND}})^\dagger L_k^{\text{DND}}, \rho\})$ with
$L_k^{\text{DND}}(t) = L_k \cdot (1 - M(t))$.

\textbf{Per-gate error:} The effective Lindblad rate scales as $(1-M(t))^2$ at leading order.
For $\varepsilon_0 \ll 1$, the per-gate error is $\varepsilon(t) = \varepsilon_0(1-M(t))$
(from $\|L_k^{\text{DND}}\| = (1-M(t))\|L_k\|$).

\textbf{Circuit fidelity:} Per-gate fidelity $F_i = 1 - \varepsilon_0(1-M(t_i))$.
For $k$ gates:
\begin{equation}
\ln F_{\text{total}} = \sum_{i=1}^{k} \ln[1 - \varepsilon_0(1-M(t_i))]
\approx -\varepsilon_0 \sum_{i=1}^{k} (1-M(t_i)) = -\varepsilon_0 k(1-\bar{M})
\end{equation}
Thus $F_{\text{total}} \approx e^{-\varepsilon_0 k(1-\bar{M})}$.

\textbf{Comparison:} Standard circuit ($M=0$): $F_{\text{std}} \approx e^{-\varepsilon_0 k}$.
D-ND fidelity improvement: $F_{\text{DND}}/F_{\text{std}} = e^{\varepsilon_0 k\bar{M}}$.

\textbf{Kraus representation:} The Kraus operators
$K_0 = \sqrt{1-\varepsilon_0(1-M(t))}I$ and
$K_j = \sqrt{\varepsilon_0(1-M(t))/3}\,\sigma_j$ ($j=1,2,3$) satisfy completeness
$\sum_j K_j^\dagger K_j = I$ and confirm the error probability $\varepsilon_0(1-M(t))$ per gate.
\end{proof}

%==============================================================================
% REFERENCES
%==============================================================================

\begin{thebibliography}{20}

\bibitem{Dirac1930}
P.~A.~M. Dirac,
\emph{The Principles of Quantum Mechanics} (Oxford University Press, 1930).

\bibitem{Nielsen2010}
M.~A. Nielsen and I.~L. Chuang,
\emph{Quantum Computation and Quantum Information} (Cambridge University Press, 2010).

\bibitem{Aharonov1997}
D.~Aharonov and M.~Ben-Or,
``Fault-tolerant quantum computation with constant error,''
\emph{SIAM J. Comput.} \textbf{38}(4), 1207--1282 (1997).

\bibitem{Nayak2008}
C.~Nayak, S.~H. Simon, A.~Stern, M.~Freedman, and S.~Das Sarma,
``Non-Abelian anyons and topological quantum computation,''
\emph{Rev. Mod. Phys.} \textbf{80}(3), 1083 (2008).

\bibitem{AspuruGuzik2005}
A.~Aspuru-Guzik, A.~D. Dutoi, P.~J. Love, and M.~Head-Gordon,
``Simulated quantum computation of molecular energies,''
\emph{Science} \textbf{309}(5741), 1704--1707 (2005).

\bibitem{Harrow2009}
A.~W. Harrow, A.~Hassidim, and S.~Lloyd,
``Quantum algorithm for linear systems of equations,''
\emph{Phys. Rev. Lett.} \textbf{103}(15), 150502 (2009).

\bibitem{Grover1997}
L.~K. Grover,
``Quantum mechanics helps in searching for a needle in a haystack,''
\emph{Phys. Rev. Lett.} \textbf{79}(2), 325 (1997).

\bibitem{Kitaev2003}
A.~Y. Kitaev,
``Fault-tolerant quantum computation by anyons,''
\emph{Ann. Phys.} \textbf{303}(1), 2--30 (2003).

\bibitem{PaperA}
D-ND Research Collective,
``Quantum Emergence from Primordial Potentiality: The Dual-Non-Dual Framework for State Differentiation''
(this volume).

\bibitem{PaperE}
D-ND Research Collective,
``Cosmological Extension of the Dual-Non-Dual Framework''
(this volume).

\bibitem{Hutchinson1981}
J.~E. Hutchinson,
``Fractals and self-similarity,''
\emph{Indiana Univ. Math. J.} \textbf{30}(5), 713--747 (1981).

\bibitem{Falconer1990}
K.~J. Falconer,
\emph{Fractal Geometry: Mathematical Foundations and Applications} (John Wiley \& Sons, 1990).

\bibitem{Shor1997}
P.~W. Shor,
``Polynomial-time algorithms for prime factorization and discrete logarithms on a quantum computer,''
\emph{SIAM J. Comput.} \textbf{26}(5), 1484--1509 (1997).

\bibitem{Preskill2018}
J.~Preskill,
``Quantum computing in the NISQ era and beyond,''
\emph{Quantum} \textbf{2}, 79 (2018).

\bibitem{Wootters1998}
W.~K. Wootters,
``Entanglement of formation of an arbitrary state of two qubits,''
\emph{Phys. Rev. Lett.} \textbf{80}(10), 2245 (1998).

\bibitem{Vidal2002}
G.~Vidal and R.~F. Werner,
``Computable measure of entanglement,''
\emph{Phys. Rev. A} \textbf{65}(3), 032314 (2002).

\bibitem{Barnsley1988}
M.~F. Barnsley,
\emph{Fractals Everywhere} (Academic Press, 1988).

\bibitem{Bennett1997}
C.~H. Bennett, E.~Bernstein, G.~Brassard, and U.~Vazirani,
``Strengths and weaknesses of quantum computing,''
\emph{SIAM J. Comput.} \textbf{26}(5), 1510--1523 (1997).

\bibitem{Lupasco1951}
S.~Lupasco,
\emph{Le principe d'antagonisme et la logique de l'\'energie} (Hermann, 1951).

\bibitem{Nicolescu2002}
B.~Nicolescu,
\emph{Manifesto of Transdisciplinarity} (SUNY Press, 2002).

\end{thebibliography}

\end{document}
